%%% The main file. It contains definitions of basic parameters and includes all other parts.

%% Settings for single-side (simplex) printing
% Margins: left 40mm, right 25mm, top and bottom 25mm
% (but beware, LaTeX adds 1in implicitly)
\documentclass[12pt,a4paper]{report}
\setlength\textwidth{145mm}
\setlength\textheight{247mm}
\setlength\oddsidemargin{15mm}
\setlength\evensidemargin{15mm}
\setlength\topmargin{0mm}
\setlength\headsep{0mm}
\setlength\headheight{0mm}
% \openright makes the following text appear on a right-hand page
\let\openright=\clearpage

\setlength{\emergencystretch}{10cm}
\pretolerance=10000
\tolerance=10000
\hfuzz=5000pt
\vfuzz=5000pt 

%% Settings for two-sided (duplex) printing
% \documentclass[12pt,a4paper,twoside,openright]{report}
% \setlength\textwidth{145mm}
% \setlength\textheight{247mm}
% \setlength\oddsidemargin{14.2mm}
% \setlength\evensidemargin{0mm}
% \setlength\topmargin{0mm}
% \setlength\headsep{0mm}
% \setlength\headheight{0mm}
% \let\openright=\cleardoublepage

%% Generate PDF/A-2u
\usepackage[a-2u]{pdfx}

%% Character encoding: usually latin2, cp1250 or utf8:
\usepackage[utf8]{inputenc}

%% Prefer Latin Modern fonts
\usepackage{lmodern}

%% Further useful packages (included in most LaTeX distributions)
\usepackage{amsmath}        % extensions for typesetting of math
\usepackage{amsfonts}       % math fonts
\usepackage{amsthm}         % theorems, definitions, etc.
\usepackage{bbding}         % various symbols (squares, asterisks, scissors, ...)
\usepackage{bm}             % boldface symbols (\bm)
\usepackage{graphicx}       % embedding of pictures
\usepackage{fancyvrb}       % improved verbatim environment
\usepackage{natbib}         % citation style AUTHOR (YEAR), or AUTHOR [NUMBER]
\usepackage[nottoc]{tocbibind} % makes sure that bibliography and the lists
			    % of figures/tables are included in the table
			    % of contents
\usepackage{dcolumn}        % improved alignment of table columns
\usepackage{booktabs}       % improved horizontal lines in tables
\usepackage[usenames, dvipsnames]{xcolor}  % typesetting in color



% Packages added by Dahn
\usepackage[colorinlistoftodos,prependcaption,textsize=tiny,disable]{todonotes} % disable
\usepackage{xargs}
\usepackage{bbm}	% For lowercase blackboard bold
\usepackage{multicol}  	% ultiple columns
\usepackage{enumerate}
\usepackage{amssymb}	% \varnothing
\usepackage{booktabs}
\newcommand{\ra}[1]{\renewcommand{\arraystretch}{#1}} % For line stretching in booktabs
\usepackage{lscape}
\usepackage[overlay,absolute]{textpos}
\usepackage{pgf,tikz,pgfplots}
\pgfplotsset{compat=1.15}
\usetikzlibrary{arrows}
\usepackage{breakcites}

\newcommand\PlaceText[3]{%
\begin{textblock*}{10in}(#1,#2)  %% change width of box from 10in as you wish
#3
\end{textblock*}
}%
\textblockorigin{-5mm}{0mm} 

% Dahn commands
\newcommand{\Rd}{{\mathbb R^d}}
\newcommand{\Rt}{{\mathbb R^3}}

\newcommandx{\problem}[2][1=]{\todo[linecolor=red,backgroundcolor=red!25,bordercolor=red,#1]{#2}}
\newcommandx{\todoo}[2][1=]{\todo[linecolor=blue,backgroundcolor=blue!25,bordercolor=blue,#1]{#2}}
\newcommandx{\note}[2][1=]{\todo[linecolor=OliveGreen,backgroundcolor=OliveGreen!25,bordercolor=OliveGreen,#1]{#2}}
\newcommandx{\unsure}[2][1=]{\todo[linecolor=Plum,backgroundcolor=Plum!25,bordercolor=Plum,#1]{#2}}

\newtheorem{theorem}{Theorem}

\theoremstyle{definition}
\newtheorem{definition}{Definition}

\theoremstyle{remark}
\newtheorem{remark}{Remark}

\theoremstyle{theorem}
\newtheorem{proposition}{Proposition}

\theoremstyle{remark}
\newtheorem{ex}{Example}


\newcommand{\tbd}{\textbf{{\color{red}~ TO BE DONE ~}}}

\newcommand{\x}{{\mathbf{x}}}
\newcommand{\p}{{\mathbbm{p}}}

%% Augumented matrix 
\newenvironment{amatrix}[1]{%
  \left(\begin{array}{@{}*{#1}{c}|c@{}}
}{%
  \end{array}\right)
}


\newenvironment{tight_enumerate}{
\begin{enumerate}
  \setlength{\itemsep}{0pt}
  \setlength{\parskip}{0pt}
}{\end{enumerate}}

%%% Basic information on the thesis

% Thesis title in English (exactly as in the formal assignment)

\def\ThesisTitle{Generalized random tessellations, their properties, simulation and applications}

% Author of the thesis
\def\ThesisAuthor{Daniel Jahn}

% Year when the thesis is submitted
\def\YearSubmitted{2019}

% Name of the department or institute, where the work was officially assigned
% (according to the Organizational Structure of MFF UK in English,
% or a full name of a department outside MFF)
\def\Department{Department of Probability and Mathematical Statistics}

% Is it a department (katedra), or an institute (ústav)?
\def\DeptType{Department}

% Thesis supervisor: name, surname and titles
\def\Supervisor{prof. RNDr. Viktor Bene\v{s}, DrSc.}

% Supervisor's department (again according to Organizational structure of MFF)
\def\SupervisorsDepartment{Department of Probability and Mathematical Statistics}

% Study programme and specialization
\def\StudyProgramme{Mathematics}

% An optional dedication: you can thank whomever you wish (your supervisor,
% consultant, a person who lent the software, etc.)
\def\Dedication{%
I would like to thank my supervisor prof. RNDr. Viktor Bene\v{s}, DrSc. for all the time and effort he put into helping me on the often thorny path of writing this thesis. I am also grateful for all the opportunities he provided me with to present and discuss my work on seminars and conferences.

If there is one other person without whom this thesis would have been much more difficult to write, it is my partner, Adela Nguy$\tilde{\hat{e}}$n. I am grateful for having her by my side throughout the whole process of writing this thesis. She has far surpassed the official requirements for the partner of a mathematician, that is to be emotionally supportive and understanding, especially when waking up in the middle of the night to find the said mathematician working, again. She has spent countless hours listening to me talk about circles and tetrahedrons and, on several occasions, picked up a pen and paper and actively helped me solve the problems. Thank you for everything.

For helping me with a difficult programming language I knew nothing about, as well as for general advice on programming, I thank my friend and an excellent programmer Jan Noha.

I thank Colin Beet for helping me improve the linguistic aspects of this thesis. He would like the reader to know that, contrary to what Remark \ref{r:abuse} claims, abuse of any kind can never be justified.

I thank my family for supporting my academic endeavours even though they still have no idea what it is I am actually doing. 

Finally, I'd like to thank the following people who have provided valuable insights and comments: Prof. David Dereudre, Jeremy Tan, Mat\v{e}j Kov\'{a}\v{r}, Prof. Volker Schmidt, Prof. David Eppstein.
}


% Abstract (recommended length around 80-200 words; this is not a copy of your thesis assignment!)
\def\Abstract{%
	The past few years have seen advances in modelling of polycrystalline materials using parametric tessellation models from stochastic geometry. A promising class of tessellations, the Gibbs-type tessellation, allows the user to specify a great variety of properties through the energy function. 
	This text focuses solely on Gibbs-type tetrahedrizations, a three-dimensional tessellation composed of tetrahedra. The existing results for two-dimensional Delaunay triangulations are extended to the case of three-dimensional Laguerre tetrahedrization. We provide a proof of existence, a \texttt{C++} implementation of the MCMC simulation and estimation of the models parameters through maximum pseudolikelihood.
}

% 3 to 5 keywords (recommended), each enclosed in curly braces
\def\Keywords{%
	{Gibbs point process} {tetrahedrization} {Laguerre}  {existence} {simulation} {maximum pseudolikelihood estimation}
}

%% The hyperref package for clickable links in PDF and also for storing
%% metadata to PDF (including the table of contents).
%% Most settings are pre-set by the pdfx package.
\hypersetup{unicode}
\hypersetup{breaklinks=true}

% Definitions of macros (see description inside)
\include{macros}

% Title page and various mandatory informational pages
\begin{document}
\include{title}


%%% A page with automatically generated table of contents of the master thesis
\tableofcontents



%%%% Glossary
\chapwithtoc{Glossary of terms and abbreviations}
\subsubsection{Mathematical operators and notation}
The following list only serves as a reference. Most of the following terms are defined more precisely when they are first used in the text.

\hspace*{-8mm}
\begin{tabular}{ l l }
	$\mathbb N_0$ & $\mathbb N \cup \{0\}$ \\
	$B(x,r)$ & Open ball with center $x$ and radius $r$ \\
	$\mathrm{diam}(A)$ & Diameter of the set $A$; $sup_{x,y\in A}\|x-y\|$ \\
	$\mathrm{card}(A)$ & Cardinality of the set $A$ \\
	$\mathrm{conv}(A)$ & Convex hull of the set $A$ \\
	$\lambda(A), |A|$ & Lebesgue measure of the set A \\
	$\partial$ & Topological boundary of a set \\
	$A+B$ & Minkowski sum of the sets $A$ and $B$; the set $\{a+b: a\in A, b\in B\}$. \\
	$\mathrm{pr}_{\Rt}(\cdot)$ & Projection from $\mathbb R^4$ to $\Rt$ \\
	$\angle pqr$ & Angle formed by $pqr$ \\
	$\propto$ & Proportional to \\
	$\delta_x(A)$ & Dirac delta function at $x$. Equals to 1 if $x\in A$, 0 otherwise.\\ 
	$\mathrm{det} M$ & Determinant of the matrix $M$ \\
	$\| \cdot \|$ & Euclidean norm \\
	$d(p,x)$ & Power distance of the point $x$ and weighted point $p$ \\
	$\rho(p,q)$ & Power product of two weighted points $p$ and $q$ \\
	$\chi(\eta)$ & For $\mathrm{card}(\eta)=4$, circumdiameter of the tetrahedron $\mathrm{conv}(\eta)$.\\
\end{tabular}




\subsubsection{Abbreviations}
\begin{tabular}{ l l }
	DLR & Dobrushin-Lanford-Ruelle (equations) \\
	GNZ & Georgii-Nguyen-Zessin (equation) \\
	MCMC & Markov chain Monte carlo \\
	MPLE & Maximum pseudolikelihood estimation
\end{tabular}




%%% Each chapter is kept in a separate file
\chapter*{Introduction}
\addcontentsline{toc}{chapter}{Introduction}
% Polycrystalline. As a part of their study, though not directly related to it, is the study of duals
% Extension of Practical - 3d and Laguerre
% Need for theoretical underpinning of the models, presented in chapters 1-3, culminating in the proofs of their existence using the hypergraph formalism from Dereudre
% We would also like to draw attention to chapter one


%Creating a standalone text about Laguerre tetrihedrization that does utilize the duality to Laguerre tessellatino, which is the usual approach in many texts. 


\chapter{Geometric preliminaries}
\unsure[inline]{Are graphs geometric? I mean, geometric graphs are geometric. But graphs in general? Are potentials part of this?}
Before diving into the mathematics of Gibbs-Laguerre-Delaunay tetrihedrization models, we must first lay out the fundamentals of their geometric and combinatorial structure. The key geometric component is the empty sphere property [...] which determines the edge structure, which is in turn analyzed in terms of hypergraphs.
\unsure[inline]{$\mathcal F$ or $\mathcal N$}
Let $\mathcal F_{lf}$ be the set of locally finite sets on $\Rt$, and $\mathcal F_{f} \subset \mathcal F_{lf}$ the set of all finite sets on $\Rt$. An elements of $F_{lf}$ will be usually denoted $\x$ and called a \textit{configuration} and its subset $\eta$. If $|\eta|=4$, as will be the case for the majority of this text, then $\eta$ will be called \textit{tetrahedron}.


\section{Tetrahedrizations}
The aim of this section is to introduce the geometric concepts necessary for the definition of the hypergraph structures in the following section. Definitions might be postponed. Note that although this text focuses solely on the three dimensional case, most ideas remain valid for a triangulation in any dimension. Furthermore, many facts have an analogous result in the case of Delaunay and Laguerre tessellations\todoo{Say this better and reference where to read about them}.
This text is concerned with two types of tetrihedrizations. 

We introduce the notion of (reinforced) general position. This requirement will be later relaxed.

\begin{definition}
Let $\x \in \mathcal F_{lf}$. We say $\x$ is in \textbf{general position} if 
$$ \eta \subset \x, 2 \leq |\eta| \leq 3 \Rightarrow \eta \text{ is affinely independent.} $$   
Denote $\mathcal F_{gp}\subset \mathcal F_{lf}$ the set of all locally finite configurations in general position.
\end{definition}
\todoo[inline]{Commment on measurability of the set of locally finite sets in general position. This comes from cite[Zessin2008] and the $\mathcal F$ $\mathcal M$ equivalence?}
\todoo[inline]{Also comment on the fact that we need a vector space with measurable inner product etc?}
\note[inline]{It's sufficient to check only subsets with $d+1$ points}
 
\begin{definition}
Let $\x \in \mathcal F_{gp}$. We say $\x$ is in \textbf{reinforced general position} if 
$$ \eta \subset \x, 3 \leq |\eta| \leq 4 \Rightarrow \eta \text{ is non-circular.} $$   
Denote $\mathcal F_{rgp}$ the set of all locally finite configurations in reinforced general position.
\end{definition}
\todoo[inline]{Define cocircular in general}
\todoo[inline]{Again, only need to check $d+2$}


\subsection{Delaunay tetrihedrization}
This section will shortly introduce the well known Delaunay tetrihedrization. There is vast literature on the topic, e.g. [ref]. 

\problem[inline]{Marks..}

\begin{definition}
	Let $\x \in \mathcal F_{gp}$, $\eta \subset \x$. An open ball $B(\eta,\x)$ such that $\eta \subset \partial B(\eta,\x)$ is called a \textit{circumball of $\eta$}. The boundary $\partial B(\eta,\x)$ is called a \textit{circumsphere}.
	Let $\eta\subset\x$, $|\eta|=4$, be a tetrahedron. Then we will denote its (uniquely defined) circumball as $B(\eta)$ as its definition does not depend on $\x$. 
\end{definition}	

Note that the circumball is uniquely defined by $\eta$. 

\begin{definition}
	Let $\x \in \mathcal F_{lf}$ and $\eta \subset \x$. We say that $(\eta,\x)$ satisfies the \textit{empty sphere property} if $B(\eta) \cap \x = \emptyset$. 
\end{definition}

\begin{definition}
	Let $\x \in \mathcal F_{lf}$. Define the set 
	$$\mathcal {D}(\x) := \{\eta\subset \x: \eta \text{ satisfies the empty sphere property }\}.$$
	and its subsets
	$$\mathcal {D}_k(\x) := \{\eta \in \mathcal {D}(\x): |\eta|=k \},\quad k = 1,\dots,4.$$
	We then define the \textit{Delaunay tetrihedrization of $\x$} as the set $\mathcal D_4$. 
\end{definition}

The set $\mathcal D_4$ contains the structure we would expect from the name tetrihedrization, namely it contains sets of 4-tuples of points whose convex hull are the tetrahedra forming the Delauany tetrihedrization. It will however be useful to also consider subsets with a different number of points.

\todoo[inline]{Talk about how we defined it, cause this ain't normal, man}

\todoo[inline]{Existence and uniqueness}



\subsection{Laguerre tetrihedrization}
A point $p=(p',p'')\in \Rt\times S$ can be seen as an open ball $B(p',\sqrt{p''})$. We will call $B_p = B(p',\sqrt{p''})$ the \textit{ball defined by $p$}. We define the sphere $S_p=\partial B_p$. 

\todoo[inline]{Probably link to credenbach or something for the properties of this}

\begin{definition}
	Define the \textit{power distance} of the unmarked point $q'\in \Rt$ from the point $p=(p',p'')\in\Rt\times S$ as
$$d(q',p) = \|q'-p'\|^2 - p''$$
\end{definition}
Much intuition can be gained from properly understanding the geometric interpretation of the power distance.

\begin{remark}[Geometric interpretation of the power distance]
We split the interpretation into two cases and use the Pythagorean theorem.
\begin{itemize}
	\item $d(q',p) \geq 0$. The point $q'$ lies outside of $B_p$. The quantity $\sqrt{d(p,q')}$ can be understood as the length of the line segment from $q'$ to the point of tangency with $B_p$ [fig]. The power distance is equal to zero precisely when $q'$ lies on the boundary $B_p$.
	\item $d(q',p) < 0$. The point $q'$ lies inside of $B_p$. The quantity $\sqrt{d(p,q')}$ now describes the length of \todoo{Describe using a fig}. 
\end{itemize}
\end{remark}
\todoo[inline]{Figures}

\begin{definition}
	For two (marked) points $p=(p',p'')$ and $q=(q',q'')$, define their \textit{power product}\footnote{ The motivation for calling the quantity $\rho(p,q)$ a product is most fascinating. It was first introduced by G. Darboux in 1866 as a generalization of the power distance. However it was later discovered that the spheres can be represented as vectors in a pseudo-Euclidean space where the power product plays the role of the quadratic form that defines the space. The resulting space is then the Minkowski space --- the setting in which the special theory of relativity is formulated. The positions of the sphere centres are then the positions in space, whereas the radius denotes a position in time. More can be found in e.g. \cite{Kocik2007}.} by 
$$\rho(p,q) = \|p'-q'\|^2 - p'' - q''.$$
Notice that $\rho(p,q) = d(p,q') - q'' = d(q,p') - p''$ and that $\rho(p,(q',0)) = d(p,q')$.
\end{definition}

Similarly to the power distance, the power product has a geometric interpretation that is vital to the understanding of the geometry of Laguerre tessellations.
% https://arxiv.org/ftp/arxiv/papers/0706/0706.0372.pdf

Let $p,q \in \Rt\times S$ be two points. The following observations follow immediately from the definiton. 
\begin{itemize}
	\item $B_p\cap B_q = \emptyset$. We obtain $\|p'-q'\|^2 \geq (\sqrt{p''} + \sqrt{q''})^2 = p'' + q'' + 2\sqrt{p''}\sqrt{q''}$ and thus $\rho(p,q) \geq 2\sqrt{p'' q''}.$ 
	\item $B_p \subset B_q$. We obtain $\|p'-q'\| + \sqrt{p''} \leq \sqrt{q''} $. Squaring the inequality yields $\rho(p,q) \leq -2\sqrt{p'' q''}.$ 
	\item $B_p \cap B_q \neq \emptyset$ and neither is a proper subset of the other. This case is the most important for us. In this case, the spheres $S_p$ and $S_q$ intersect at two points. Denote $a'$ the point of their intersection (it does not matter which one) and $\theta$ the angle $\angle p'a'q'$. We then obtain from the law of cosines. 
	$$- 2\sqrt{p'' q''}\cos \theta = \|p'-q'\|^2 - p'' - q'' = \rho(p,q)$$
\end{itemize}
\todoo[inline]{Some diagram to visualise the proposition?}

The above observations allow us to interpret the power product as a kind of distance of two marked points. The case $\rho(p,q)=0$ is crucial for the Laguerre geometry. If $p$ and $q$ satisfy this equality then they are said to be \textit{orthogonal}. 

We are now well-equiped to define the central terms necessary for the definition of the Laguerre tetrihedrization.

\begin{definition}
	Let $\eta\in\mathcal F_{gp}$. Define the \textit{characteristic point} of $\eta$ as the point $p_{\eta} = (p'_\eta, p''_\eta)$ which is orthogonal to every $p\in \eta$. If such point exists, we call $\eta$ \textit{Laguerre-coocircular}. 
\end{definition}
\todoo[inline]{Possibly add the characterization through power distance}
The characteristic point can thus be interpreted as a sphere that intersects each sphere $S_p, p\in\eta$ at a right angle. Note also that for each $p\in\eta$, we have
$$d(p'_\eta,p)=p''_\eta.$$

The following proposition looks at the existence and uniqueness of the characteristic point. Its proof is crucial.

\todoo[inline]{Existence and uniqueness}

\begin{proposition}[Existence and uniqueness of the characteristic point]
	Let $\eta\in\mathcal F_{gp}$. Then the following holds for the characteristic point $p_\eta$.
	\begin{enumerate}
		\item  If $|\eta|<4$, then the $p_\eta$ exists and is not unique.
		\item If $|\eta|=4$, then the $p_\eta$ exists and is unique.
		\item If $|\eta|>4$, then the $p_\eta$ exists if and only if $\eta$ is \todoo{define the term} Laguerre-cocircular.
	\end{enumerate}
\end{proposition}
\begin{proof}
	\todoo[inline]{Possibly rewrite this, or add a lemma that shows general position => full row rank (for $\leq 4$ rows)}
	We will look at the case $|\eta|=4$, from which the rest will \unsure{Not really follow, more like be directly observable}follow. Let $\eta = \{p_1, \dots, p_4\}$ and denote the coordinates of $p'_i$ as $x_i, y_i, z_i, i=1,\dots 4$. The characteristic point $p_\eta$ must satisfy the set of equations
	$$\|p'_\eta - p'_i\|^2 - p''_\eta - p''_i =0 \quad i=1,\dots,4$$
	If we denote $\alpha = x_\eta^2+y_\eta^2+z_\eta^2-p''_\eta$, where $(x_\eta,y_\eta,z_\eta)$ are the coordinates of $p'_\eta$, we obtain the equations
	$$\alpha - 2x_i x_\eta - 2y_i y_\eta - 2z_i z_\eta   = w_i - x^2_i - y^2_i - z^2, $$
	a system of equations which is linear with respect to $(\alpha,x_\eta,y_\eta,z_\eta)$. In an augumented matrix form, the system is written as
	\begin{equation}\label{augmat}
		\begin{amatrix}{4}
		1 & -2x_1 & -2y_1 & -2z_1 & p''_1 - x_1^2 - y_1^2 \\
		1 & -2x_2 & -2y_2 & -2z_2 & p''_2 - x_2^2 - y_2^2 \\
		1 & -2x_3 & -2y_3 & -2z_3 & p''_3 - x_3^2 - y_3^2 \\
		1 & -2x_4 & -2y_4 & -2z_4 & p''_4 - x_4^2 - y_4^2 \\
	\end{amatrix}
	\end{equation}
	The fact that $\eta\in\mathcal F_{gp}$ implies that $p'_1, \dots, p'_4$ are affinely independent, i.e. not coplanar. This means that the homogenous system of linear equations defined by the matrix
	$$
	\begin{pmatrix}\label{hommat}
		1 & x_1 & y_1 & z_1 \\
		1 & x_2 & y_2 & z_2 \\
		1 & x_3 & y_3 & z_3 \\
		1 & x_4 & y_4 & z_4 \\
	\end{pmatrix}
	$$
	does not have a solution, that is, the matrix has full rank. If it did, the points $p'_1,\dots,p'_4$ would all satisfy the equation $Ax+By+Cz+D=0$ for some $A,B,C,D \in \R$. The matrix \ref{hommat} has the same column space as the left hand side of \ref{augmat} and therefore the system has a unique solution.

	If $|\eta|<4$, we would obtain an underdetermined system, having either infinitely many or no solutions. \todoo{Write better later} Here, again, the general position property gives us full row rank of the left side of the augumented matrix, implying that there are infinitely many solutions. For $|\eta|=2$, general position implies that the points are unequal. For $|\eta| =3$, general position implies that the points are not collinear.


	If $|\eta|>4$, the system is overdetermined and has no solution, unless the whole augumented matrix has rank $4$. For e.g. $|\eta|=5$, this means that the homogenous system given by the matrix 
	$$
	\begin{pmatrix}\label{circmat}
		1 & x_1 & y_1 & z_1 & x_1^2 + y_1^2 + z_1^2 - p''_1  \\
		1 & x_2 & y_2 & z_2 & x_2^2 + y_2^2 + z_2^2 - p''_2  \\
		1 & x_3 & y_3 & z_3 & x_3^2 + y_3^2 + z_3^2 - p''_3  \\
		1 & x_4 & y_4 & z_4 & x_4^2 + y_4^2 + z_4^2 - p''_4  \\
		1 & x_5 & y_5 & z_5 & x_5^2 + y_5^2 + z_5^2 - p''_5  \\
	\end{pmatrix}
	$$
	However, this is equivalent to saying that there exists $p_\eta$ such that $\rho(p_\eta,p_i)=0$, i.e. that $\eta$ is Laguerre-cocircular.
\end{proof}
\todoo[inline]{Connect this to incircle?}

\begin{definition}
	Let $x\in \mathcal F_{gp}$ be a configuration, $\eta \subset \x$ and $p_\eta$ its characteristic point. We say that the pair $(\eta,\x)$ is \textit{regular}, or that $\eta$ is \textit{regular in} $\x$, if $\rho(p_\eta,p)\geq 0$ for all $p\in\x$.	
	For convenience, for $\x \in \mathcal F_{lf}\setminus \mathcal F_{gp}$, we define any $\eta\subset \x$ that does not satisfy the assumptions of general position as not regular.
\end{definition}
The definition can also be equivalently stated as 
$$\text{There is no point } q\in\x \text{ such that } d(p'_\eta,q)<p''_\eta$$

The regularity property ensures that no point of $\x$ is closer to the characteristic point $p_\eta$ in the power distance than the points of $\eta$. This is analogous to the empty sphere property in Delaunay tetrihedrization, where the circumball plays the role of the characteristic point. \todoo{c.f. remark that comes later}  

\begin{definition}
	Let $\x \in \mathcal F_{lf}$. Define the set 
	$$\mathcal {LD}(\x) := \{\eta\subset \x: \eta \text{ is regular}\}.$$
	and its subsets
	$$\mathcal {LD}_k(\x) := \{\eta \in \mathcal {LD}(\x): |\eta|=k \},\quad k = 1,\dots,4.$$
	We then define the \textit{Laguerre tetrihedrization of $\x$} as the set $\mathcal {LD}_4$. 
\end{definition}


\todoo[inline]{Talk about how cocircular points create multiplicities in the cliques - no they don't, since we're limiting $k$ to max $4$}

\begin{remark}[Invariance in weights]
	\todoo{Why? Also write a bit more}Notice that adding or subtracting weights to all points in $\x$ does not change regularity of any $\eta \subset \x$. This implies that the Laguere tetrihedrization is invariant under this operation.  
\end{remark}

\begin{remark}[Delaunay as a special case of Laguerre]
\tbd
\end{remark}

\subsubsection{Redundant points}

A major difference of the Laguerre tetrihedrization is the fact that some points may not play any role in the resulting structure.

\begin{definition}
	We call a point $p \in \x$ \textit{redundant in} $\x$ if $\mathcal {LD}(\x) = \mathcal {LD}(\x \setminus \{p\})$.
\end{definition}

To find more about redundant points, it is useful to introduce the notion of a Laguerre cell.

\begin{definition}
Let $p\in \x$. We then define the \textit{Laguerre cell of $p$ in $\x$}, denoted $C_p$, as the set
$$C_p := \{x' \in \Rt: d(x',p) \leq d(x',q) \; \forall q \in \x\}.$$ 
\end{definition}

\begin{proposition}
	A point $p$ is redundant if and only if $C_p=\emptyset$.
\end{proposition}
\begin{proof}
	($\Leftarrow$) Assume $p$ is not redundant. That means there exists a regular $\eta\subset \x$ with a characteristic point $p_\eta$ such that $\rho(q,p_\eta)=0$ for all $q\in\eta$ and $\rho(q,p_\eta)\geq 0$ for all $q\in \x$. This however means that $d(p'_\eta,p) = p''_\eta \leq d(p'_\eta,q)$ for all $q\in\x$, implying $p'_\eta \in C_p$. \newline
	($\Rightarrow$) Assume $C_p \neq \emptyset$. There exist $x' \in C_p$ and $q\in\x, q\neq p$, such that $d(x',q)=d(x',p)$, due to continuity of the power distance. But this implies that the point $p_\eta = (x', d(x',p))$ is the characteristic point of $\eta=\{p,q\}$ and that $\eta$ is regular.
\end{proof}

Apart from the empty Laguerre cell, there is, to our knowledge, no simple geometric characterization of a redundant point. There is however a necessary condition.

\begin{proposition}
	If $p$ is redundant in $\x$, then the sphere $B_p$ is completely contained in the balls of other points in $\x$, that is 
	$$B_p \subset \bigcup_{q \in \x\setminus \{p\}} B_q.$$
\end{proposition}
\begin{proof}
	Assume there exists $x' \in B_p$ such that $x' \notin B_q$ for any $q\neq p$. Then $x' \in C_p$, since $d(x', p) \leq 0$, while $d(x',q) \geq 0$ for all $q\in \x, q\neq p$.
\end{proof}


To interpret this fact intuitively see fig. [fig]. \todoo{Perhaps talk a bit more about the interpretation, e.g. why it's not sufficient}





\unsure[inline]{Restrict on non-redundant points? Measurability?}


\section{Hypergraph structures}
Both Delaunay and Laguerre tetrihedrizations can be seen as graphs where two points $p,q\in\x$ are joined if they are part of the same tetrahedron\todoo{satisfying ESP or sth}. For the purposes of this text, a more natural structure will be the hypergraph.

\subsection{Tetrihedrizations as hypergraphs}
 
\begin{definition}
	A \textit{hypergraph structure} is a measurable subset $\mathcal E$ of $(F_f\times N, \mathcal F_f \otimes \mathcal F)$ such that $\eta \subset \x$ for all $(\eta,\x)\in\mathcal E$. We call $\eta$ a \textit{hyperedge} of $\x$ and write $\eta \in \mathcal E(\x)$, where $\mathcal E(\x) = \{\eta: (\eta,\x) \in \mathcal E\}$. For a given $\x \in \mathcal F_{lf}$, the pair $(\x, \mathcal E(\x))$ is called a \textit{hypergraph}.
\end{definition}
A hypergraph is thus a generalization of a graph in the sense that edges are now allowed to "join" any number of points. A hypergraph structure can be thought of as a rule that turns a configuration $\x$ into the hypergraph $(\x,\mathcal E(\x))$. 

The subset $\eta \subset \x$ now plays the role of a hyperedge. e.g. tetrahedron.

The beauty in this approach is that we do not need to impose any additional structure on $\mathcal D(\x)$ or $\mathcal {LD}(\x)$ --- they already directly define a hypergraph structure! 

\begin{definition}[Delaunay and Laguerre-Delaunay hypergraph structures]
	\begin{itemize}
		\item $\mathcal D = \{(\eta,\x): \eta \in \mathcal D(\x)\}$
		\item 	$\mathcal D_k = \{(\eta,\x): \eta \in \mathcal D_k(\x)\}, k=1,\dots,4$
		\item 	$\mathcal {LD} = \{(\eta,\x): \eta \in \mathcal {LD}(\x)\}$
		\item 	$\mathcal {LD}_k = \{(\eta,\x): \eta \in \mathcal {LD}(\x)\}, k=1,\dots,4$
	\end{itemize}
\end{definition}

\todoo[inline]{$\mathcal {LD}$ only makes sense now, when it's Laguerre-Delaunay. Comment on it before or sth.}

\subsubsection{Hyperedge potentials}
The set $\mathcal E$ defines the structure of the hypergraph. What we are ultimately interest in is assigning a numeric value to each hyperedge and thus to (a region of) the hypergraph. To this end, we define the \textit{hyperedge potential}.
kkk
\begin{definition}
A \textit{hyperedge potential} is a measurable function $\varphi:\mathcal E\to \mathbb R \cup \{+\infty\}$. 

Hyperedge potential is \todoo{Define $\vartheta_x$} \textit{shift-invariant} if 
	$$(\vartheta_x \eta, \vartheta_x \x) \in \mathcal E \text{ and } \varphi(\vartheta_x \eta, \vartheta_x \x) = \varphi(\eta,\x) \text { for all } (\eta,\x)\in\mathcal E \text{ and }x \in \R,$$
where $\vartheta_x(\x) = \{(x',x'')\in\Rt\times S: (x'+x,x'')\in\x\}$ is the translation of the positional part of the configurations by the vector $-x \in \Rt$.	

For notational convenience, we set $\vartheta = 0$ on $\mathcal E^c$.
\end{definition}

The fact that the hyperedge potential contains $\x$ as a second argument suggests that it is allowed to depend on points of $\x$ other than those in $\eta$.

\begin{example}[Hyperedge potentials]
	The hyperedge potential can take various forms. As we will see later, its specification radically alters the distribution of the resulting Gibbs measure thus alowing a great freedom in the types of hypergraphs we can obtain.\newline
	\textbf{Volume of tetrahedron}: $\eta\in\mathcal E(\x)$ on $\mathcal D_4$ or $\mathcal {LD}_4$
	$$\varphi(\eta,\x) = |\text{conv}(\eta)| .$$
	Where $\text{conv}(\eta)$ is the convex hull of $\eta$.	\newline
	\textbf{Hard-core exclusion}: $\eta\in\mathcal E(\x)$ on $\mathcal D_4$ or $\mathcal {LD}_4$, $\alpha >0$
	$$\varphi(\eta,\x) = \delta(\eta)\;\; \text{ if } \delta(\eta) \leq \alpha$$
	$$\varphi(\eta,\x) = \infty \;\; \text { if } \delta(\eta) < \alpha$$
	Where $\delta(\eta)= \text{diam}B(\eta)$ is the diameter of the circumscribed ball. Notice that this potential becomes infinite on tetrahedra with circumdiameter larger than $\alpha$. As we will see later, this allows us to restrict the resulting tetrahedronization only those tetrahedra $\eta$ for which $\varphi(\eta,\x) \leq \alpha$.\newline
	\textbf{Laguerre cell interaction}: For $\eta \in \mathcal E(x)$ on $\mathcal {LD}_2$ such that $\eta=\{p,q\}$ and $|C_p| < \infty, |C_q| < \infty$, $\theta \neq 0$.
$$\varphi(\eta,\x) = \theta \bigg (\frac {\max ( Vol(C_p), Vol(C_q))}{\min(Vol(C_p),Vol(C_q))} - 1\bigg)$$
where the potential now depends on the size of neighboring Laguerre cells. Notice that $\theta$ can be negative, yielding a negative potential. \newline
\textbf{Tetrahedral interaction}: In the present setting, we cannot specify interaction between tetrahedra in $\mathcal D_4$ or $\mathcal {LD}_4$ as easily as between Laguerre cells. This can be solved by for example defining a new hypergraph structure
$$\mathcal {LD}^2_4 = \{(\eta,\x): \exists \eta_1,\eta_2\in\mathcal {LD}_4(\x), |\eta_1 \cap \eta_2|=3, \eta = \eta_1 \cup \eta_2\}$$
Which contains the quintuples of points which form adjacent tetrahedra in $\mathcal {LD}_4(\x)$. 
	
\end{example}


For a given hypergraph structure $\mathcal E$, the \textit{energy} of a finite configuration $\x\in\mathcal F_{f}$ is defined as the function\footnote{The letter $H$ is often used for the energy in statistical mechanics, possibly stemming from the fact that it is also often called the Hamiltonian}
$$H(\x) = \sum_{\eta \in \mathcal E(\x)} \varphi(\eta,\x).$$
However, in our case, we will typically deal with $\x\in \mathcal F_{lf}$, for this such potentials would typically be equal to $\pm \infty$. We will therefore be interested in the energy for only a bounded window $\Delta \in \mathcal B_0$. Currently, we don't have the necessary terms to describe such energy function precisely, thus we will postpose its definition to the next section. 

The words \textit{potential} and \textit{energy} suggest a connection with statistical mechanics, which gave rise to many of the concepts used in this text. Gibbs measure and concepts related to them continue to be an area with a rich interplay between statistical mechanics and probability theory. \footnote{In fact, Gibbs measures beginning of statistical mechanics -, name after Josiah Willard Gibbs, who coined the term statistical mechanics}.




\subsection{Hypergraph potentials and locality}
A natural question to ask is ``How do the points of $\x$ influence each other?''. We've seen that there is a type of locality at play, for example in $\mathcal D_4$ the empty sphere property of a tetrahedron $\eta$ is dependent solely on presence of points of $\x$ inside $B(\eta)$. The question is further complicated by the presence of the hyperedge potential. This section will refine the question by definining different locality properties.

As we will see in chapter [ref], this locality is essential for the existence of our models and Gibbs measures in general.

\begin{definition}
	A set $\Delta \in \mathcal B_0$ is a \textit{finite horizon} for the pair $(\eta,\x) \in \mathcal E$ and the hyperedge potential $\varphi$ if for all $\tilde{\x} \in N, \tilde{\x} = \x$ on $\Delta\times S$ 
$$(\eta,\tilde{\x})\in\mathcal E \text{ and } \varphi(\eta,\tilde{\x}) = \varphi(\eta,\x). $$
The pair $(\mathcal E, \varphi)$ satisfies the \textit{finite-horizon property} if each $(\eta,\x)\in \mathcal E$ has a finite horizon.
\end{definition}

The finite horizon of $(\eta,\x)$ delineates the region outside which points can no longer violate the regularity (or the empty sphere property) of $\eta$. 

\begin{remark} [Finite horizons for $\mathcal D$ and $\mathcal {LD}$]
For $\mathcal D$, the closed circumball $\bar B(\eta,\x)$ itself is a finite horizon for $(\eta,\x)$.

For $\mathcal {LD}$, the situation is slightly more difficult. For one, $B(p'_\eta, \sqrt{p''_\eta})$ does not contain the points of $\eta$. To see this, take two points $p,q$ with $p'',q''>0$ such that $\rho(p,q)=0$. Then $q'' = d(q',p) < \|q'-p'\|^2$ and thus $\sqrt{q''} < \|q'-p'\|$. More importantly, however, any point $s$ outside of $B(p'_\eta, \sqrt{p''_\eta})$ with a sufficiently large weight can violate the inequality $\rho(p_\eta,s) = \|p'_\eta - x'\|^2 - p''_\eta - s'' \geq 0$. 

To obtain a finite horizon for $\mathcal {LD}$, we need to use the fact that the mark space is bounded, $S=[0,W]$. If $s'' \leq W$, then $\Delta = B(p'_\eta, \sqrt{p''_\eta + W})$ is sufficient as a horizon, since any point $s$ outside $\Delta$ satisfies
$$\rho(p_\eta, s) = \|p'_\eta - s'\|^2 - p''_\eta - s'' \geq (\sqrt{p''_\eta+W})^2-p''_\eta-W = 0.$$ 

From a practical perspective, the maximum weight $W$ limits the resulting tessellation in the sense that the difference of weights can never be greater than $W$. Marks greater than $W$ are not necessarily a problem, as we can always find an identical tessellation with marks bounded by $W$, as long as there no two points $p,q$ with $|p''-q''|>W$ (see remark on invariance).
\end{remark}

Let us now return again to the task of defining an energy function $H$ that depends on the configuration in some bounded window $\Lambda \in \mathcal B_0$. To that end, we must define the set of hyperedges for which the hyperedge potential depends on the configuration inside $\Lambda$. 

\begin{definition} 
$$\mathcal E_\Lambda(\x) := \{ \eta \in \mathcal E(\x): \varphi(\eta,\zeta \cup \x_{\Lambda^c}) \neq \varphi(\eta,\x) \text{ for some } \zeta \in N_\Lambda \}$$
\end{definition}
\note[inline]{Later in the text, these are exactly the sets of tetrahedra used for the calculation, connect those two}

Recall that we defined $\varphi=0$ on $\mathcal E^c$. This means that for $\eta \in \mathcal E(\x)$ such that $\varphi(\eta,\x)\neq 0$ we have
$$\eta \notin \mathcal E(\zeta \cup \x_{\Lambda^c})\text{ for some }\zeta \in \mathcal F_{\Lambda} \Rightarrow \eta \in \mathcal E_{\Lambda}(\x)$$ 

Notice that $\x_\Lambda$ does not play any role in the definition. The configuration $\x$ thus only plays the role of a boundary condition.

With this definition, we are now ready for the desired definition of the energy function.

\begin{definition}
The \textit{energy of $\zeta$ in $\Lambda$ with boundary condition $\x$} is given by the formula
$$E_{\Lambda,\x}(\zeta) = \sum_{\eta \in \mathcal E_\Lambda(\zeta \cup \x_{\Lambda^c})} \varphi(\eta, \zeta \cup \x_{\Lambda^C})$$
for $\zeta \in \mathcal F_{\Lambda}$, provided the sum is well-defined.
\end{definition}

\begin{remark}[$\mathcal E_\Lambda(\x)$ for $\mathcal D$ and $\mathcal {LD}$]
For $\mathcal D$, $\eta \in \mathcal D_\Lambda(\x)$ $\iff$ $B(\eta,\x) \cap \Lambda \neq \emptyset$. \newline
For \todoo{Explain why}$\mathcal {LD}$, $\eta \in \mathcal {LD}_\Lambda(\x) \iff d(p'_\eta,\Lambda) \leq \sqrt{p''_\eta + W}$, where \todoo{Confusing notation, $d$ is reserved for the power distance}$d(p'_\eta,\Lambda) = \inf\{\|p'_\eta - x\|: x \in \Lambda\}$ is the distance of $p'_\eta$ from $\Lambda$.    \newline
\end{remark}

The final basic term again characterizes a type of finite-range property, this time as a property of the configuration $\x$.

\begin{definition}
	Let $\Lambda \in \mathcal B_0$ be given. We say a configuration $\x\ \in N$ \textit{confines the range of $\varphi$ from $\Lambda$} if there exists a set $\partial \Lambda(\x) \in \mathcal B_0$ such that $\varphi(\eta,\zeta \cup \tilde{\x}_{\Lambda^c}) = \varphi(\eta,\zeta\cup\x_{\Lambda^c})$ whenever $\tilde{\x} = \x$ on $\partial \Lambda(\x)\times S$, $\zeta \in N_\Lambda$ and $\eta \in \mathcal E_\Lambda(\zeta\cup\x_{\Lambda^c})$. In this case we write $\x \in N^\Lambda_\text{cr}$. We denote $r_{\Lambda,\x}$ the smallest possible $r$ such that $(\Lambda + B(0,r))\setminus \Lambda$ satisfies the definition of $\partial \Lambda(\x)$. We will use the abbreviation $\partial_\Lambda \x = \x_{\partial \Lambda(\x)}$.
\end{definition}

While the set $\mathcal E_\Lambda(\x)$ contains hyperedges $\eta$ which can be influenced by points in $\Lambda$, the set $\partial_\Lambda \x$  contains those points of $\x$ that influence the value of those $\eta$. This allows us to express $H_{\Lambda,\x}$ truly locally.

\begin{proposition}Let $\x\in N^\Lambda_\text{cr}$. Then 
	$$H_{\Lambda,\x}(\zeta) = \sum_{\eta \in \mathcal E_\Lambda(\zeta \cup \partial_\Lambda \x)} \varphi(\eta, \zeta \cup \partial_\Lambda \x).$$
\end{proposition}
\begin{proof} The definition of $N^\Lambda_\text{cr}$ implies the hyperedge potential does not depend on the points $\x \setminus \partial_\Lambda \x$ and $\mathcal E_\Lambda(\x)$ inherits this property by its definition.
\end{proof}

\todoo[inline]{Comment on the definition and what it means for $\mathcal D$ and $\mathcal {LD}$.}



\todoo[inline]{Measurability}

\chapter{Stochastic geometry}

\section{Gibbs point process}

\section{Random tessellations}

\chapter{Existence of Gibbs-type models}\label{ch:3}
In this chapter, the theorem from \cite{DDG12} will be presented and then we will proceed to verify its assumptions for our models.

\section{Existence theorem}
In this section we first state the two existence theorems from \cite{DDG12} and then proceed to introduce its assumptions.

\begin{theorem}
	For every hypergraph structure $\mathcal E$, hyperedge potential $\varphi$ and activity $z>0$ satisfying \textbf{(S)}, \textbf{(R)} and \textbf{(U)} there exists at least one Gibbs measure.
\end{theorem}

\begin{theorem}
	For every hypergraph structure $\mathcal E$, hyperedge potential $\varphi$ and activity $z>0$ satisfying \textbf{(S)}, \textbf{(R)} and \textbf{(\^{U})} there exists at least one Gibbs measure.
\end{theorem}

Proofs of both theorems can be found in \cite{DDG12}, see also remark 3.7. in the same paper about the marked case.

\subsection{Stability}
A standard assumption without which it is impossible to define the Gibbs measure is the stability assumption. 

\begin{enumerate}[\textbf{(S)}] 
	\item \textit{Stability}. The energy function $H$ is called \textit{stable} if there exists a constant $c_S \geq 0$ such that 
		$$H_{\Lambda,\x}(\zeta) \geq -c_S \cdot \mathrm{card}(\zeta \cup \partial_\Lambda \x)$$
for all $\Lambda \in \mathcal B_0, \zeta \in \mathbf N_\Lambda, \x \in N^\Lambda_{\text{cr}}$.
\end{enumerate}


The first thing to note that when $\varphi$ is non-negative, then we can simply choose $c_S = 0$. The interesting cases therefore is when $\varphi$ can attain negative values.\newline

\subsubsection{Stability in $\mathbb R^2$}
\tbd
\subsubsection{Stability in $\mathbb R^3$}
\tbd
\note[inline]{Possibly move the discussion to an appendix}
\note[inline]{Could we at least use spread for gibbs with limited distance between points?}

\noindent \textbf{Assumption 1}: All hyperedge potentials in the remained of this text are assumed to be non-negative.

\subsection{Range condition} \label{sec:range}
As stated previously, the fact that the hyergraph structures posses a type of locality property is crucial for the existence of Gibbs measures. The simplest such assumption is the \textit{finite range} assumption, see definition 7 in \cite{Dereudre2017}, which roughly states that there exists $R>0$ such that the energy of $\x$ in $\Lambda$ only depends on points in $\Lambda + b(0,R)$. This is a strong assumption and one that is not fulfilled by our models. 

This is reflected in part in the range condition introduced here and later in the uniform confinement condition \ref{eq:U1}.

\begin{enumerate}[\textbf{(R)}]\label{(R)}
	\item \textit{Range condition}. There exist constants $\ell_R,n_R \in \mathbb N$ and $\delta_R < \infty$ such that for all $(\eta,\x) \in \mathcal E$ there exists a finite horizon $\Delta$ satisfying: For every $x,y \in \Delta$ there exist $\ell$ open balls $B_1, \dots, B_\ell$ (with $\ell \leq \ell_R$) such that
	\begin{enumerate}[-]
		\item the set $\cup^\ell_{i=1} \bar B_i$ is connected and contains $x$ and $y$, and 
		\item for each $i$, either $\text{diam} B_i \leq \delta_R$ or $N_{B_i}(\x) \leq n_R$.
	\end{enumerate}
\end{enumerate}


Apart from being one of the assumptions necessary for the existence, the range condition also gives us the following crucial result we used in the definition of GPP.

\begin{proposition}\label{prop:cr-a.s.}
	Let $\Lambda \in \mathcal B_0(\Rt)$. Under the assumption \ref{(R)}, there exists a set $\hat {\mathbf N}^\Lambda_{cr} \in \mathbf N_{\Lambda^c}$ such that $\hat {\mathbf N}^\Lambda_{cr} \subset \mathbf N^\Lambda_{cr}$ and $P(\hat {\mathbf N}^\Lambda_{cr})=1$ for all $P \in \mathcal P_\Theta$ with $P(\varnothing)=0$.
\end{proposition}
\begin{proof}
	Can be found in proposition 5.4. in \cite{DDG12}. See also remark 3.7. in connection to the marked case.
\end{proof}
\problem[inline]{This is wrong, since we're using the wrong set $\mathbf N_\Lambda$}

The proposition shows that any $\Theta$-invariant probability measure on $(\mathbf N_{lf},\mathcal N_{lf})$ is concentrated on the set $\mathbf N^\Lambda_{cr}$ for any $\Lambda \in \mathcal B_0(\Rt)$.

\subsection{Upper regularity}


In order to present the upper regularity conditions, we introduce the notion of \textit{pseudo-periodic} configurations. 

Let $M\in\mathbb R^{3\times 3}$ be an invertible $3\times 3$ matrix with column vectors $(M_1,M_2,M_3)$. For each $k \in \mathbb Z^3$ define the cell
$$C(k) =  \{Mx \in \Rt: x-k \in \left[ -1/2, 1/2 \right)^3 \}.$$
These cells partition $\R$ into parallelepipeds. We write $C=C(0)$. Let \problem{Again, need to define these sets} $\Gamma \in \mathcal N'_C$ be non-empty. Then we define the \textit{pseudo-periodic} configurations $\bar \Gamma$ as
$$\bar \Gamma = \{ \x \in \mathbf N_{lf}: \vartheta_{Mk}(\x_{C(k)}) \in \Gamma \text{ for all } k \in \mathbb Z^3 \},$$
the set of all configurations whose restriction to $C(k)$, when shifted back to $C$, belongs to $\Gamma$. The prefix pseudo- refers to the fact that the configuration itself does not need to be identical in all $C(k)$, it merely needs to belong to the same class of configurations.

\begin{enumerate}[\textbf{(U)}] 
	\item \textit{Upper regularity}. $M$ and $\Gamma$ can be chosen so that the following holds. 
		\begin{enumerate}[(U1)]
			\item \textit{Uniform confinement}: $\bar \Gamma \subset N^\Lambda_\text{cr}$ for all $\Lambda \in \mathcal B_0$ and 
			\begin{equation}\label{eq:U1}r_\Gamma := \sup_{\Lambda\in\mathcal B_0}\sup_{\x \in \bar\Gamma} r_{\Lambda, \x} < \infty\end{equation}
			\item \textit{Uniform summability}: 
			$$c^+_\Gamma := \sup_{\x \in \bar\Gamma}  \sum_{\eta \in \mathcal E(\x): \eta \cap C \neq \emptyset} \frac{\varphi^+(\eta,\x)}{\#(\hat\eta)} < \infty,$$
where $\hat\eta := \{k \in \mathbb Z^3: \eta \cap C(k) \neq \emptyset\}$ and $\varphi^+ = \max(\varphi,0)$ is the positive part of $\varphi$.
\item \textit{Strong non-rigidity}: $e^{z|C|} \Pi^z_C(\Gamma) > e^{c_\Gamma}$, where $c_\Gamma$ is defined as in (U2) with $\varphi$ in place of $\varphi^+$.
		\end{enumerate}
\end{enumerate}

Notice that (U1) is very close to the classic finite range property mentioned at the beginning of section \ref{sec:range}. The major difference is that here the property is only required of the pseudo-periodic configuration.


As long as $\Pi^z_C (\Gamma) >0$, (U3) will always hold for all $z$ exceeding some threshold $z_0 \geq 0$. This is because the left hand side is an increasing function of $z$, as can be seen from the equality 
$$e^{z|C|} \Pi^z_C(\Gamma) = \sum^\infty_{k=1} \frac{z^k}{k!} \int_C \cdots \int_C 1_{\Gamma} \left(\sum^k_{i=1} \delta_{X_i}\right) dx_1, \dots, dx_k,$$
which can be derived using \ref{eq:poiscalc}. 


\todoo[inline]{Add more intuition about U3 and comment on why \^U is useful}

For some models it is possible to replace the upper regularity assumptions by their alternative and prove the existence for all $z>0$.

\begin{enumerate}[(\textbf{\^{U}})]
	\item \textit{Alternative upper regularity}. $M$ and $\Gamma$ can be chosen so that the following holds.
	\begin{enumerate}[(\^U1)]
		\item \textit{Lower density bound}: There exist constants $c,d > 0$ such that $\mathrm{card}(\zeta) \geq c|\Lambda| - d$ whenever $\zeta \in \mathbf N_f\cap\mathbf  N_\Lambda$ is such that $H_{\Lambda,\x}(\zeta)<\infty$ for some $\Lambda \in \mathcal B_0$ and some $\x \in \bar\Gamma$.
		\item = (U2) \textit{Uniform summability}.
		\item \textit{Weak non-rigidity}: $\Pi^z_C(\Gamma) > 0$.
	\end{enumerate}
\end{enumerate}






\section{Verifying the assumptions}

\subsection{The choice of $\Gamma$ and $M$ for Laguerre-Delaunay models}\label{sec:MGamma}
Fix some $A \subset C\times S$ and define
$$\Gamma^A = \{\zeta \in \mathbf N_C: \zeta = \{p\}, p \in A\},$$
the set of configurations consisting of exactly one point in the set $A$. The set of pseudo-periodic configurations $\bar\Gamma$ thus contains only one point in each $C(k), k~\in~\mathbb Z^3$.

Let $M$ be such that $|M_i| = a > 0$ for $i=1,2,3$ and $\angle(M_i,M_j) = \pi / 3$ for $i\neq j$.

\subsubsection{Choice of the set $A$}
In \cite{DDG12}, $A$ is chosen to be $B(0,b)$ for $b\leq \rho a$ for some sufficiently small $\rho >0$. 

We will use this form for the positions of the points as well --- the question, however, is how to choose the mark set. For Delaunay models, we choose $A=B(0,b)\times\{0\}$. It would be convenient to do this in the Laguerre case and only deal with the Delaunay tetrahedronization. However, for Laguerre-Delaunay models, this  would mean that $\Pi^z_C(\Gamma) = 0$, conflicting with both $(U3)$ and $(\hat U3)$. The choice $A=B(0,b)\times S$ could, for a small enough $a$, result in some balls being fully contained in their neighboring balls, possibly resulting in redundant points, thus changing the desired properties of $\Gamma$. It is thus necessary to choose the mark space dependent on $a$. For given $a$, $\rho$, the minimum distance between individual points, is $a-2\rho a = a(1-2\rho)$. For $\mathcal {LD}$ models we therefore choose 
$$A = B(0,b)\times \left[0, \sqrt{\frac a2(1-2\rho)}\right]$$ 
in order for balls to never overlap \unsure{This is perhaps unnecessarily conservative, we could widen it}. 




\begin{remark}[Simplification of (U2) and (U3)]\label{r:UA}
	Using the set $\Gamma^A$, we can simplify the assumptions (U2) and (U3).
\begin{enumerate}[(U2)]	
	\item	We now have \todoo{Check how I am using $|\cdot |$ and $\#$} $\#(\hat\eta) = \mathrm{card}(\eta)$, since now each point of $\eta$ is necessarily in a different set $C(k)$.

\item $\Pi^z_C(\Gamma)$ can now be directly calculated.
	\begin{align*} 
		\Pi^z_C(\Gamma) &= \Pi^z_C(\{\zeta \in N_C: \zeta = \{p\}, p \in A\}) \\
		& = e^{-z|A|} z |A| e^{-z|C\setminus A|} \\
		& = e^{-z|C|} z |A|,
	\end{align*}
	and thus (U3) becomes
	$$z|A| > e^{c_{A}},$$
	where $c_A := c_{\Gamma^A}$.

	In  the case $A = B(0,\rho a)\times [0, \sqrt{\frac a2(1-2\rho)}]$ for $\mathcal {LD}$, we have
	$$|A| = \frac 43 \pi (\rho a)^3 \cdot \sqrt{\frac a2(1-2\rho)} = \frac {4\pi}{3\sqrt{2}}\cdot  \rho^3 \sqrt{1-2\rho} \cdot a^{7/2}$$

\end{enumerate}
\end{remark}


\subsection{Geometrical structure of the tetrihedrizations defined by $\Gamma^A$ and $M$}
\problem[inline]{Am I talking about tetrihedrization or hypergraph? Check and unify this}
To understand the advantage of the particular choice of $M$ and $\Gamma^A$ we first turn to the two-dimensional case. For $\mathbb R^2$, the two column vectors with angle $\pi/3$ define a triangulation made of equilateral triangles. Depending on the bound for $\rho$, the points never become collinear ($\sqrt 3/4$), have a bound for the circumradius that is linear in $\rho$ ($\sqrt 3/6$) or even always generate the same triangultaion ($(\sqrt 3 - 1)/4$) up to the movement of points within their respective set $A$. Thus the resulting triangulation has many desirable properties. \newline

It is not however obvious that the desirable properties carry over to $\mathbb R^3$. Before we investigate the structure of the resulting tetrihedrizations, we list the properties we are interested in obtaining.
\begin{enumerate}
	\item A description of the tetrahedra present in the tetrahedrization.
	\item The number of tetrahedra incident to the point in $C$,  
		$$n_T := \mathrm{card} \{\eta \in \mathcal E(\x): \eta' \cap C \neq \emptyset\}.$$
	\item Bounds for circumdiameters of the tetrahedra.
	\item The position of points with respect to the (reinforced) general position.  
	\item Boundedness of the weight of the characteristic points.
\end{enumerate}

\problem[inline]{There's now a double use of the word regular.}
As noted previously, using an analogous definition of $M$ in $\mathbb R^2$ forms a triangulation containing equilateral triangles. Sadly, the three-dimensional case is not as simple\footnote{And it could not be, because the analogue of the two-dimensional equilateral triangle, the regular tetrahedron, does not tessellate, as Aristotle famously wrongly claimed, see e.g. \cite{Lagarias12}}.

\subsubsection{The structure of the tetrahedrization formed by $\bar\Gamma^A$}
To better understand the structure of the resulting tetrahedrizations, we choose a particular example of a configuration from $\bar\Gamma^A$. 
$$\x_0 = \{(M_ak,0) \in \Rt\times S: k \in \mathbb Z^3\} \in \bar\Gamma,$$ 
the set of zero-weight points lying in the center of their respective cells $C(k)$, where

$$
M_a := a \begin{pmatrix}
1 & \frac 12 & \frac 12 \\
0 & \frac {\sqrt{3}}2 & \frac 1{2\sqrt{3}}  \\
0 & 0 & \sqrt{\frac 23} \\
\end{pmatrix}
$$

is a particular example of the matrix $M$. \newline

\noindent From remark \ref{rem:LaguerreToDelaunay} we know that $\mathcal {LD}_4(\x_0) = \mathcal D_4(\x_0)$, therefore we can work with its Delaunay tetrihedrization.
\problem[inline]{Again we're not using marks without comment}

To further simplify the line of reasoning, we will look at only a subset $\x_1\subset \x_0$ of the points whose preimage under $M_a$ are the boundary points of the unit cube $[0,1]^3$. The points of $\x_1$, denoted $p_1, \dots, p_8$ then are:

\problem[inline]{It's unclear what $p_i$ are}

$$\begin{matrix}
	p_1: & (0,0,0) & \longmapsto & a(0,0,0) \\
	p_2: & (1,0,0) & \longmapsto & a(1,0,0)\\
	p_3: & (0,1,0) & \longmapsto & a(1/2,\sqrt{3}/2,0)\\
	p_4: & (1,1,0) & \longmapsto & a(3/2,\sqrt{3}/2,0)\\

	p_5: & (0,0,1) & \longmapsto & a(1/2,1/(2\sqrt{3}),\sqrt{2/3})\\
	p_6: & (1,0,1) & \longmapsto & a(3/2,1/(2\sqrt{3}),\sqrt{2/3})\\
	p_7: & (0,1,1) & \longmapsto & a(1,2/\sqrt3, \sqrt{2/3})\\
	p_8: & (1,1,1) & \longmapsto & a(2,2/\sqrt3, \sqrt{2/3})\\
\end{matrix}$$
To obtain the tetrihedrization of the parallelopiped formed by $\x_1$, we could mechanically perform the INCIRCLE test on all quintuples of points in $\x_1$ (see remark \ref{r:construct}). Such approach is lengthy and ultimately not very illuminative. We will therefore derive its structure through a few geometric observations.

\todoo[inline]{These lemmas are almost impossible to follow without figures}
\begin{lemma}
	$\mathrm{NNG}(\x_1)$ is formed by two regular tetrahedra, $\{p_1,p_2,p_3,p_5\}$ and $\{p_4,p_6,p_7,p_8\}$, and an regular octahedron $\{p_2,\dots,p_7\}$.
\end{lemma}
\begin{proof}
	Any invertible linear transformation maps a parallelepiped onto a parallelepiped. Since $\|p_2 - p_1\| = \|p_3-p_1\| = \|p_5-p_1\|=a$ by definition of $M$, we obtain that all the edges of the parallelepiped $\{p_1,\dots,p_8\}$ have length $a$. Furthermore, each face of the parallelopiped can be split into two equilateral triangles, e.g. $\|p_3-p_2\|=a$. Consequently $\{p_1,p_2,p_3,p_5\}$ and $\{p_4,p_6,p_7,p_8\}$ are regular tetrahedra, the regularity coming from the fact that all edges have length $a$. Similarly, the sextuple $\{p_2,\dots,p_7\}$ is a regular octahedron, as all its edges have length $a$.
\end{proof}

This polyhedral configuration is well known to tessellate\footnote{ The tessellation is of great importance to many fields and thus is known under many names. In mathematics, it is most commonly called the \textit{tetrahedral-octahedral honeycomb}, or the \textit{alternated cubic honeycomb}. In structural engineering, it is known as the \textit{octet truss}, as named by Buckminster Fuller, or the \textit{isotropic vector matrix}. It is stored as \textit{fcu} in the Reticular Chemistry Structure Resource\cite{RCSR}. It is also the nearest-neighbor-graph of the face-centered cubic (fcc) crystal in crystallography\cite{Gabbrielli12}.  }. The knowledge of $\mathrm{NNG}(\x_1)$ allows us to fully categorize the tetrahedra in $\mathcal D_4(\x_0)$.

\begin{proposition}\label{prop:tetraInTess} $\mathcal D_4(\x_0)$ contains two types of tetrahedra, $T_1$ and $T_2$, with edge lengths
$$T_1: (a,a,a,a,a,a) \quad T_2:(a,a,a,a,a,\sqrt 2a)$$
\end{proposition}
\begin{proof}
From proposition \ref{prop:nng} we know that $\mathrm{NNG}(\x_1) \subset \mathcal D_2(\x_1)$. 

We know that $\text{NNG}(\x_1)$ is composed of two regular tetrahedra and one regular octahedron $O$ with all edge lengths equal to $a$. Therefore all that remains to be done is to tetrahedronize the octahedron. By the symmetry of the regular octahedron, all the tetrahedra inside $O$ must be the same up to rotation. Each tetrahedron has five out of six edge lengths equal to $a$, therefore we only need to determine the remaining edge length. We can take e.g. any four points forming a square with side lengths $a$ to see that the remaining edge length is $\sqrt 2a$.

Since $\mathcal D_4(\x_0)$ is tessellated by copies of $\mathcal D_4(\x_1)$ translated by vectors $k\in\mathbb Z^3$, we have fully characterized the tetrahedra of $\mathcal D_4(\x_0)$. 
\end{proof}

We note that the circumradii of the tetrahedra can be calculated using the Cayley-Menger determinant (see Appendix \ref{appendix}) and are $\sqrt{6}/4 \cdot a$ for $T_1$ and $1/\sqrt{2}\cdot a$ for $T_2$.

\subsubsection{Combinatorial structure of $\mathcal D_4(\x_0)$}\label{sec:combinatorial}
Now we turn to the combinatorial structure of $\mathcal D_4(\x_0)$. In the tetrahedronized regular octahedron, each vertex is incident to $\binom{5}{3}-2=8$ tetrahedra. In the tetrahedron-octahedron tessellation, each \todoo{Reference, possibly using Schlafli symbols} vertex is incident to eight regular tetrahedra and six regular octahedra. This gives us $n_T = 8 + 6\cdot 8 = 56$. While still large, this is less than quarter of \problem{Overcounting degenerate cases}$8\cdot \binom{7}{3} = 280$ for the case of regular cube tessellation induced by the choice $M=aE$. Note that $n_T$ is much smaller for the non-degenerate case, when $O$ contains only $4$ tetrahedra and its vertices are incident either to $2$ or $4$ tetrahedra. In this case, $n_T\leq 8+6\cdot 4 = 32$.


\subsubsection{Circumdiameter and characteristic point weight}
The bound on circumdiameters of the circumballs and characteristic point weights is crucial for the assumption (U1) as well as (U2) and (U3) for potentials that include them. Without such a bound, we have no uniform confinement and the hyperege potential can grow to infinity. We therefore have to investigate the shape of the tetrahedra that are possible with $\x \in \bar\Gamma$. 

\begin{proposition}\label{prop:maxPeta}
Let $\x \in \bar\Gamma^A$ with $\rho < 1/4$. Then there exists $C>0$ such that $p_\eta'' \leq C$ for all $\eta \in \mathcal {LD}_4(\x)$. 
\end{proposition}
\begin{proof}
Denote $\eta=\{p_1,p_2,p_3,p_4\}$, denote their positions $\eta'$ and weights $\eta''$. From proposition \ref{prop:charpointHyperplane} and the remark below it, we know that $p'_\eta = H(p_1,p_2)\cap H(p_1,p_3) \cap H(p_1,p_4)$.

Fix the positions $\eta'$.  Changing any of the points' weights ammounts to translation of the radical hyperplanes defined by that point (see note after proposition \ref{prop:charpointHyperplane}). Given the fact that weights are bounded, $S=[0,W]$, we find that for given positions $\eta'$ there exists $W_{\eta'}>0$ such that , we have $p''_\eta \leq W_{\eta'}$ regardless of the weights.

It remains to prove that $\sup_{\eta'} W_{\eta'} < \infty$, i.e. changing the points' positions can produce only bounded $p''_\eta$. This amounts to proving that the points of $\eta$ are not allowed to come arbitrarily close to (or even attain) a non-general position. This is equivalent with boundedness of the circumsphere of $\eta'$, which is proved for $\rho<1/4$ in the appendix \ref{appendix}.
\end{proof}



\subsection{Existence propositions}
\todoo[inline]{Specify these things as ``models'' of the form $(\mathcal D_4, \varphi_S)$.Specify the measure $\mu$ in there, too}
In this section, we will verify the assumptions for the existence of Gibbs measures with the energy function defined on the hypergraphs $\mathcal D_4$ and $\mathcal {LD}_4$. We use the general letter $\mathcal E$ when we mean either $\mathcal D$ or $\mathcal {LD}$. Two potentials will be considered

\textbf{Smooth interaction}:  For $\eta\in\mathcal E_4(\x)$ define the potential $\varphi_S$ as an unary potential such that
$$\varphi_S(\eta,\x) \leq K_0 + K_1 \delta(\eta)^{\alpha}$$
for some $K_0,K_1 \geq 0, \alpha >0$\newline
\textbf{Hard-core interaction}: For $\eta\in\mathcal E_4(\x)$ define the potential $\varphi_{HC}$ as an unary potential such that
$$\sup_{\eta: d_0 \leq \delta(\eta) \leq d_1} \varphi_{HC}(\eta,\x) = < \infty \text{ and } \varphi_{HC}(\eta,\x)=\infty \text{ if } \delta(\eta)>\alpha.$$ 
for some $0\leq d_0 < d_1 \leq \alpha$ \unsure{How exactly does this look? Why?}

We assume by Assumption 1 that $\varphi_S,\varphi_{HC}\geq 0$.

We first present a general lemma. Recall the definition of $r_\Gamma$ from (U1). 
\begin{lemma}\label{lemma:U1}
	Let $\Gamma \subset N_{lf}$ be a class of configurations. If there exists $d_{max}>0$ such that $\mathrm{diam}\Delta < d_{max}$ for the horizon $\Delta$ of any $(\eta,\x), \eta \in \mathcal E(\x), \x \in \Gamma$, then 
	$$r_\Gamma < d_{max}.$$
\end{lemma}
\begin{proof}
	Choose $\Lambda\in \mathcal B_0$ and $\x \in \Gamma$. Let $\zeta \in N_\Lambda$ and $\eta \in \mathcal E_\Lambda(\zeta \cup \x_{\Lambda^c})$ and denote $\Delta$ the finite horizon of $(\eta,\x)$. From lemma \ref{lemma:horizEset} we obtain $\Delta\cap\Lambda \neq \emptyset$. Then $\Delta \subset \Lambda + B(0,d_max)$. If we take $\tilde\x \in \Gamma$ such that $\tilde\x = \x$ on $\partial\Lambda(\x)$ then $\varphi(\eta,\zeta \cup \x_{\Lambda^c}) = \varphi(\eta,\zeta\cup \tilde \x_{\Lambda^c})$ since $\zeta\cup\x_{\Lambda^c}$ and $\zeta\cup\tilde\x_{\Lambda^c}$ differ only on $\Delta^c$.
\end{proof}


\begin{proposition}\label{prop:E1}
	There exists at least one Gibbs measure for the model $(\mathcal D_4,\varphi_S)$ and every activity 
	$$z> \frac{3}{4\pi}e^{14 K_0}   (2K_1 \alpha e^3/3)^{1/\alpha} \frac{(\delta_1(\rho)^\alpha + \delta_2(\rho)^\alpha)^{1/\alpha}}{\rho^3}.$$
\end{proposition}
\begin{proof}
\begin{enumerate}[]
	\item \textbf{(R)} The finite-horizon $\Lambda = \bar B(\eta,\x)$ with $\ell_R = 1, n_R = 0$ and $\delta_R$ arbitrary can be used. This is because it itself contains no points of $\x$ by definition of $\mathcal D$ and acts as the open ball from the definition of the range condition.
	\item \textbf{(S)} Stability is satisfied because of $\varphi$ is non-negative.
	\item \textbf{(U)} We choose $M$ and $\Gamma$ as in section \ref{sec:MGamma}.
		\begin{enumerate}[(U1)]
			\item We know from \ref{prop:Apollonius} that the exists $d_{max}>0$ such that $\text{diam}B(\eta)\leq d_{max}$. By lemma \ref{lemma:U1} $r_\Gamma\leq d_{max}.$
			\item is trivially satisfied since $n_T < 58$ and $\varphi_S$ is bounded by \ref{prop:Apollonius}.
			\item By remark \ref{r:UA} we want to find $z$ as small as possible such that $z|A|>e^{c_A}.$ We know from \ref{sec:combinatorial} that there are $8$ $T_1$ and $48$ $T_2$ tetrahedra intersecting $C$, therefore from [ref]
				$$c_A \leq \frac a4 (8\cdot \delta_1 + 48\cdot \delta_2 )$$
				This yield the bound
				\begin{align*}z &> \frac{4\pi\rho^3}{3} e^{2(K_0 + K_1 (a\delta_1)^\alpha) + 12(K_0 + K_1(a \delta_2)^\alpha)} / a^3  \\
					&= C_0 e^{C_1 a^\alpha} / a^3  
				\end{align*}
				where $C_0 = 3e^{14K_0}/(4\pi \rho^3),  C_1 = 2k_1(\delta_1^\alpha + 6 \delta_2^\alpha)$.

				We now choose $a$ to minize the expression above. By optimizing over $a$ we obtain $a=(3/(C_1 \alpha))^{1/\alpha}$ which yields the bound 
				$$z> C_0(C_1 \alpha e^3/ 3 )^{1/\alpha}.$$
		\end{enumerate}
\end{enumerate}
\end{proof}



\begin{proposition}\label{prop:E2}
	There exists at least one Gibbs measure for the model $(\mathcal D_4,\varphi_{HC})$ and every activity $z>0.$
\end{proposition}
\begin{proof}
\begin{enumerate}[]
	\item \textbf{(R)} Again, $\Lambda = \bar B(\eta,\x)$ with $\ell_R = 1, n_R = 0$. Because of the hard-core condition, we can also take $\delta_R = 2\alpha$.
	\item \textbf{(S)} Stability is satisfied because of $\varphi$ is non-negative.
	\item \textbf{(\^U)} We choose $M$ and $\Gamma$ as in section \ref{sec:MGamma}.
		\begin{enumerate}[(\^U1)]
			\item For all $\eta \in \mathcal D_4(\x)$ for $\x\in\Gamma^A$ such that $H_\Lambda(\x)<\infty$ we have $\delta(\eta) < \alpha$. This imposes a minimum density of points, since e.g. no ball with diameter $\alpha$ can be empty. 
			\item We have $n_T<56$ and thus the only quantity in question is $\varphi_{HC}$. By \ref{prop:Apollonius}, we have $\delta(\eta)\leq a\delta^{max}_2(\rho)$, thus we only need to choose $a$ such that $\delta^{max}_2(\rho) \leq \alpha / a$.
			\item $\Pi^z_\Lambda(\Gamma)>0$ by remark \ref{r:UA}.
		\end{enumerate}
\end{enumerate}
\end{proof}



\begin{proposition}\label{prop:E3}
	There exists at least one Gibbs measure for the model $(\mathcal {LD}_4,\varphi_S)$ and every activity 
	$$z> \frac{3\sqrt 2}{4\pi}e^{14 K_0}   (2K_1 \alpha e^{7/2}/(7/2))^{1/\alpha} \frac{(\delta_1(\rho)^\alpha + \delta_2(\rho)^\alpha)^{1/\alpha}}{\rho^3 \sqrt{1-2\rho}}.$$
\end{proposition}
\begin{proof}
\begin{enumerate}[]
	\item \textbf{(R)} Take the horizon set $\Delta = B(p'_\eta, \sqrt{p''_\eta + W})$. $\Delta$ can be decomposed into the sphere $p_\eta$ and $\Delta \setminus p_\eta$, a 3-dimensional annulus with width $\sqrt{p''_\eta+W} -\sqrt{p''_\eta}=W/(\sqrt{p''_\eta+W} + \sqrt{p''_\eta})$. By definition of $\mathcal {LD}$ and remark, $p_\eta$  cannot contain any points of $\x$. \todoo{Ugly line placements, improve} Although the annulus $\Delta \setminus p_\eta$ does not have any bound on the number of points, its width is bounded by $\sqrt W \geq  W/(\sqrt{p''_\eta+W} + \sqrt{p''_\eta})$. This means that any $x,y\in \Delta$ can be connected by the spheres $B(x,\sqrt W), p_\eta, B(y,\sqrt W)$, yielding the parameters $\ell_R = 3,n_R=0,\delta_R=2\sqrt W$.
	\item \textbf{(S)} Stability is satisfied because of $\varphi$ is non-negative.
	\item \textbf{(U)} We choose $M$ and $\Gamma$ as in section \ref{sec:MGamma}.
		\begin{enumerate}[(U1)]
			\item By proposition \ref{prop:maxPeta} there is $C>0$ such that $p''_\eta\leq C$ for all $\eta \in \mathcal {LD}_4(\x), \x \in \bar\Gamma^A$. By lemma \ref{lemma:U1} we have $r_\Gamma\leq \sqrt{C + W}$.
			\item is trivial since $n_T<56$ and $\varphi_{S}$ is bounded by \ref{sec:combinatorial}.
			\item We proceed similarly as in \ref{prop:E1} and obtain
				$$z>C_0 e^{C_1 a^\alpha} / a^{7/2}$$
				where $C_0=3 \sqrt 2 e^{14K_0} / (4\pi \rho^2 \sqrt{1-2\rho})$, $C_1 = 2K_1(\delta_1^\alpha + 6\delta_2^\alpha)$. Optimizing over $a$ we obtain $a=(7/(2C_1\alpha))^{1/\alpha}$ arriving at the bound
				$$z> C_0 (C_1 \alpha e^{7/2} / (7/2))^{1/\alpha}.$$
		\end{enumerate}
\end{enumerate}
\end{proof}



\begin{proposition}\label{prop:E4}
	There exists at least one Gibbs measure for the model $(\mathcal D_4,\varphi_S)$ and every activity $$z>0.$$
\end{proposition}
\begin{proof}
\begin{enumerate}[]
	\item \textbf{(R)} The horizon set is $\Delta = B(p'_\eta,\sqrt{p''_\eta +W})$. Parameters can be chosen as in proposition 3. \unsure{Is it a problem that there's no $n_R$ circle? Cause the proof suggested something like that?}
	\item \textbf{(S)} Stability is satisfied because of $\varphi$ is non-negative.
	\item \textbf{(U)} We choose $M$ and $\Gamma$ as in section \ref{sec:MGamma}.
		\begin{enumerate}[(U1)]
			\item Same as in proposition \label{prop:E2}. Although the underlying structure is different, the potential still depends on $\delta(\eta)$ and (\^U1) requires the configuration to have non-infinite energy.
			\item Same as in proposition \label{prop:E2}, $n_T<56$ and we choose an appropriate $a$.
			\item $\Pi^z_\Lambda(\Gamma)>0$ by remark \ref{r:UA}
		\end{enumerate}
\end{enumerate}
\end{proof}



\begin{remark}[Extending to other potentials]
	.
\note[inline]{Directly obtainable results: 1) Smooth interaction for other unary potentials such as $k$-facet volume (use Hadamard inequality to bound them). 2) Adding additional constraints to hardcore models as long as we can find $a$ to satisfy the constraints.}
\end{remark}


\chapter{Simulation and Estimation}
\section{Simulation}
\subsection{Monte Chain Markov Carlo}
\subsection{Practical implementation}

\section{Estimation}
\subsection{Maximum pseudolikelihood}
\subsection{Practical implementation}

\chapter{Estimation}
\section{Maximum pseudolikelihood}
Assume now that we obtain the point configuration $\gamma$ on the observation window $\Lambda_n = [-n,n]^3\times W$ and wish to estimate the model parameters.
\unsure[inline]{Boundary problems. Do they simply exist because we're assuming to *only* know the configuration on $\Lambda_n$?}
The estimation procedure closely follows that from~\cite{DL10}. That is a two-step approach, first estimating the hardcore parameters $\beta = (\epsilon,\alpha)$ and then using the estimates to obtain the estimate of $\theta$ through maximum pseudolikelihood (MPLE). 
\unsure[inline]{What exactly is the role of the growing window in~\cite{DL10}?}
\subsection{Estimation of the hardcore parameters}
Thanks to the \unsure{\cite{DL10} has this the other way around?}fact that the hardcore parameter $\epsilon$ satisfies
$$ \text{if } \epsilon > \epsilon' \text{ then  } \forall \Lambda, \; E^{\epsilon, \alpha,\theta}_\Lambda(\gamma_\Lambda,\gamma_{\Lambda^c}) < \infty \Rightarrow  E^{\epsilon',\alpha,\theta}_\Lambda(\gamma_\Lambda,\gamma_{\Lambda^c})<\infty,$$ 

and the hardcore parameter $\alpha$ satisfies

$$ \text{if } \alpha < \alpha' \text{ then  } \forall \Lambda, \; E^{\epsilon,\alpha,\theta}_\Lambda(\gamma_\Lambda,\gamma_{\Lambda^c}) < \infty \Rightarrow  E^{\epsilon,\alpha',\theta}_\Lambda(\gamma_\Lambda,\gamma_{\Lambda^c})<\infty,$$ 

their consistent estimators are:
$$\hat\epsilon = \inf\{\epsilon > 0, E_\Lambda(\gamma_\Lambda, \gamma_\Lambda^c) < \infty \},$$
$$\hat\alpha = \sup\{\alpha > 0, E_\Lambda(\gamma_\Lambda, \gamma_\Lambda^c) < \infty \}.$$

\unsure{Are these consistent? Why?}In practice, the parameters are estimated as
$$\hat\epsilon = \min\{a(T), T\in Del_\Lambda(\gamma)\},$$
$$\hat\alpha = \max\{r(T), T\in Del_\Lambda(\gamma)\}.$$

The estimate $\hat\beta = (\hat\epsilon,\hat\alpha)$ is then used in the pseudo-likelihood function in the second estimation step.


\subsection{Estimation of the smooth interaction parameters}
\todoo[inline]{Equation references}


The classical version of MPLE requires hereditarity of the interactions. Hereditarity means that for every permissible $\gamma$, the point pattern $\gamma\setminus\{x\}$ remains permissible for every $x\in\gamma$, that is any point can be removed from the point pattern. The hardcore interaction in the model \ref{model} does not satisfy this condition. However,~\cite{DL07} \unsure{How does it relate to my case, exactly?}extends MPLE to the non-hereditary case. 

Since some points cannot be removed from the tessellation, we need to introduce the notion of a removable points. A point $x\in\gamma$ is \note{The definition in~\cite{DL07} is actually different and this is given as a proposition.}\unsure{Is there any reason to define it the way~\cite{DL07} does?}removable in $\gamma$ iff $\gamma\setminus\{x\}$ is permissible. We denote $\mathcal R^\beta(\gamma)$ the set of removable points of $\gamma$. Similarly the notion of an addable point will be useful. A point $x\in\gamma$ is addable in $\gamma$ iff $\gamma \cup \{x\}$ is permissible.


In the non-hereditary case, the pseudo-likelihood function then becomes:

\begin{equation}\label{PLL}
PLL_{\Lambda_n}(\gamma,z,\beta, \theta) = \int_{\Lambda'_n} z \exp (-h^{\beta,\theta}(x,\gamma)) dx + \sum_{x\in\mathcal R^\beta(\gamma)\cap \Lambda_n} \big(h^{\beta,\theta}(x,\gamma\setminus\{x\}) - \ln(z)\big),
\end{equation}
where $\Lambda'_n$ is the set of all addable points in $\Lambda_n$ and $h^{\beta,\theta}(x, \gamma \setminus \{x\})$ is \note{Connection between this and Papangelou could be useful}local energy of $x$ in $\gamma$ defined for every $x\in\mathcal R^\beta(\gamma)$ by:
$$h^{\beta,\theta}(x, \gamma \setminus \{x\}) = E^{\beta,\theta}_\Lambda(\gamma_\Lambda, \gamma_{\Lambda^c}) - E^{\beta,\theta}_\Lambda(\gamma_\Lambda\setminus\{x\}, \gamma_{\Lambda^c}).$$

The estimates $\hat\theta$ and $\hat z$ are obtained through minimizing the $PLL_{\Lambda_n}$ function \ref{PLL}:
$$(\hat z, \hat\theta) = \text{argmin}_{z,\theta} PLL_{\Lambda_n} (\gamma, z, \hat\beta,\theta).$$

By differentiating the PLL function \ref{PLL} with respect to $z$, respectivelly $\theta$, and setting them equal to zero, we obtain the estimate for $\hat z$,

\begin{equation}\label{z_hat}
\hat z = \frac{\mbox{card}(\mathcal R^\beta(\gamma)\cap \Lambda_n)}{\int_{\Lambda_n} \exp{\left( -h^{\hat\beta,\theta}(x,\gamma)\right)} dx},
\end{equation}
and the estimate $\hat\theta$ as the solution of

\begin{equation}\label{theta_hat} 
z \int_{\Lambda'_n} (h^{\hat\beta,1}(x,\gamma)\exp{\left(-h^{\hat\beta,\theta}(x,\gamma)\right)}) dx = \sum_{x \in \mathcal R^{\hat\beta}(\gamma)\cap \Lambda_n} h^{\hat\beta,1}(x,\gamma\setminus\{x\}),
\end{equation}
where we have used the fact that the local energy depends on $\theta$ linearly, yielding

$$\frac{\partial h^{\hat\beta,\theta}}{\partial \theta} (x,\gamma) = h^{\hat\beta,1}(x,\gamma).$$

\subsubsection{Practical implementation}
We obtain the estimate of $\theta$ by substituting the expression for $\hat z$ \ref{z_hat} into \ref{theta_hat}.
This leads to the equation
$$ 
\frac{\int_{\Lambda'_n} (h^{\hat\beta,1}(x,\gamma)\exp{\left(-h^{\hat\beta,\theta}(x,\gamma)\right)}) dx} {  \int_{\Lambda_n} \exp{\left( -h^{\hat\beta,\theta}(x,\gamma)\right)} dx} 
= \frac {\sum_{x \in \mathcal R^{\hat\beta}(\gamma)\cap \Lambda_n} h^{\hat\beta,1}(x,\gamma\setminus\{x\})} { \mbox{card}(\mathcal R^\beta(\gamma)\cap \Lambda_n) }. 
$$

In order to simplify the estimation of $\theta$, we can simplify this equation further. First, we denote the righ-hand-side of the equation as $c$ as it  is constant with respect to $\theta$. Second, we note that $x \notin \Lambda_n'  \Rightarrow \exp{\left(-h^{\hat\beta,\theta}(x,\gamma)\right)}= 0$ which enables us to integrate over $\Lambda'_n$ instead of the whole $\Lambda_n$. Lastly we denote the local energy $h^{\hat\beta,1}(x,\gamma) =: h(x)$, yielding the expression
$$ \int_{\Lambda'_n} h(x) \exp{\left(-\theta h(x)\right)} dx = c \int_{\Lambda'_n} \exp{\left(-\theta h(x)\right)}, $$
leading into the final expression
\begin{equation}\label{hat_theta_final} 
\int_{\Lambda'_n} \exp{\left(-\theta h(x)\right)} (h(x) - c) dx .
\end{equation}

The integral \ref{hat_theta_final} is estimated using Monte-Carlo integration, i.e. is approximately equal to
$$ \frac 1N \sum_{i=0}^N 1_{\Lambda'_n}(x_i) \exp{\left( - \theta h_i \right )} (h_i- c) dx $$
where $h_i = h^{\hat\beta,1}(x_i, \gamma)$ and $x_1,\dots,x_N$ is a random sample from the \unsure{Do we need the indicator function if we're only sampling from $\Lambda'_n$ ?}uniform distribution on $\Lambda'_n$

After $\hat\theta$ is estimated, we then obtain the estimate $\hat z$ with $\hat\theta$ instead of $\theta$ and the integral replaced by a MC-integration approximation.


\subsection{Consistency}
\tbd

\chapter{Numerical Results}\label{ch:numeric}


\chapter*{Conclusion}
\addcontentsline{toc}{chapter}{Conclusion}

This text has dealt with stochastic Laguerre-Delaunay tetrahedrizations based on the Gibbs point process. We have presented parametric hard-core models which allow us to control various properties of the resulting tetrahedrizations, such as the surface area of the tetrahedra. Existence of these models has been proven using the approach in \cite{DDG12}. Then we focused on extending the results of \cite{DereudreLavancier2011} in simulation and estimation of model parameters for these models. Simulation was done through a Markov chain Monte Carlo procudure and implemented in \texttt{C++}. Estimation was done through the maximum pseudolikelihood method, extended to the non-hereditary case in \cite{DereudreLavancier2009}.


A considerable number of results is outside the scope of this text.

The first extentsion would be to complete the results in this text --- that is extend the existing results of \cite{DereudreLavancier2009}  in irreducibility of the Markov chain and \cite{DereudreLavancier2011} in consistency of the MPLE estimates to the three-dimensional Laguerre-Delaunay case. As with existence, it is likely the case that the extension to Laguerre would be easier than the extension to three dimensions. 

Second extension is to study the stability of the potentials without the assumptions of non-negativity. The answer depends on the complexity of the tetrahedrization. It seems likely to us that there are relatively easily obtainable results on the complexity of the tetrahedrization of the Poisson process and then perhaps even for the Gibbs point process. Obtaining some results for the tetrahedrization generated by the Poisson process would be interesting in and of itself, as the study of such structures is popular in the field of computational geometry (see e.g. \cite{Amenta07}, \cite{Erickson05}).

A number of other extensions were considered. There are many possibilities in specifying potentials of a different form, in particular studying non-unary potentials by including explicit tetrahedral interactions. This would require new proofs of existence, convergence of MCMC and consistency of the estimation techniques. In the practical sense, it should be relatively easy to extend our \texttt{C++} implementation to include tetrahedral interactions.

Different estimation techniques could be tried. Although we are limited in their choice by the absence of heredity, options do exist. One of them is the variational estimator which is applicable even in the non-hereditary case (\cite{BaddeleyDereudre2013}).

In the practical implementation, an interesting comparison would be to try to use the periodic outside configuration as \cite{DereudreLavancier2011} does.

Finally, instead of using the power distance to define the tetrahedrization, we could use a weighted Euclidean metric (\cite{Gavrilova}), resulting in the dual of the so-called Apollonius diagram, or Johnson-Mehl tessellation. 



%%% Bibliography
\include{bibliography}

%%% Figures used in the thesis (consider if this is needed)
\listoffigures

%%% Tables used in the thesis (consider if this is needed)
%%% In mathematical theses, it could be better to move the list of tables to the beginning of the thesis.
\listoftables

\appendix

\chapter{Appendix: Geometry}\label{appendix}
This appendix investigates some facts and proposition about geometry in $\mathbb R^3$. Since marked points are not present here, the dashed notation introduced in chapter \ref{ch:1} will be dropped.

\problem[inline]{This chapter needs better notation. E.g. $S(p_1,p_2,p_3,p_4)$ for a sphere defined by those points, etc.}
\section{Calculating the circumdiameter}
\todoo[inline]{Check circumdiameter x circumradius, it's a bit confusing in many places}
Here we describe how to calculate the circumdiameter of a $3-$simplex through the Cayley-Menger determinant\cite{Cayley1841}, \cite{Menger28}, \cite{Uspensky48} \todoo{Improve the references (chapter, placement,..)}.

Consider the points $p_1,\dots, p_5 \in \mathbb R^4$ which form a $4$-simplex. Denote $d_{ij} = \|p_i - p_j\|, i,j=1,\dots,5$. Then its area $A$ is given by the \textbf{Cayley-Menger determinant}[ref sommervile]. 

$$
-9216 A^2 =
\begin{vmatrix}
0 & 1 & 1 & 1 & 1 & 1 \\
1 & 0 & d^2_{12} & d^2_{13} & d^2_{14} & d^2_{15} \\
1 & d^2_{21} & 0 & d^2_{23} & d^2_{24} & d^2_{25}  \\
1 & d^2_{31} & d^2_{32} & 0 & d^2_{34} & d^2_{35} \\ 
1 & d^2_{41} & d^2_{42} & d^2_{43} & 0 & d^2_{44} \\
1 & d^2_{51} & d^2_{52} & d^2_{53} & d^2_{54} & 0 
\end{vmatrix} 
$$

Now consider non-coplanar points $p_1,\dots, p_4 \in \Rt$ forming a $3$-simplex, i.e. a tetrahedron. To obtain the circumradius of this tetrahedron, we imagine $p_1,\dots, p_4$ to lie on a $3$-dimensional hyperplane $H$ in $\mathbb R^4$ and we consider the point $c \in H$ such that $\|c-p_i\| = r \;\forall i=1,\dots,4$ $r\in \mathbb R$. The point $c$ is, by definition, the center of the circumsphere of $p_1,\dots,p_4$ and $d$ is the circumradius. The circumradius $r$ can be obtained using the Cayley-Menger determinant, since $p_1,\dots,p_4,c$ now form a $4$-dimensional simplex of volume $0$. We therefore have 


\begin{equation}\label{eq:CMeq}
0 = 
\begin{vmatrix}
0 & 1 & 1 & 1 & 1 & 1 \\
1 & 0 & d^2_{12} & d^2_{13} & d^2_{14} & r^2 \\
1 & d^2_{21} & 0 & d^2_{23} & d^2_{24} & r^2 \\
1 & d^2_{31} & d^2_{32} & 0 & d^2_{34} & r^2 \\ 
1 & d^2_{41} & d^2_{42} & d^2_{43} & 0 & r^2 \\
1 & r^2 & r^2 & r^2 & r^2 & 0 \\
\end{vmatrix}, 
\end{equation}

where we again have   $d_{ij} = \|p_i - p_j\|, i,j=1,\dots,4$.\newline  

It would be possible to solve \ref{eq:CMeq} as an equation of $r$. A better approach is to  subtract $r^2$ times the first row from last and subtract $r^2$ times the first column from the last to obtain the determinant 



$$
\begin{vmatrix}
0 & 1 & 1 & 1 & 1 & 1 \\
1 & 0 & d^2_{12} & d^2_{13} & d^2_{14} & 0 \\
1 & d^2_{21} & 0 & d^2_{23} & d^2_{24} & 0 \\
1 & d^2_{31} & d^2_{32} & 0 & d^2_{34} & 0 \\ 
1 & d^2_{41} & d^2_{42} & d^2_{43} & 0 & 0 \\
1 & 0 & 0 & 0 & 0 & -2r^2 \\
\end{vmatrix}. 
$$

By expanding by the last row, we obtain the equation

$$
2r^2 \begin{vmatrix}
0 & 1 & 1 & 1 & 1 \\
1 & 0 & d^2_{12} & d^2_{13} & d^2_{14} \\
1 & d^2_{21} & 0 & d^2_{23} & d^2_{24} \\
1 & d^2_{31} & d^2_{32} & 0 & d^2_{34} \\ 
1 & d^2_{41} & d^2_{42} & d^2_{43} & 0 \\
\end{vmatrix} 
-
\begin{vmatrix}
1 & 1 & 1 & 1 & 1 \\
0 & d^2_{12} & d^2_{13} & d^2_{14} & 0 \\
d^2_{21} & 0 & d^2_{23} & d^2_{24} & 0 \\
d^2_{31} & d^2_{32} & 0 & d^2_{34} & 0 \\ 
d^2_{41} & d^2_{42} & d^2_{43} & 0 & 0 \\
\end{vmatrix} = 0,
$$

from which $r^2$ is directly expressible.

\begin{equation}\label{eq:Cayley-Menger-expanded}
r^2 
=
\frac{
\begin{vmatrix}
1 & 1 & 1 & 1 & 1 \\
0 & d^2_{12} & d^2_{13} & d^2_{14} & 0 \\
d^2_{21} & 0 & d^2_{23} & d^2_{24} & 0 \\
d^2_{31} & d^2_{32} & 0 & d^2_{34} & 0 \\ 
d^2_{41} & d^2_{42} & d^2_{43} & 0 & 0 \\
\end{vmatrix}}
{2 \begin{vmatrix}
0 & 1 & 1 & 1 & 1 \\
1 & 0 & d^2_{12} & d^2_{13} & d^2_{14} \\
1 & d^2_{21} & 0 & d^2_{23} & d^2_{24} \\
1 & d^2_{31} & d^2_{32} & 0 & d^2_{34} \\ 
1 & d^2_{41} & d^2_{42} & d^2_{43} & 0 \\
\end{vmatrix} 
}.
\end{equation}

It is worth noting that the determinant in the quotient cannot equal zero, since it is again a Cayley-Menger determinant and we assumed $p_1,\dots,p_4$ to be non-coplanar. 



\section{Bounding the circumdiameter}
This section proves the bound used in [ref].

\subsection{Statement of the problem}

The problem of founding the bounds can be stated as the following two optimization problems. \newline

\noindent For the tetrahedron $T_1$, the problem is 
\begin{equation}\label{prob:tetra1}
\begin{aligned}
& \underset{p_1,p_2,p_3,p_4\in \Rt}{\text{maximize}}
& & \delta(\{p_1,p_2,p_3,p_4\}) \\
& \text{subject to}
& & \| p_i - t_i\| \leq \rho a, t_i\in \Rt, i=1,2,3,4, \\
& & &\|t_i - t_j\| = a, i=1,2,3,4. 
\end{aligned}
\end{equation}

\noindent To state the problem for the tetrahedron $T_2$, first denote 
$$D = \begin{pmatrix}
0 & \sqrt a & a & a \\
\sqrt a & 0 & a & a \\
a & a & 0 & a\\
a & a & a & 0
\end{pmatrix},
$$
and denote the entries of matrix $D$ as $d_{ij}, i,j=1,2,3,4$. Then the statement is:
\begin{equation}\label{prob:tetra2}
\begin{aligned}
& \underset{p_1,p_2,p_3,p_4\in \Rt}{\text{maximize}}
& & \delta(\{p_1,p_2,p_3,p_4\}) \\
& \text{subject to}
& & \|p_i - t_i\| \leq \rho a, t_i\in \Rt, i=1,2,3,4, \\
& & & \|t_i-t_j\| = d_{ij}, i,j=1,2,3,4.\\
\end{aligned}
\end{equation}
This is a non-linear optimization problem. We can arrive at its solution through some careful geometric arguments.

\subsection{Solution to the problem}
First, define the \textit{circumdiameter function} of point $p \in \Rt$ with respect to non-collinear points $p_1,p_2,p_3 \in \Rt$:
$$c(p) = \delta(\{p,p_1,p_2,p_3\}).$$
Denote $(x_i,y_i,z_i)$ the coordinates of $p_i, i=1,\dots,3$. The following lemma describes the properties of $c(p)$.

\todoo[inline]{Define $\delta$ for triangles, too}
\begin{lemma} $c(p)$ is continuous, has a global minimum $c_{min} := \delta(\{p_1,p_2,p_3\})$ and level sets 
	$$L_a := \{p \in \Rt: c(p)=a\} = S_{a1} \cup S_{a2}, \quad a \geq c_{min},$$ 
	where $S_{a1}$ and $S_{a2}$ are two spheres with diameter $a$ such that $p_1,p_2,p_3 \in S_{a1}\cap S_{a2}$. Furthermore, the centers $c_1, c_2$ of $S_{a1},S_{a2}$ respectivelly, lie \todoo{Improve the wording}in the halfspaces
$$H_+ = \{x \in \Rt: Ax \geq 0 \},\; H_- = \{x \in \Rt: Ax \leq 0\},$$
where $A$ defines the hyperplane $H=\{x\in\Rt: Ax = 0\}$ on which $p_1,p_2,p_3$ lie.
\end{lemma}
\begin{proof}
	Continuity: From \ref{eq:Cayley-Menger-expanded} we see that $c(p)$ can be seen as a composition of a norm, determinants and division. Determinant is continuous as a function of elements of the matrix since it is a polynomial function. Thus $c(p)$ is continuous.\newline

\noindent We can rewrite $L_a$ as
$$\{p \in \Rt: \exists \text{ sphere } S \text{ s.t. } p_1,p_2,p_3,p \in S \text{ and diam}S = a\}.$$
We must therefore find the number of spheres going through the points $p_1,p_2,p_3$ with the diameter $a$. Denote $S$ a sphere such that $\{p_1,p_2,p_3\}\subset S$ with $\mathrm{diam}(S)=a$. Define the hyperplanes
$$H_{12} = \{x\in\Rt: \|x-p_1\| = \|x-p_2\|\}, \;\; H_{23} = \{x\in\Rt: \|x-p_2\|=\|x-p_3\|\}.$$
The intersection $H_{12}\cap H_{23}$ is a line $L$, as $p_1,p_2,p_3$ are non-collinear.  The center of $S$ is at distance $a/2$ from all three points and thus lies on $L$. For any point, there are at most two points on the line $L$ at a given distance from the point. This proves that there are at most two spheres satisfying the definition of $S$.

Using the line $L$, we can also deduce the rest of the proposition. The point on $L$ at a minimum distance to $p_1,p_2,p_3$ is the point $p_{min}:=L\cap H$. We know that $p_{min}$ is equidistant from $p_1,p_2,p_3$ and that it lies on the hyperplane $H$, therefore it is the circmradius of the triangle defined by $p_1,p_2,p_3$ and we have $c(p_{min}) = \delta(\{p_1,p_2,p_3\})$.  

\todoo[inline]{If possible, simplify this argument}
To see that $c_1$ and $c_2$ must be (non-strictly) separated by the hyperplane $H$, assume WLOG $\{c_1,c_2\}\subset H_+, c_1\neq c_2$. Let $p \in S_{a1}$ and let  $p_R\in \Rt$ be the reflection of $p$ through the hyperplane $H$. The tetrahedron $p_1,p_2,p_3,p_R$ then is a reflection of the tetrahedron $p_1,p_2,\dots, p$ and therefore its circumsphere has diameter $a$. However, its centre lies in $H_-$, which is a contradiction. 
\end{proof}

Note that $S_{a1}$ and $S_{a2}$ are not necessarily distinct. In fact, we can see from the proof that $S_{a1}=S_{a2}$ precisely when $a=c_{min}$.


We are now ready to characterize the set of solutions to \ref{prob:tetra1} and \ref{prob:tetra2}. For the next proposition, we say a point lies ``inside'' or ``outside'' of the sphere $S$ if the the point lies in $B$ or in $B^c$ respectivelly, where $B$ is the closed ball such that $\partial B = S$.


\begin{proposition}\label{prop:Apollonius}
Any solution $(p_1,p_2,p_3,p_4)$ of the problem \ref{prob:tetra1} will lie on a sphere $S$ that is (internally or externally) tangent to the spheres $\partial B(t_i,\rho a), i =1,2,3,4$. 
\end{proposition}
\begin{proof}
	Let $(p_1,p_2,p_3,p_4)$ be a solution of \ref{prob:tetra1}. Denote $c(p_1)=\delta(\{p_1,p_2,p_3,p_4\})=c$ and $S$ the circumshphere of $\{p_1,\dots,p_4\}\subset S$. 
	First, WLOG assume that $p_1 \in B(t_1,\rho a)$. Because $p_1$ maximizes the function $c(p)$, we have $c(p_1)\geq c(p), p\in U$, where $U$ is some small neighborhood of $p_1$. Choose two points, $p_O,p_I\in U\setminus S$ such that 
\begin{enumerate} 
\item $c(p_O)=c(p_I)=b$,
\item $p_I$ is on the inside of $S$ and $p_0$ on the outside of $S$ 
\item \todoo{Define this notation}$S(p_I,p_2,p_3,p_4)$ and $S(p_O,p_2,p_3,p_4)$ do not equal and their centers lie on the same halfspace ($H_+$ or $H_-$) as $S$. 
\end{enumerate}
Such choice is possible due to continuity of $c(p)$. Yet we arrive at a contradiction, as the level set $L_b$ now contains two distinct spheres with centres in the same halfspace. 

Assume now that $p_1 \in \partial B(t_1,\rho a)=: S_1$. We now choose $p_I$ and $p_O$ with the additional requirement that they must both lie on $\partial B(t_1,\rho a)$. Such choise is not possible precisely when $S_1$ and $S$ are tangent, since then \todoo{Make sure "inside" a sphere has a clear meaning} $S_1$ lies either completely inside or outside $S$ and it is no longer possible to choose points both outside and inside. 
\end{proof}
\unsure{Is this a good way of proceeding?}Note that proposition \ref{prop:Apollonius} is formulated for problem \ref{prob:tetra1}. However, we could repeat the same exact argument for \ref{prob:tetra2} and thus the same holds for both problems.\newline

We have found that the solutions to \ref{prob:tetra1} and \ref{prob:tetra2} must lie on a sphere that tangent to the spheres within which points can move. This is a dramatic improvement --- we have narrowed the previously infinite space of possible solutions down to just $2^4=16$ possible quadruples of points (and even less beacause of symmetries). We also note that the set of solutions to our problem is precisely the set of solutions of a three-dimensional equivalent of the more than two thousand years old \todoo{Encoding problem} \textbf{Apollonius problem} 
% \cite{GischRibando2006}.


\subsection{Apollonius problem in $\mathbb R^3$}
We want to find all the spheres that are externally or internally tangent to the spheres $\partial B(t_i,\rho a), i=2,3,4$ as defined in problems \ref{prob:tetra1} and \ref{prob:tetra2}.

First note that two externally tangent spheres $S_1=((x_1,y_1,z_1),r_1), S_2=((x_2,y_2,z_2),r_2)$  satisfy
$$\|(x_1,y_1,z_1) - (x_2,y_2,z_2)\| = r_1+r_2.$$
Similarly, two externally tangent spheres satisfy
$$\|(x_1,y_1,z_1) - (x_2,y_2,z_2)\| = |r_1 - r_2|.$$
By squaring both equations, we obtain the equality
$$(x_1-x_2)^2 + (y_1-y_2)^2 + (z_1-z_2)^2 = (r_1 \pm r_2)^2$$
Where we use $+$ for externally and $-$ for internally tangent spheres.

The Apollonius problem for spheres $S_1,S_2,S_3,S_4$ is therefore solved by any $S=((x,y,z),r)$ such that
\begin{align}\label{eq:Apollonius}
  (x_1-x)^2 + (y_1-y)^2 + (z_1-z)^2 &= (r_1 \pm r)^2 \\
  (x_2-x)^2 + (y_2-y)^2 + (z_2-z)^2 &= (r_2 \pm r)^2 \nonumber \\
  (x_3-x)^2 + (y_3-y)^2 + (z_3-z)^2 &= (r_3 \pm r)^2 \nonumber \\
  (x_4-x)^2 + (y_4-y)^2 + (z_4-z)^2 &= (r_4 \pm r)^2, \nonumber  
\end{align}

where we can take any combination of $+$ or $-$, yielding altogether $16$ possible solutions. We do not consider degenerate cases as they cannot happen in our setting. 

As noted previously, the number of solutions for both $T_1$ and $T_2$ will reduce significantly. For $T_1$, the spheres are completely interchangeable and thus only solutions with different number of $+$ will differ. This yields $5$ possible solutions. Geometrically the number of $+$ can be seen as the number of spheres the solution is externally tangent to. For $T_2$ the situation is more complex, as the problem isn't entirely symmetric with respect to the four points. Still, symmetries do exist and the number of solution will be reduced.  

Sadly, for most choices of $+$ and $-$, these equations still seem to be too complex for Mathematica to solve. Luckily, we can simplify them further. 

\subsubsection{Solving the equations \ref{eq:Apollonius} by linearizing}

We formulate the solution as a proposition.
\begin{proposition}\label{prop:Apollonius}
	For $\rho < 1/(2\sqrt 6)$, the maximum in \ref{prob:tetra1} is
	$$\delta_1 := 2a(\sqrt 6/4 + \rho).$$
	For $\rho < 1/4$, the maximum in \ref{prob:tetra2} is
	$$\delta_2 := 2a \frac{2\rho + \sqrt{2 - 32\rho^2 + 64 \rho^4}}{2-32\rho^2}.$$

\end{proposition}
\begin{proof}
	Recall that the solution must lie on a sphere solving the equations \ref{eq:Apollonius}. We must therefore solve them and find the solution with the largest circumdiameter.

First, for clarity, we define the variables $s_i\in\{+1,-1\},i=1,\dots,4$ instead of relying on the notation $\pm$. We begin by expanding the parentheses to obtain the equations
$$x^2+y^2+z^2 + x_i^2 + y_i^2 + z_i^2 - 2xx_i - 2yy_i - 2zz_i = r^2 + r_i^2 + 2(s_1 r_1 - s_2 r_2)r,\quad i=1,2,3,4$$

By subtracting the $2,3,4$-th equation from the first, we get rid of the quadratic terms and obtain a system of linear equations with four variables and three equations:
\begin{align*}-&2(x_1-x_i)x - 2(y_1-y_i)y -2 (z_1-z_i)z - 2(s_1r_1 - s_2r_2)r \\
	& + x_1^2-x_i^2 + y_1^2-y_i^2 + z_1^2 - z_i^2 -r_1^2 + r_i^2 = 0, \quad i=2,3,4
\end{align*}
This system can be solved to obtain expression of $x,y,z$ in terms of $r$. We then substitute those expression into \ref{eq:Apollonius} to obtain $r$\footnote{Note that exact solutions of $x,y,z$, which we are not interested in, could then by obtained through substituting $r$ back into the linear system.}. 

We have used Wolfram Mathematica \cite{Mathematica} to find the solutions. The full implementation can be found in the file \texttt{ApolloniusProblem.nb}. By comparing the circumdiameters of the solutions, we obtain the proposition.

\end{proof}
All the solutions for the choice $a=1$ can be seen in figures \ref{fig:Apollonius1} and \ref{fig:Apollonius2}. We can see that for $T_1, \rho < 1/\sqrt 6$, we have the two solutions
$$a(\sqrt 6 / 4 + \rho), a \frac{\rho - \sqrt 6 (4\rho^2 - 1)}{4-24 \rho^2}$$
which itersect at $\rho =1/(2\sqrt 6)$. 

Notice the simple linear form of the first solution --- it is precisely the sphere which is internally tangent to all four spheres. This sphere has the same center as the circumsphere of tetrahedron $\{t_1,t_2,t_3,t_4\}$. Thus the solution is a sum of circumradius of the tetrahedron, $\sqrt 6 /4$, and the radius of the four spheres, $\rho$. We can see similar behaviour in the solution that is externally tangent to all four spheres.

For $T_2$, the linear solution will no longer be the largest, as now we obtain a larger circumradius by using a sphere that is externally tangent to some of the spheres. 


\begin{remark}\label{rem:GP}[General position]
	From the form of the solutions one can also obtain the necessary bounds for $\rho$ for the points to remain in general position. The points cease to be in general position precisely when any one of the solutions becomes infinite. This gives us $\rho < 1/(2\sqrt 6)$ for $T_1$ and $\rho<1/4$ for $T_2$. Since we must control the circumdiamteter for all tetrahedra, we must assume $\rho < 1/4$.
\end{remark}


\begin{figure}
	\includegraphics[width=1\textwidth]{../img/t1.pdf}
	\caption{All solutions to Apollonius problem with $T_1$, $a=1$.}\label{fig:Apollonius1}
\end{figure}

\begin{figure}
	\includegraphics[width=1\textwidth]{../img/t2.pdf}
	\caption{All solutions to Apollonius problem with $T_2$, $a=1$}\label{fig:Apollonius2}
\end{figure}

\chapter{Appendix: Implementation details}\label{appendix:implementation}
The latest version is available at \url{https://github.com/DahnJ/Gibbs-Delaunay}\todoo{Actually, change the url to reflect the Laguere case}.
\section{C++ and CGAL}
\tbd
\section{Python analysis}
\tbd
\section{Mathematica}
\tbd

%%% Abbreviations used in the thesis, if any, including their explanation
%%% In mathematical theses, it could be better to move the list of abbreviations to the beginning of the thesis.


%%% Attachments to the master thesis, if any. Each attachment must be
%%% referred to at least once from the text of the thesis. Attachments
%%% are numbered.
%%%
%%% The printed version should preferably contain attachments, which can be
%%% read (additional tables and charts, supplementary text, examples of
%%% program output, etc.). The electronic version is more suited for attachments
%%% which will likely be used in an electronic form rather than read (program
%%% source code, data files, interactive charts, etc.). Electronic attachments
%%% should be uploaded to SIS and optionally also included in the thesis on a~CD/DVD.
%%% Allowed file formats are specified in provision of the rector no. 72/2017.


% \listoftodos
\openright
\end{document}
