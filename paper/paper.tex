\documentclass{kybernetika}
%the class kybernetika includes these packages:
% graphicx, amssymb, amsmath

% used environment for theorems:

\newtheorem{theorem}{Theorem}[section]
\newtheorem{lemma}[theorem]{Lemma}
\newtheorem{proposition}[theorem]{Proposition}
\newtheorem{corollary}[theorem]{Corollary}
\newtheorem{remark}[theorem]{Remark}
\newtheorem{fact}[theorem]{Fact}
\newtheorem{example}[theorem]{Example}
\newtheorem{definition}[theorem]{Definition}
\newtheorem{observation}[theorem]{Observation}

\usepackage{xargs}  % newcommandx
\usepackage{xcolor} % todonotes colors
\usepackage{enumerate}
% \usepackage[colorinlistoftodos,prependcaption,textsize=tiny]{todonotes}

% \newcommandx{\problem}[2][1=]{\todo[linecolor=red,backgroundcolor=red!25,bordercolor=red,#1]{#2}}
% \newcommandx{\todoo}[2][1=]{\todo[linecolor=blue,backgroundcolor=blue!25,bordercolor=blue,#1]{#2}}
% \newcommandx{\note}[2][1=]{\todo[linecolor=OliveGreen,backgroundcolor=OliveGreen!25,bordercolor=OliveGreen,#1]{#2}}
% \newcommandx{\unsure}[2][1=]{\todo[linecolor=violet,backgroundcolor=violet!25,bordercolor=violet,#1]{#2}}


\newcommand{\Rt}{{\mathbb R^3}}
\newcommand{\x}{{\mathbf{x}}}




\begin{document}
\pagestyle{myheadings}




\title{Existence and simulation of \\ Gibbs-Laguerre-Delaunay tetrahedrizations} 

% all authors
\author{Daniel Jahn}

% contact information of authors. Use one \contact command per person.
% Specify name(s), surname(s), postal address and e-mail.
\contact{Daniel}{Jahn}{Department of Probability and Mathematical Statistics, Charles University, Sokolovsk\'{a} 83, 186 75 Praha 8, Czech Republic}{jahn@karlin.mff.cuni.cz}

% heading contains authors and title of the paper, please shorten the title if necessary
% (it will appear only at the heading of every even page)
\markboth{D. Jahn} {Existence and simulation of GLD tetrahedrizations}

\maketitle

\begin{abstract}
We present revolutionary unprecedented results using newly developed methods. Songs will be sung about this paper. 
\end{abstract}


% here please insert two or more keywords appropriate to the topic of your paper
\keywords{Gibbs point process,
Laguerre-Delauay triangulation,
tetrahedralization,
MCMC simulation}

% here please insert one, two or more (5-digit) "Mathematics Subject Classification" codes appropriate to the topic of your paper
% you can find and choose them at the web page of the American Mathematical Society: http://www.ams.org/msc/msc2010.html
% \classification{00A01, 00M02, 00S03}
\classification{60K35, 60G55}
% 
% 60K35  Interacting random processes; statistical mechanics type models; percolation theory 
% 60G55  Point processes
% 60D05  Geometric probability, stochastic geometry, random sets [See also 52A22, 53C65]

\section{INTRODUCTION}
This is an interesting field. We did some things in this interesting field. Here they are.

\section{PRELIMINARIES}
See \cite{D17}.
\begin{itemize}
\item Laguerre geometry
\item Gibbs point process - already tie in \cite{DDG12}.
\end{itemize}

\section{EXISTENCE}
Basically \cite{DDG12}.

\begin{itemize}
\item Specification of the model
\item Proof of existence
\end{itemize}

\noindent \textbf{Smooth interaction}:  For $\eta\in\mathcal {LD}(\x)$ define the potential $\varphi_S$ as a unary potential such that
$$\varphi_S(\eta,\x) \leq K_0 + K_1 \chi(\eta)^{\beta}$$
for some $K_0,K_1 \geq 0, \beta >0$\newline
\textbf{Hard-core interaction}: For $\eta\in\mathcal {LD}(\x)$ define the potential $\varphi_{HC}$ as a unary potential such that
$$\sup_{\eta: d_0 \leq \chi(\eta) \leq d_1} \varphi_{HC}(\eta,\x)  < \infty \text{ and } \varphi_{HC}(\eta,\x)=\infty \text{ if } \chi(\eta)>\alpha$$ 
for some $0\leq d_0 < d_1 \leq \alpha$. \unsure{How exactly does this look? Why?}

\subsection{Existence theorems}


\begin{theorem}
	For every hypergraph structure $\mathcal E$, hyperedge potential $\varphi$ and activity $z>0$ satisfying \textbf{(S)}, \textbf{(R)} and \textbf{(U)} there exists at least one Gibbs measure.
\end{theorem}

\begin{theorem}
	For every hypergraph structure $\mathcal E$, hyperedge potential $\varphi$ and activity $z>0$ satisfying \textbf{(S)}, \textbf{(R)} and \textbf{(\^{U})} there exists at least one Gibbs measure.
\end{theorem}

Proofs of both theorems can be found in \cite{DDG12}, see also Remark 3.7. in the same paper about the marked case.




\begin{enumerate}[\textbf{(S)}] 
	\item \textit{Stability}. The energy function $H$ is called \textit{stable} if there exists a constant $c_S \geq 0$ such that 
		$$H_{\Lambda,\x}(\zeta) \geq -c_S \cdot \mathrm{card}(\zeta \cup \partial_\Lambda \x)$$
		for all $\Lambda \in \mathcal B_0, \zeta \in \mathbf N_\Lambda, \x \in \mathbf N^\Lambda_{\text{cr}}$.
\end{enumerate}



\begin{enumerate}[\textbf{(R)}]\label{(R)}
	\item \textit{Range condition}. There exist constants $\ell_R,n_R \in \mathbb N$ and $\chi_R < \infty$ such that for all $(\eta,\x) \in \mathcal E$ there exists a finite horizon $\Delta$ satisfying: For every $x,y \in \Delta$ there exist $\ell$ open balls $B_1, \dots, B_\ell$ (with $\ell \leq \ell_R$) such that
	\begin{enumerate}[-]
		\item the set $\cup^\ell_{i=1} \bar B_i$ is connected and contains $x$ and $y$, and 
		\item for each $i$, either $\text{diam} B_i \leq \chi_R$ or $N_{B_i}(\x) \leq n_R$.
	\end{enumerate}
\end{enumerate}

\noindent Fix some $A \subset C\times S$ and define
$$\Gamma^A = \{\zeta \in \mathbf N_C: \zeta = \{p\}, p \in A\},$$
the set of configurations consisting of exactly one point in the set $A$. The set of pseudo-periodic configurations $\bar\Gamma$ thus contains only one point in each $C(k), k~\in~\mathbb Z^3$.

Let $M$ be such that $|M_i| = a > 0$ for $i=1,2,3$ and $\angle(M_i,M_j) = \pi / 3$ for $i\neq j$.
For $\mathcal {LD}$ models we choose 
\begin{equation}\label{eq:choiceA} A = B(0,b)\times \left[0, \sqrt{\frac a2(1-2\rho)}\right] \end{equation}
in order for balls to never overlap. We further assume $\rho < 1/4$, see Appendix \ref{sec:boundingdiameter}.


\begin{enumerate}[\textbf{(U)}] 
	\item \textit{Upper regularity}. $M$ and $\Gamma$ can be chosen so that the following holds. 
		\begin{enumerate}[(U1)]
			\item \textit{Uniform confinement}: $\bar \Gamma \subset \mathbf N^\Lambda_\text{cr}$ for all $\Lambda \in \mathcal B_0$ and 
				\begin{equation}\label{eq:U1}r_\Gamma := \sup_{\Lambda\in\mathcal B_0}\sup_{\x \in \bar\Gamma} r_{\Lambda, \x} < \infty. \end{equation}
			\item \textit{Uniform summability}: 
			$$c_\Gamma := \sup_{\x \in \bar\Gamma}  \sum_{\eta \in \mathcal E(\x): \eta' \cap C \neq \emptyset} \frac{\varphi(\eta,\x)}{\#(\hat\eta)} < \infty,$$
where $\hat\eta := \{k \in \mathbb Z^3: \eta \cap C(k) \neq \emptyset\}$.
\item \textit{Strong non-rigidity}: $e^{z|C|} \Pi^z_C(\Gamma) > e^{c_\Gamma}$.
		\end{enumerate}
\end{enumerate}

For some models it is possible to replace the upper regularity assumptions by their alternative and prove the existence for all $z>0$.

\begin{enumerate}[(\textbf{\^{U}})]
	\item \textit{Alternative upper regularity}. $M$ and $\Gamma$ can be chosen so that the following holds.
	\begin{enumerate}[(\^U1)]
		\item \textit{Lower density bound}: There exist constants $c,d > 0$ such that 
			$$\mathrm{card}(\zeta) \geq c|\Lambda| - d$$
			whenever $\zeta \in \mathbf N_f\cap\mathbf  N_\Lambda$ is such that $H_{\Lambda,\x}(\zeta)<\infty$ for some $\Lambda \in \mathcal B_0$ and some $\x \in \bar\Gamma$.
		\item = (U2) \textit{Uniform summability}.
		\item \textit{Weak non-rigidity}: $\Pi^z_C(\Gamma) > 0$.
	\end{enumerate}
\end{enumerate}


\subsection{Verification}
A useful mental model of how to think of the class $\bar \Gamma$ is to start from a configuration 
$$\x_0 = \{(M_ak,0) \in \Rt\times S: k \in \mathbb Z^3\} \in \bar\Gamma,$$ 
with points at the centers of the set $A$ and then imagine any configuration $\x \in \bar \Gamma$ as a perturbed version of $\x_0$.
While in the two-dimensional case the point configuration forms a tessellation out of equilateral triangles (up to a perturbation), the three-dimensional case results into the so-called tetrahedral-octahedral honeycomb (up to a perturbation). 
This tessellation, if tetrahedrized, contains two different types of tetrahedra: a regular tetrahedron with side length $a$ and an irregular tetrahedron with side lengths $(a,a,a,a,a,\sqrt 2a)$, again, up to a perturbation. We will refer to a tetrahedron that is a perturbed version fo the regular tetrahedron as $T_1$ and similarly $T_2$ for the irregular tetrahedron. 

In the tetrahedron-octahedron tessellation, each \todoo{Reference, possibly using Schlafli symbols} vertex is incident to eight regular tetrahedra and six regular octahedra. 
Since each octahedron contains four tetrahedra, we obtain the bound $n_T\leq 8+6\cdot 4 = 32$.


To show that the range of interactions is limited for configurations in $\bar \Gamma$ two quantities need to be shown to be uniformly bounded. 
The first is the circumdiameter of the tetrahedra. 
For $T1$ tetrahedra, we obtain the bound

$$\chi_1(\rho) := 2(\sqrt 6/4 + \rho)$$

and for T2, we obtain the bound
$$\chi_2(\rho) := 2 \frac{2\rho + \sqrt{2 - 32\rho^2 + 64 \rho^4}}{2-32\rho^2},$$
where $\rho < 1/4$. 


The second quantity to be bounded is the weight of a characteristic point. Since the perturbation happens on a bounded window and the points' weights are bounded, this amounts to proving that the points cannot come arbitrarily close to, or even attain, a coplanar position.
 However, this is equivalent to the boundedness of the circumdiameter of the tetrahedron and is guaranteed by Theorem [ref].


We first prove that a uniformly bounded finite horizons imply uniform confinement. 


\begin{lemma}\label{lemma:U1}
	Let $\Gamma \subset \mathbf N_{lf}$ be a class of configurations. If there exists $d_{max}>0$ such that $\mathrm{diam}\Delta < d_{max}$ for the horizon $\Delta$ of any $(\eta,\x), \eta \in \mathcal E(\x), \x \in \Gamma$, then 
	$$r_\Gamma < d_{max}.$$
\end{lemma}
\begin{proof}
	Choose $\Lambda\in \mathcal B_0$ and $\x \in \Gamma$. Let $\zeta \in \mathbf N_\Lambda$,  $\eta \in \mathcal E_\Lambda(\zeta \cup \x_{\Lambda^c})$ and denote $\Delta$ the finite horizon of $(\eta,\x)$. From Lemma \ref{lemma:horizEset} we obtain $\Delta\cap\Lambda \neq \emptyset$. Then $\Delta \subset \Lambda + B(0,d_{max})$. If we take $\tilde\x \in \Gamma$ such that $\tilde\x = \x$ on $\partial\Lambda(\x)$ then $\varphi(\eta,\zeta \cup \x_{\Lambda^c}) = \varphi(\eta,\zeta\cup \tilde \x_{\Lambda^c})$ since $\zeta\cup\x_{\Lambda^c}$ and $\zeta\cup\tilde\x_{\Lambda^c}$ differ only on $\Delta^c$.
\end{proof}




\begin{theorem}\label{thm:E3}
	There exists at least one Gibbs measure for the model $(\mathcal {LD}_4,\varphi_S)$ and every activity 
	$$z> \frac{3\sqrt 2}{4\pi}e^{14 K_0}   (4K_1 \beta e^{7/2}/7)^{1/\beta} \frac{(\chi_1(\rho)^\beta + 6\chi_2(\rho)^\beta)^{1/\beta}}{\rho^3 \sqrt{1-2\rho}}.$$
\end{theorem}
\begin{proof}
\begin{enumerate}[]
	\item \textbf{(R)} Take the horizon set $\Delta = B(p'_\eta, \sqrt{p''_\eta + W})$. $\Delta$ can be decomposed into the sphere $p_\eta$ and $\Delta \setminus p_\eta$, a 3-dimensional annulus with width 
		$$\sqrt{p''_\eta+W} -\sqrt{p''_\eta}=W/(\sqrt{p''_\eta+W} + \sqrt{p''_\eta}).$$
		By definition of $\mathcal {LD}$ and Remark \ref{r:horizons}, $p_\eta$  cannot contain any points of $\x$. Although the annulus $\Delta \setminus p_\eta$ does not have any bound on the number of points, its width is bounded by $\sqrt W \geq  W/(\sqrt{p''_\eta+W} + \sqrt{p''_\eta})$. This means that any $x,y\in \Delta$ can be connected by the spheres $B(x,\sqrt W), p_\eta, B(y,\sqrt W)$, yielding the parameters $\ell_R = 3,n_R=0,\chi_R=2\sqrt W$.
	\item \textbf{(S)} Stability is satisfied because of $\varphi$ is non-negative.
	\item \textbf{(U)} We choose $M$ and $\Gamma$ as in Section \ref{sec:MGamma}.
		\begin{enumerate}[(U1)]
			\item By Proposition \ref{prop:maxPeta} there is $C>0$ such that $p''_\eta\leq C$ for all $\eta \in \mathcal {LD}_4(\x), \x \in \bar\Gamma^A$. By Lemma \ref{lemma:U1} we have $r_\Gamma\leq \sqrt{C + W}$.
			\item is trivial since $n_S<56$ and $\varphi_{S}$ is bounded by Proposition \ref{prop:maxcircum}.
			\item We proceed similarly as in Theorem \ref{thm:E1} and obtain
				$$z>C_0 e^{C_1 a^\beta} / a^{7/2}$$
				where $C_0=3 \sqrt 2 e^{14K_0} / (4\pi \rho^3 \sqrt{1-2\rho})$, $C_1 = 2K_1(\chi_1^\beta + 6\chi_2^\beta)$. Optimizing over $a$ we obtain $a=(7/(2C_1\beta))^{1/\beta}$ arriving at the bound
				$$z> C_0 (C_1 \beta e^{7/2} / (7/2))^{1/\beta}.$$
		\end{enumerate}
\end{enumerate}
\end{proof}



\begin{theorem}\label{thm:E4}
	There exists at least one Gibbs measure for the model $(\mathcal {LD}_4,\varphi_{HC})$ and every activity $$z>0.$$
\end{theorem}
\begin{proof}
\begin{enumerate}[]
	\item \textbf{(R)} The horizon set is $\Delta = B(p'_\eta,\sqrt{p''_\eta +W})$. Parameters can be chosen as in Theorem \ref{thm:E3}. \unsure{Is it a problem that there's no $n_R$ circle? Cause the proof suggested something like that?}
	\item \textbf{(S)} Stability is satisfied because of $\varphi$ is non-negative.
	\item \textbf{(\^U)} We choose $M$ and $\Gamma$ as in Section \ref{sec:MGamma}.
		\begin{enumerate}[(U1)]
			\item The same as in Theorem \ref{thm:E2}. Although the underlying structure is different, the potential still depends on $\chi(\eta)$ and (\^U1) requires the configuration to have a non-infinite energy.
			\item The same as in Theorem \ref{thm:E2}, $n_S<56$ and we choose an appropriate $a$ and $\rho$.
			\item $\Pi^z_\Lambda(\Gamma)>0$ by Remark \ref{r:UA}.
		\end{enumerate}
\end{enumerate}
\end{proof}



Using the same approach and the same pseudo-periodic configurations $\bar\Gamma$, it is easy enough to prove the existence of many different forms of unary hyperedge potentials. In the following, we suggest some alternate hyperedge potentials. 


\begin{remark}[Other smooth interaction potentials]\label{r:otherpotentials}
	
	Alternate smooth interaction models could be considered using characteristics of $k$-faces of $\eta$ instead of the circumdiameter. An example is the volume potential $\varphi_{V}$ defined as a unary potential such that for $\eta\in\mathcal E_4(\x), \x \in \bar\Gamma$  we have
	$$\varphi_{V}(\eta,\x) \leq K_0 + K_1 \mathrm{V}(\eta)^{\beta}$$
	for some $K_0,K_1 \geq 0, \beta >0$, where $\mathrm{V}(\eta)=|\mathrm{conv}(\eta)|$ is the volume of the tetrahedron $\mathrm{conv}(\eta)$. Volume of a $3-$simplex with positions $\eta'=\{x_0,x_1,x_2,x_3\}\subset \Rt$ can be calculated using the Cayley-Menger determinant (Appendix \ref{A:CM}), but the following expression (\cite{Stein1966}) lends itself better to finding a bound: 
	$$V(\eta)=\left| \frac 1{n!} \mathrm{det}(x_1-x_0,x_2-x_0,x_3-x_0) \right|,$$
	where the determinant is of a $3 \times 3$ matrix with column vectors $x_1-x_0,x_2-x_0,x_3-x_0$. This quantity can be bounded using Hadamard's inequality (\cite{Hadamard1893}),
	$$\mathrm{det}(x_1-x_0,x_2-x_0,x_3-x_0) \leq \prod^3_{i=1} \|x_i - x_0\| .$$
	Notice that we only need the length of three edges of the tetrahedron. Thanks to this, tetrahedra of type T1 and T2 both can be bounded by
	$$\varphi_V(\eta,\x) \leq K_0 + K_1 \left(\tfrac 16 a(1+2\rho)^3\right)^{\beta}, \quad \eta\in\mathcal E_4(\x), \x\in \bar \Gamma.$$
	The bound for the intensity $z$ in the model $(\mathcal {LD}_4, \varphi_V)$ then becomes
	\begin{equation}\label{eq:volbound}z> \frac{3\sqrt 2}{4\pi}e^{14 K_0}   (4K_1 \beta e^{7/2})^{1/\beta} \frac{\frac 16 (1+2\rho)^3}{\rho^3 \sqrt{1-2\rho}}.\end{equation}
	Note that the same approach could be used for e.g. surface area of the tetrahedron, where we would simply replace $\tfrac 16 (1+2\rho)^3$ for $\tfrac 42 (1+2\rho)^2$ in \eqref{eq:volbound}.  \newline

\end{remark}


\begin{remark}[Other hardcore interaction potentials]
	Other forms of the hard-core potential can be obtained relatively easily. For example, additional constraints can be added, e.g. minimum edge length $\ell>0$. Finite horizons remain the same for all unary potentials, and therefore \textbf{(R)} holds. Stability \textbf{(S)} is satisfied, as the potential is still non-negative. The alternate upper regularity conditions \textbf{(\^{U}1)} and \textbf{(\^{U}3)} are satisfied for the same reasons. Care has to be taken for \textbf{(\^{U}2)} to hold, since the pseudo-periodic configurations $\x\in\bar \Gamma$ now must satisfy the new criterion. This can be done by choosing $a$ and $\rho$ such that $a(1-2\rho) < \ell$.
\end{remark}






\section{SIMULATION}
Basically \cite{DL11}.

\begin{itemize}
\item Specification of the simulated model.
\item Results
\end{itemize}

\section*{ACKNOWLEDGEMENT}
\small
This work was partially supported by GACR 17-00393J (investigator V. Bene\v{s}). This work was was fully supported by Panda.

\makesubmdate

\begin{thebibliography}{000}
% References should be listed in alphabetical order.


\bibitem{D17}
D.~Dereudre:
\newblock{Introduction to the theory of Gibbs point processes.}
\newblock{ArXiv {\mi 1701.08105} [math.PR] (2017).}


\bibitem{DDG12}
D.~Dereudre, R.~Drouilhet, H.O.~Georgii:
\newblock{Existence of Gibbsian point processes with geometry-dependent interactions.}
\newblock{Probability Theory and Related Fields {\mi 153 (3)} (2012), 643--670.}

\bibitem{DL11}
D.~Dereudre and F.~Lavancier:
\newblock{Practical simulation and estimation for Gibbs Delaunay-Voronoi tessellations with geometric hardcore interaction.}
\newblock{Computational Statistics and Data Analysis {\mi 55 (1)} (2011), 498--519.}





% For books, research reports and proceedings
% \bibitem{B}
% Author(s):
% \newblock{Title of book.}
% \newblock{Edition (if other than first). Publisher's name, place and year of publication (For books, research reports and proceedings).}
% 
% % Example
% \bibitem{HoTu96}
% R.~Horst and H.~Tuy:
% \newblock{Global Optimization.}
% \newblock{Springer--Verlag, Berlin 1996.}
% 
% % For journal article
% \bibitem{P}
% Author(s):
% \newblock{Title of paper.}
% \newblock{Title of the journal (abbreviated in accordance with Math. Reviews), volume, year of publication in brackets, inclusive pagination (For journal article).}
% 
% % Example
% \bibitem{No85}
% M.~Nov\'{a}k:
% \newblock{A note on the algorithms for determining the model structure.}
% \newblock{Kybernetika {\mi 21} (1985), 164--178.}
% 
% % For a paper in a bound collection
% \bibitem{PC}
% Author(s):
% \newblock{Title of paper.}
% \newblock{In: Title of collection, editor name(s) (in brackets), publisher's name, place and year of publication, inclusive pagination (For a paper in a bound collection).}
% 
% % Example
% \bibitem{Pe81}
% V.~Peterka:
% \newblock{Bayesian system identification.}
% \newblock{In: Trends and Progress in System Identification (P.~Eykhoff, ed.), Pergamon Press, Oxford 1981, pp. 239--304.}

\end{thebibliography}

\makecontacts

\end{document}
