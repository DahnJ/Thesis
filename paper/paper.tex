\documentclass{kybernetika}
%the class kybernetika includes these packages:
% graphicx, amssymb, amsmath

% used environment for theorems:

\newtheorem{theorem}{Theorem}[section]
\newtheorem{lemma}[theorem]{Lemma}
\newtheorem{proposition}[theorem]{Proposition}
\newtheorem{corollary}[theorem]{Corollary}
\newtheorem{remark}[theorem]{Remark}
\newtheorem{fact}[theorem]{Fact}
\newtheorem{example}[theorem]{Example}
\newtheorem{definition}[theorem]{Definition}
\newtheorem{observation}[theorem]{Observation}

\usepackage{xargs}  % newcommandx
\usepackage{xcolor} % todonotes colors
\usepackage{enumerate}
\usepackage[colorinlistoftodos,prependcaption,textsize=tiny]{todonotes}

\newcommandx{\problem}[2][1=]{\todo[linecolor=red,backgroundcolor=red!25,bordercolor=red,#1]{#2}}
\newcommandx{\todoo}[2][1=]{\todo[linecolor=blue,backgroundcolor=blue!25,bordercolor=blue,#1]{#2}}
\newcommandx{\note}[2][1=]{\todo[linecolor=OliveGreen,backgroundcolor=OliveGreen!25,bordercolor=OliveGreen,#1]{#2}}
\newcommandx{\unsure}[2][1=]{\todo[linecolor=violet,backgroundcolor=violet!25,bordercolor=violet,#1]{#2}}


\newcommand{\Rt}{{\mathbb R^3}}
\newcommand{\x}{{\mathbf{x}}}




\begin{document}
\pagestyle{myheadings}




\title{Existence and simulation of \\ Gibbs-Laguerre-Delaunay tetrahedrizations} 

% all authors
\author{Daniel Jahn}

% contact information of authors. Use one \contact command per person.
% Specify name(s), surname(s), postal address and e-mail.
\contact{Daniel}{Jahn}{Department of Probability and Mathematical Statistics, Charles University, Sokolovsk\'{a} 83, 186 75 Praha 8, Czech Republic}{jahn@karlin.mff.cuni.cz}

% heading contains authors and title of the paper, please shorten the title if necessary
% (it will appear only at the heading of every even page)
\markboth{D. Jahn} {Existence and simulation of GLD tetrahedrizations}

\maketitle

\begin{abstract}
We present revolutionary unprecedented results using newly developed methods. Songs will be sung about this paper. 
\end{abstract}


% here please insert two or more keywords appropriate to the topic of your paper
\keywords{Gibbs point process,
Laguerre-Delauay triangulation,
tetrahedralization,
MCMC simulation}

% here please insert one, two or more (5-digit) "Mathematics Subject Classification" codes appropriate to the topic of your paper
% you can find and choose them at the web page of the American Mathematical Society: http://www.ams.org/msc/msc2010.html
% \classification{00A01, 00M02, 00S03}
\classification{60K35, 60G55}
% 
% 60K35  Interacting random processes; statistical mechanics type models; percolation theory 
% 60G55  Point processes
% 60D05  Geometric probability, stochastic geometry, random sets [See also 52A22, 53C65]

\section{INTRODUCTION}
This is an interesting field. We did some things in this interesting field. Here they are.

\section{PRELIMINARIES}
See \cite{D17}.
\begin{itemize}
\item Laguerre geometry
\item Gibbs point process - already tie in \cite{DDG12}.
\end{itemize}

\section{EXISTENCE}
Basically \cite{DDG12}.

\begin{itemize}
\item Specification of the model
\item Proof of existence
\end{itemize}


\noindent \textbf{Smooth interaction}:  For $\eta\in\mathcal {LD}(\x)$ define the potential $\varphi_S$ as a unary potential such that
$$\varphi_S(\eta,\x) \leq K_0 + K_1 \chi(\eta)^{\beta}$$
for some $K_0,K_1 \geq 0, \beta >0$\newline
\textbf{Hard-core interaction}: For $\eta\in\mathcal {LD}(\x)$ define the potential $\varphi_{HC}$ as a unary potential such that
$$\sup_{\eta: d_0 \leq \chi(\eta) \leq d_1} \varphi_{HC}(\eta,\x)  < \infty \text{ and } \varphi_{HC}(\eta,\x)=\infty \text{ if } \chi(\eta)>\alpha$$ 
for some $0\leq d_0 < d_1 \leq \alpha$. \unsure{How exactly does this look? Why?}



\begin{enumerate}[\textbf{(S)}] 
	\item \textit{Stability}. The energy function $H$ is called \textit{stable} if there exists a constant $c_S \geq 0$ such that 
		$$H_{\Lambda,\x}(\zeta) \geq -c_S \cdot \mathrm{card}(\zeta \cup \partial_\Lambda \x)$$
		for all $\Lambda \in \mathcal B_0, \zeta \in \mathbf N_\Lambda, \x \in \mathbf N^\Lambda_{\text{cr}}$.
\end{enumerate}



\begin{enumerate}[\textbf{(R)}]\label{(R)}
	\item \textit{Range condition}. There exist constants $\ell_R,n_R \in \mathbb N$ and $\chi_R < \infty$ such that for all $(\eta,\x) \in \mathcal E$ there exists a finite horizon $\Delta$ satisfying: For every $x,y \in \Delta$ there exist $\ell$ open balls $B_1, \dots, B_\ell$ (with $\ell \leq \ell_R$) such that
	\begin{enumerate}[-]
		\item the set $\cup^\ell_{i=1} \bar B_i$ is connected and contains $x$ and $y$, and 
		\item for each $i$, either $\text{diam} B_i \leq \chi_R$ or $N_{B_i}(\x) \leq n_R$.
	\end{enumerate}
\end{enumerate}

\begin{enumerate}[\textbf{(U)}] 
	\item \textit{Upper regularity}. $M$ and $\Gamma$ can be chosen so that the following holds. 
		\begin{enumerate}[(U1)]
			\item \textit{Uniform confinement}: $\bar \Gamma \subset \mathbf N^\Lambda_\text{cr}$ for all $\Lambda \in \mathcal B_0$ and 
				\begin{equation}\label{eq:U1}r_\Gamma := \sup_{\Lambda\in\mathcal B_0}\sup_{\x \in \bar\Gamma} r_{\Lambda, \x} < \infty. \end{equation}
			\item \textit{Uniform summability}: 
			$$c_\Gamma := \sup_{\x \in \bar\Gamma}  \sum_{\eta \in \mathcal E(\x): \eta' \cap C \neq \emptyset} \frac{\varphi(\eta,\x)}{\#(\hat\eta)} < \infty,$$
where $\hat\eta := \{k \in \mathbb Z^3: \eta \cap C(k) \neq \emptyset\}$.
\item \textit{Strong non-rigidity}: $e^{z|C|} \Pi^z_C(\Gamma) > e^{c_\Gamma}$.
		\end{enumerate}
\end{enumerate}

For some models it is possible to replace the upper regularity assumptions by their alternative and prove the existence for all $z>0$.

\begin{enumerate}[(\textbf{\^{U}})]
	\item \textit{Alternative upper regularity}. $M$ and $\Gamma$ can be chosen so that the following holds.
	\begin{enumerate}[(\^U1)]
		\item \textit{Lower density bound}: There exist constants $c,d > 0$ such that 
			$$\mathrm{card}(\zeta) \geq c|\Lambda| - d$$
			whenever $\zeta \in \mathbf N_f\cap\mathbf  N_\Lambda$ is such that $H_{\Lambda,\x}(\zeta)<\infty$ for some $\Lambda \in \mathcal B_0$ and some $\x \in \bar\Gamma$.
		\item = (U2) \textit{Uniform summability}.
		\item \textit{Weak non-rigidity}: $\Pi^z_C(\Gamma) > 0$.
	\end{enumerate}
\end{enumerate}








\section{SIMULATION}
Basically \cite{DL11}.

\begin{itemize}
\item Specification of the simulated model.
\item Results
\end{itemize}

\section*{ACKNOWLEDGEMENT}
\small
This work was partially supported by GACR 17-00393J (investigator V. Bene\v{s}). This work was was fully supported by Panda.

\makesubmdate

\begin{thebibliography}{000}
% References should be listed in alphabetical order.


\bibitem{D17}
D.~Dereudre:
\newblock{Introduction to the theory of Gibbs point processes.}
\newblock{ArXiv {\mi 1701.08105} [math.PR] (2017).}


\bibitem{DDG12}
D.~Dereudre, R.~Drouilhet, H.O.~Georgii:
\newblock{Existence of Gibbsian point processes with geometry-dependent interactions.}
\newblock{Probability Theory and Related Fields {\mi 153 (3)} (2012), 643--670.}

\bibitem{DL11}
D.~Dereudre and F.~Lavancier:
\newblock{Practical simulation and estimation for Gibbs Delaunay-Voronoi tessellations with geometric hardcore interaction.}
\newblock{Computational Statistics and Data Analyssi {\mi 55 (1)} (2011), 498--519.}





% For books, research reports and proceedings
% \bibitem{B}
% Author(s):
% \newblock{Title of book.}
% \newblock{Edition (if other than first). Publisher's name, place and year of publication (For books, research reports and proceedings).}
% 
% % Example
% \bibitem{HoTu96}
% R.~Horst and H.~Tuy:
% \newblock{Global Optimization.}
% \newblock{Springer--Verlag, Berlin 1996.}
% 
% % For journal article
% \bibitem{P}
% Author(s):
% \newblock{Title of paper.}
% \newblock{Title of the journal (abbreviated in accordance with Math. Reviews), volume, year of publication in brackets, inclusive pagination (For journal article).}
% 
% % Example
% \bibitem{No85}
% M.~Nov\'{a}k:
% \newblock{A note on the algorithms for determining the model structure.}
% \newblock{Kybernetika {\mi 21} (1985), 164--178.}
% 
% % For a paper in a bound collection
% \bibitem{PC}
% Author(s):
% \newblock{Title of paper.}
% \newblock{In: Title of collection, editor name(s) (in brackets), publisher's name, place and year of publication, inclusive pagination (For a paper in a bound collection).}
% 
% % Example
% \bibitem{Pe81}
% V.~Peterka:
% \newblock{Bayesian system identification.}
% \newblock{In: Trends and Progress in System Identification (P.~Eykhoff, ed.), Pergamon Press, Oxford 1981, pp. 239--304.}

\end{thebibliography}

\makecontacts

\end{document}
