\chapter{Estimation}\label{ch:estimation}
Assume now that we obtain the point configuration $\gamma$ from model $(\varphi^{\theta,\alpha}_{HC},\mathcal {LD}_4)$ on the observation window $\Lambda_n = [-n,n]^3$ and wish to estimate the model parameters $(z,\theta,\alpha)$. To simplify the notation, we introduce the set $\Lambda^S_n = \Lambda_n \times S$.

The estimation procedure closely follows that from~\cite{DereudreLavancier2011}. That is a two-step approach, first estimating the hardcore parameter $\alpha$ and then using the estimates to obtain the estimates of $\theta$ and $z$ through maximum pseudolikelihood (MPLE). The numerical results are presented in Chapter \ref{ch:numeric}.

To underscore the dependence on the maximum circumdiameter $\alpha$ and smooth interaction parameter $\theta$, we now denote the energy $H^{\theta,\alpha}$.
\section{Estimation of the hardcore parameter}
In the first step we estimate the hardcore parameter $\alpha$. 
 Thanks to the fact that the hardcore parameter $\alpha$ satisfies

 $$ \text{if } \alpha < \alpha' \text{ then  } \forall \Lambda \in \mathcal B_0, \; H^{\theta,\alpha}_\Lambda(\gamma) < \infty \Rightarrow  H^{\theta,\alpha'}_\Lambda(\gamma)<\infty,$$ 

we can estimate it as
$$\hat\alpha = \inf\{\alpha > 0, H^{\theta,\alpha}_{\Lambda_n}(\gamma) < \infty \} = \max\{\chi(\eta), \eta\in \mathcal {LD}_{4\Lambda_n}(\gamma)\},$$
where $\mathcal {LD}_{4\Lambda_n}$ denotes the set $\mathcal E_{\Lambda_n}$ for the hypergraph structure $\mathcal E = \mathcal {LD}_4$.

The estimate $\hat\alpha$ is then used in the pseudo-likelihood function in the second estimation step.


\section{Estimation of the smooth interaction parameters}
In the second step we estimate the smooth interaction parameters through maximum pseudolikelihood.
\todoo[inline]{Improve handling of the Models - as of now they're sort of scattered}
\todoo[inline]{Equation references}
The classical version of MPLE requires heredity of the energy function. Recall that heredity means every point of the configuration is removable, see Section \ref{sec:energyfunction} for a definition and discussion. The hardcore interaction in the model $(\mathcal {LD}_4,\varphi^{\theta,\alpha}_{HC})$\todoo{again, references, clear model definitions,..} does not satisfy this condition. For that purpose, Section \ref{sec:non-hereditary} presented the extension to the non-hereditary case from \cite{DereudreLavancier2009}. 

We denote the set of removable points of $\gamma$ by $\mathcal R^\alpha(\gamma)$. Similarly, the notion of an addable point will be useful. A point $x\in\gamma$ is \textit{addable} in $\gamma$ if $\gamma \cup \{x\}$ is permissible and we denote $\tilde\Lambda^S_n$ the set of addable points in $\Lambda^S_n$. 

In the non-hereditary case, the pseudo-likelihood function then becomes (\cite{DereudreLavancier2011}):

\begin{equation}\label{PLL}
	PLL_{\Lambda_n}(\gamma,z,\theta, \alpha) = \int_{\Lambda^S_n} z \exp (-h^{\theta,\alpha}(x,\gamma)) dx + \sum_{x\in\mathcal R^\alpha(\gamma)\cap \Lambda^S_n} \big(h^{\theta,\alpha}(x,\gamma\setminus\{x\}) - \ln(z)\big),
\end{equation}
where $h^{\theta,\alpha}(x, \gamma \setminus \{x\})$ is local energy of $x$ in $\gamma$ defined in Definition \ref{def:localenergy}. 
\todoo[inline]{Do something about the fact that we're not integrating over poitns from the configuration so the local energy kinda doesn't work}
The estimates $\hat\theta$ and $\hat z$ are obtained through minimizing the $PLL_{\Lambda_n}$ function:
$$(\hat z, \hat\theta) = \text{argmin}_{z,\theta} PLL_{\Lambda_n} (\gamma, z, \hat\alpha,\theta).$$

By \unsure{Is the function convex?} differentiating the PLL function \eqref{PLL} with respect to $z$, respectively $\theta$, and setting them equal to zero, we obtain the estimate for $\hat z$,

\begin{equation}\label{z_hat}
	\hat z = \frac{\mbox{card}(\mathcal R^{\hat\alpha}(\gamma)\cap \Lambda^S_n)}{\int_{\Lambda^S_n} \exp{\left( -h^{\hat\alpha,\theta}(x,\gamma)\right)} dx},
\end{equation}
and the estimate $\hat\theta$ as the solution of

\begin{equation}\label{theta_hat} 
	z \int_{\tilde\Lambda^S_n}  h^{\hat\alpha,1}(x,\gamma)\exp{\left(-h^{\hat\alpha,\theta}(x,\gamma)\right)} dx = \sum_{x \in \mathcal R^{\hat\alpha}(\gamma)\cap \Lambda^S_n} h^{\hat\alpha,1}(x,\gamma\setminus\{x\}),
\end{equation}
where we have used the fact that the local energy depends on $\theta$ linearly, yielding

$$\frac{\partial h^{\hat\alpha,\theta}}{\partial \theta} (x,\gamma) = h^{\hat\alpha,1}(x,\gamma).$$

\subsection{Practical implementation}
We obtain the estimate of $\theta$ by substituting the expression for $\hat z$ \eqref{z_hat} into \eqref{theta_hat}.
This leads to the equation
\begin{equation}\label{eq:estunsimple} 
\frac{\int_{\tilde\Lambda^S_n}  h^{\hat\alpha,1}(x,\gamma)\exp{\left(-h^{\hat\alpha,\theta}(x,\gamma)\right)} dx} {  \int_{\Lambda^S_n} \exp{\left( -h^{\hat\alpha,\theta}(x,\gamma)\right)} dx} 
= \frac {\sum_{x \in \mathcal R^{\hat\alpha}(\gamma)\cap \Lambda^S_n} h^{\hat\alpha,1}(x,\gamma\setminus\{x\})} { \mbox{card}(\mathcal R^{\hat\alpha}(\gamma)\cap \Lambda^S_n) }. 
\end{equation}

In order to simplify the estimation of $\theta$, we can simplify this equation further. First, we denote the right-hand side of \eqref{eq:estunsimple} as $c$ as it  is constant with respect to $\theta$. Second, we note that $x \notin \tilde\Lambda^S_n  \Rightarrow \exp{\left(-h^{\hat\alpha,\theta}(x,\gamma)\right)}= 0$ which enables us to integrate over $\tilde\Lambda^S_n$ instead of the whole window $\Lambda^S_n$. Lastly we denote the local energy $h^{\hat\alpha,1}(x,\gamma) =: h(x)$, yielding the equation 
$$ \int_{\tilde\Lambda^S_n} h(x) \exp{\left(-\theta h(x)\right)} dx = c \int_{\tilde\Lambda^S_n} \exp{\left(-\theta h(x)\right)}\;dx, $$
leading into the final equation
\begin{equation}\label{hat_theta_final} 
\int_{\tilde\Lambda^S_n} \exp{\left(-\theta h(x)\right)} (h(x) - c) dx = 0 .
\end{equation}

The integral in \eqref{hat_theta_final} is estimated using Monte-Carlo integration, i.e. is approximately equal to
$$ \frac {|\tilde\Lambda^S_n|}N \sum_{i=0}^N 1_{\tilde\Lambda^S_n}(x_i) \exp{\left( - \theta h_i \right )} (h_i- c) dx $$
where $h_i = h^{\hat\alpha,1}(x_i, \gamma)$ and $x_1,\dots,x_N$ is a random sample from the \unsure{Do we need the indicator function if we're only sampling from $\tilde\Lambda^S_n$ ?}uniform distribution on $\Lambda'_n$. Note that although $|\tilde\Lambda^S_n|$ would be very difficult to obtain, the solution of the equation does not depend on it.

After $\hat\theta$ is estimated, we then obtain the estimate $\hat z$ by substituting $\theta$ by $\hat\theta$ in $\eqref{z_hat}$ and with the integral replaced by a Monte-Carlo integration approximation.


\subsection{Consistency}
For the $\mathbb R^2$ case, consistency of this estimation approach is proven in \cite{DereudreLavancier2009}. However, similarly to the irregularity of the Markov chains in Chapter \ref{ch:simulation}, the verification of the required assumptions relies on arguments that are not easy to extend to $\Rt$ and the Laguerre-Delaunay case. We have sadly not succeeded at proving the consistency of this approach for our models. This shortcoming is especially unfortunate as the numerical results given in Chapter \ref{ch:numeric} seem to point to a bias of the estimates. 
