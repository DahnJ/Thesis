\chapter{Estimation}
Assume now that we obtain the point configuration $\gamma$ from model $(\varphi^{\theta,\alpha}_{HC},\mathcal {LD}_4)$ on the observation window $\Lambda^M_n = [-n,n]^3\times S$ and wish to estimate the model parameters $(\theta,\alpha)$. We use the superscript $M$ in $\Lambda^M_n$ to differentiate from the unmarked window.

The estimation procedure closely follows that from~\cite{DereudreLavancier2010}. That is a two-step approach, first estimating the hardcore parameter $\alpha$ and then using the estimates to obtain the estimate of $\theta$ through maximum pseudolikelihood (MPLE). 

To underscore the dependence on the maximum circumdiameter $\alpha$ and smooth interaction parameter $\theta$, we now denote the energy $H^{\theta,\alpha}$.
\section{Estimation of the hardcore parameter}
In the first step we estimate the hardcore parameter $\alpha$. 
 Thanks to the fact that the hardcore parameter $\alpha$ satisfies

 $$ \text{if } \alpha < \alpha' \text{ then  } \forall \Lambda \in \mathcal B_0(\Rt), \; H^{\theta,\alpha}_\Lambda(\gamma) < \infty \Rightarrow  H^{\theta,\alpha'}_\Lambda(\gamma)<\infty,$$ 

we can estimate it as
$$\hat\alpha = \sup\{\alpha > 0, H^{\theta,\alpha}_\Lambda(\gamma) < \infty \}.$$

\unsure{Are these consistent? Why?}In practice, the parameter can be estimated as
$$\hat\alpha = \max\{\delta(\eta), \eta\in \mathcal {LD}_{4\Lambda}(\gamma)\}.$$

The estimate $\hat\alpha$ is then used in the pseudo-likelihood function in the second estimation step.


\section{Estimation of the smooth interaction parameters}
In the second step we estimate the smooth interaction parameters through maximum pseudolikelihood.
\todoo[inline]{Improve handling of the Models - as of now they're sort of scattered}
\todoo[inline]{Equation references}
The classical version of MPLE requires hereditarity of the energy function. Hereditarity means that for every permissible $\gamma$, the point pattern $\gamma\setminus\{x\}$ remains permissible for every $x\in\gamma$, see Section \ref{sec:energyfunction} for definition and discussion. The hardcore interaction in the model $(\mathcal {LD}_4,\varphi_HC)$\todoo{again, references, clear model definitions,..} does not satisfy this condition. For that purpose, Section \ref{sec-non-hereditarity} presented the extension to the non-hereditary case from \cite{DereudreLavancier2007}. 

Recall that a point $x\in\gamma$ is removable in a configuration $\gamma\in\mathbf N_{lf}$ if $\gamma\setminus\{x\}$ is permissible. We denote $\mathcal R^\alpha(\gamma)$ the set of removable points of $\gamma$. Similarly the notion of an addable point will be useful. A point $x\in\gamma$ is \textit{addable} in $\gamma$ if $\gamma \cup \{x\}$ is permissible and we denote $\mathcal A^\alpha(gamma)$.

In the non-hereditary case, the pseudo-likelihood function then becomes:

\begin{equation}\label{PLL}
	PLL_{\Lambda_n}(\gamma,z,\theta, \alpha) = \int_{\mathcal A^\alpha(\gamma)\cap (\Lambda_n\times S)} z \exp (-h^{\theta,\alpha}(x,\gamma)) dx + \sum_{x\in\mathcal R^\beta(\gamma)\cap (\Lambda_n\times S} \big(h^{\theta,\alpha}(x,\gamma\setminus\{x\}) - \ln(z)\big),
\end{equation}
where $\Lambda'_n$ is the set of all addable points in $\Lambda_n$ and $h^{\theta,\alpha}(x, \gamma \setminus \{x\})$ is \note{Connection between this and Papangelou could be useful}local energy of $x$ in $\gamma$ defined for every $x\in\mathcal R^\beta(\gamma)$ by:
$$h^{\theta,\alpha}(x, \gamma \setminus \{x\}) = E^{\theta,\alpha}_\Lambda(\gamma_\Lambda, \gamma_{\Lambda^c}) - E^{\theta,\alpha}_\Lambda(\gamma_\Lambda\setminus\{x\}, \gamma_{\Lambda^c}).$$

The estimates $\hat\theta$ and $\hat z$ are obtained through minimizing the $PLL_{\Lambda_n}$ function \ref{PLL}:
$$(\hat z, \hat\theta) = \text{argmin}_{z,\theta} PLL_{\Lambda_n} (\gamma, z, \hat\beta,\theta).$$

By differentiating the PLL function \ref{PLL} with respect to $z$, respectivelly $\theta$, and setting them equal to zero, we obtain the estimate for $\hat z$,

\begin{equation}\label{z_hat}
\hat z = \frac{\mbox{card}(\mathcal R^\beta(\gamma)\cap \Lambda_n)}{\int_{\Lambda_n} \exp{\left( -h^{\hat\beta,\theta}(x,\gamma)\right)} dx},
\end{equation}
and the estimate $\hat\theta$ as the solution of

\begin{equation}\label{theta_hat} 
z \int_{\Lambda'_n} (h^{\hat\beta,1}(x,\gamma)\exp{\left(-h^{\hat\beta,\theta}(x,\gamma)\right)}) dx = \sum_{x \in \mathcal R^{\hat\beta}(\gamma)\cap \Lambda_n} h^{\hat\beta,1}(x,\gamma\setminus\{x\}),
\end{equation}
where we have used the fact that the local energy depends on $\theta$ linearly, yielding

$$\frac{\partial h^{\hat\beta,\theta}}{\partial \theta} (x,\gamma) = h^{\hat\beta,1}(x,\gamma).$$

\subsection{Practical implementation}
We obtain the estimate of $\theta$ by substituting the expression for $\hat z$ \ref{z_hat} into \ref{theta_hat}.
This leads to the equation
$$ 
\frac{\int_{\Lambda'_n} (h^{\hat\beta,1}(x,\gamma)\exp{\left(-h^{\hat\beta,\theta}(x,\gamma)\right)}) dx} {  \int_{\Lambda_n} \exp{\left( -h^{\hat\beta,\theta}(x,\gamma)\right)} dx} 
= \frac {\sum_{x \in \mathcal R^{\hat\beta}(\gamma)\cap \Lambda_n} h^{\hat\beta,1}(x,\gamma\setminus\{x\})} { \mbox{card}(\mathcal R^\beta(\gamma)\cap \Lambda_n) }. 
$$

In order to simplify the estimation of $\theta$, we can simplify this equation further. First, we denote the righ-hand-side of the equation as $c$ as it  is constant with respect to $\theta$. Second, we note that $x \notin \Lambda_n'  \Rightarrow \exp{\left(-h^{\hat\beta,\theta}(x,\gamma)\right)}= 0$ which enables us to integrate over $\Lambda'_n$ instead of the whole $\Lambda_n$. Lastly we denote the local energy $h^{\hat\beta,1}(x,\gamma) =: h(x)$, yielding the expression
$$ \int_{\Lambda'_n} h(x) \exp{\left(-\theta h(x)\right)} dx = c \int_{\Lambda'_n} \exp{\left(-\theta h(x)\right)}, $$
leading into the final expression
\begin{equation}\label{hat_theta_final} 
\int_{\Lambda'_n} \exp{\left(-\theta h(x)\right)} (h(x) - c) dx .
\end{equation}

The integral \ref{hat_theta_final} is estimated using Monte-Carlo integration, i.e. is approximately equal to
$$ \frac 1N \sum_{i=0}^N 1_{\Lambda'_n}(x_i) \exp{\left( - \theta h_i \right )} (h_i- c) dx $$
where $h_i = h^{\hat\beta,1}(x_i, \gamma)$ and $x_1,\dots,x_N$ is a random sample from the \unsure{Do we need the indicator function if we're only sampling from $\Lambda'_n$ ?}uniform distribution on $\Lambda'_n$

After $\hat\theta$ is estimated, we then obtain the estimate $\hat z$ with $\hat\theta$ instead of $\theta$ and the integral replaced by a MC-integration approximation.


\section{Consistency}
\tbd
