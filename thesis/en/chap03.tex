\chapter{Existence of Gibbs-type models}\label{ch:3}
In this chapter, the theorem from \cite{DDG12} will be presented and then we will proceed to verify its assumptions for our models.

\section{Existence theorem}
In this section we first state the two existence theorems from \cite{DDG12}. The assumptions  \textbf{(S)}, \textbf{(R)}, \textbf{(U)} and \textbf{(\^U)} are introduced in the following sections. 

\begin{theorem}
	For every hypergraph structure $\mathcal E$, hyperedge potential $\varphi$ and activity $z>0$ satisfying \textbf{(S)}, \textbf{(R)} and \textbf{(U)} there exists at least one Gibbs measure.
\end{theorem}

\begin{theorem}
	For every hypergraph structure $\mathcal E$, hyperedge potential $\varphi$ and activity $z>0$ satisfying \textbf{(S)}, \textbf{(R)} and \textbf{(\^{U})} there exists at least one Gibbs measure.
\end{theorem}

Proofs of both theorems can be found in \cite{DDG12}, see also Remark 3.7. in the same paper about the marked case.

\subsection{Stability}\label{sec:stability}
A standard assumption without which it is impossible to define the Gibbs measure is the stability assumption. 

\begin{enumerate}[\textbf{(S)}] 
	\item \textit{Stability}. The energy function $H$ is called \textit{stable} if there exists a constant $c_S \geq 0$ such that 
		$$H_{\Lambda,\x}(\zeta) \geq -c_S \cdot \mathrm{card}(\zeta \cup \partial_\Lambda \x)$$
for all $\Lambda \in \mathcal B_0, \zeta \in \mathbf N_\Lambda, \x \in N^\Lambda_{\text{cr}}$.
\end{enumerate}


The first thing to note that when $\varphi$ is non-negative, then we can simply choose $c_S = 0$. The interesting cases therefore is when $\varphi$ can attain negative values.\newline

\subsubsection{Stability in $\mathbb R^2$}
\tbd
\subsubsection{Stability in $\mathbb R^3$}
\tbd
\note[inline]{Possibly move the discussion to an appendix}
\note[inline]{Could we at least use spread for Gibbs with limited distance between points?}

\noindent \textbf{Assumption 1}: All hyperedge potentials in the remained of this text are assumed to be non-negative.

\subsection{Range condition} \label{sec:range}
As stated previously, the fact that the hypergraph structures posses a type of locality property is crucial for the existence of Gibbs measures. The simplest such assumption is the \textit{finite range} assumption, see Definition 7 in \cite{Dereudre2017}, which roughly states that there exists $R>0$ such that the energy of $\x$ in $\Lambda$ only depends on points in $\Lambda + b(0,R)$. This is a strong assumption and one that is not fulfilled by our models. 

This is reflected in part in the range condition introduced here and later in the uniform confinement condition \eqref{eq:U1}.

\begin{enumerate}[\textbf{(R)}]\label{(R)}
	\item \textit{Range condition}. There exist constants $\ell_R,n_R \in \mathbb N$ and $\delta_R < \infty$ such that for all $(\eta,\x) \in \mathcal E$ there exists a finite horizon $\Delta$ satisfying: For every $x,y \in \Delta$ there exist $\ell$ open balls $B_1, \dots, B_\ell$ (with $\ell \leq \ell_R$) such that
	\begin{enumerate}[-]
		\item the set $\cup^\ell_{i=1} \bar B_i$ is connected and contains $x$ and $y$, and 
		\item for each $i$, either $\text{diam} B_i \leq \delta_R$ or $N_{B_i}(\x) \leq n_R$.
	\end{enumerate}
\end{enumerate}


Apart from being one of the assumptions necessary for the existence, the range condition also gives us the following crucial result we used in the definition of GPP.

\begin{proposition}\label{prop:cr-a.s.}
	Let $\Lambda \in \mathcal B_0(\Rt)$. Under the assumption \ref{(R)}, there exists a set $\hat {\mathbf N}^\Lambda_{cr} \in \mathbf N_{\Lambda^c}$ such that $\hat {\mathbf N}^\Lambda_{cr} \subset \mathbf N^\Lambda_{cr}$ and $P(\hat {\mathbf N}^\Lambda_{cr})=1$ for all $P \in \mathcal P_\Theta$ with $P(\varnothing)=0$.
\end{proposition}
\begin{proof}
	Can be found in Theorem 5.4. in \cite{DDG12}. See also Remark 3.7. in connection to the marked case.
\end{proof}
\problem[inline]{This is wrong, since we're using the wrong set $\mathbf N_\Lambda$}

The proposition shows that any $\Theta$-invariant probability measure on $(\mathbf N_{lf},\mathcal N_{lf})$ is concentrated on the set $\mathbf N^\Lambda_{cr}$ for any $\Lambda \in \mathcal B_0(\Rt)$.

\subsection{Upper regularity}\label{sec:upperregularity}


In order to present the upper regularity conditions, we introduce the notion of \textit{pseudo-periodic} configurations. 

Let $M\in\mathbb R^{3\times 3}$ be an invertible $3\times 3$ matrix with column vectors $(M_1,M_2,M_3)$. For each $k \in \mathbb Z^3$ define the cell
$$C(k) =  \{Mx \in \Rt: x-k \in \left[ -1/2, 1/2 \right)^3 \}.$$
These cells partition $\R$ into parallelepipeds. We write $C=C(0)$. Let \problem{Again, need to define these sets} $\Gamma \in \mathcal N'_C$ be non-empty. Then we define the \textit{pseudo-periodic} configurations $\bar \Gamma$ as
$$\bar \Gamma = \{ \x \in \mathbf N_{lf}: \vartheta_{Mk}(\x_{C(k)}) \in \Gamma \text{ for all } k \in \mathbb Z^3 \},$$
the set of all configurations whose restriction to $C(k)$, when shifted back to $C$, belongs to $\Gamma$. The prefix pseudo- refers to the fact that the configuration itself does not need to be identical in all $C(k)$, it merely needs to belong to the same class of configurations.

\begin{enumerate}[\textbf{(U)}] 
	\item \textit{Upper regularity}. $M$ and $\Gamma$ can be chosen so that the following holds. 
		\begin{enumerate}[(U1)]
			\item \textit{Uniform confinement}: $\bar \Gamma \subset N^\Lambda_\text{cr}$ for all $\Lambda \in \mathcal B_0$ and 
			\begin{equation}\label{eq:U1}r_\Gamma := \sup_{\Lambda\in\mathcal B_0}\sup_{\x \in \bar\Gamma} r_{\Lambda, \x} < \infty\end{equation}
			\item \textit{Uniform summability}: 
			$$c^+_\Gamma := \sup_{\x \in \bar\Gamma}  \sum_{\eta \in \mathcal E(\x): \eta \cap C \neq \emptyset} \frac{\varphi^+(\eta,\x)}{\#(\hat\eta)} < \infty,$$
where $\hat\eta := \{k \in \mathbb Z^3: \eta \cap C(k) \neq \emptyset\}$ and $\varphi^+ = \max(\varphi,0)$ is the positive part of $\varphi$.
\item \textit{Strong non-rigidity}: $e^{z|C|} \Pi^z_C(\Gamma) > e^{c_\Gamma}$, where $c_\Gamma$ is defined as in (U2) with $\varphi$ in place of $\varphi^+$.
		\end{enumerate}
\end{enumerate}

Notice that (U1) is very close to the classic finite range property mentioned at the beginning of Section \ref{sec:range}. The major difference is that here the property is only required of the pseudo-periodic configuration.


As long as $\Pi^z_C (\Gamma) >0$, (U3) will always hold for all $z$ exceeding some threshold $z_0 \geq 0$. This is because the left hand side is an increasing function of $z$, as can be seen from the equality 
$$e^{z|C|} \Pi^z_C(\Gamma) = \sum^\infty_{k=1} \frac{z^k}{k!} \int_C \cdots \int_C 1_{\Gamma} \left(\sum^k_{i=1} \delta_{X_i}\right) dx_1, \dots, dx_k,$$
which can be derived using \eqref{eq:poiscalc}. 


\todoo[inline]{Add more intuition about U3 and comment on why \^U is useful}

For some models it is possible to replace the upper regularity assumptions by their alternative and prove the existence for all $z>0$.

\begin{enumerate}[(\textbf{\^{U}})]
	\item \textit{Alternative upper regularity}. $M$ and $\Gamma$ can be chosen so that the following holds.
	\begin{enumerate}[(\^U1)]
		\item \textit{Lower density bound}: There exist constants $c,d > 0$ such that $\mathrm{card}(\zeta) \geq c|\Lambda| - d$ whenever $\zeta \in \mathbf N_f\cap\mathbf  N_\Lambda$ is such that $H_{\Lambda,\x}(\zeta)<\infty$ for some $\Lambda \in \mathcal B_0$ and some $\x \in \bar\Gamma$.
		\item = (U2) \textit{Uniform summability}.
		\item \textit{Weak non-rigidity}: $\Pi^z_C(\Gamma) > 0$.
	\end{enumerate}
\end{enumerate}






\section{Verifying the assumptions}\label{sec:verifyassumptions}

\subsection{The choice of $\Gamma$ and $M$ for Laguerre-Delaunay models}\label{sec:MGamma}
Fix some $A \subset C\times S$ and define
$$\Gamma^A = \{\zeta \in \mathbf N_C: \zeta = \{p\}, p \in A\},$$
the set of configurations consisting of exactly one point in the set $A$. The set of pseudo-periodic configurations $\bar\Gamma$ thus contains only one point in each $C(k), k~\in~\mathbb Z^3$.

Let $M$ be such that $|M_i| = a > 0$ for $i=1,2,3$ and $\angle(M_i,M_j) = \pi / 3$ for $i\neq j$.

\subsubsection{Choice of the set $A$}
In \cite{DDG12}, $A$ is chosen to be $B(0,b)$ for $b\leq \rho a$ for some sufficiently small $\rho >0$. 

We will use this form for the positions of the points as well --- the question, however, is how to choose the mark set. For Delaunay models, we choose $A=B(0,b)\times\{0\}$. It would be convenient to do this in the Laguerre case and only deal with the Delaunay tetrahedronization. However, for Laguerre-Delaunay models, this  would mean that $\Pi^z_C(\Gamma) = 0$, conflicting with both $(U3)$ and $(\hat U3)$. The choice $A=B(0,b)\times S$ could, for a small enough $a$, result in some balls being fully contained in their neighboring balls, possibly resulting in redundant points, thus changing the desired properties of $\Gamma$. It is thus necessary to choose the mark space dependent on $a$. For given $a$, $\rho$, the minimum distance between individual points, is $a-2\rho a = a(1-2\rho)$. For $\mathcal {LD}$ models we therefore choose 
$$A = B(0,b)\times \left[0, \sqrt{\frac a2(1-2\rho)}\right]$$ 
in order for balls to never overlap \unsure{This is perhaps unnecessarily conservative, we could widen it}. 




\begin{remark}[Simplification of (U2) and (U3)]\label{r:UA}
	Using the set $\Gamma^A$, we can simplify the assumptions (U2) and (U3).
\begin{enumerate}[(U1)]	
	\setcounter{enumi}{1}
	\item	We now have \todoo{Check how I am using $|\cdot |$ and $\#$} $\#(\hat\eta) = \mathrm{card}(\eta)$, since now each point of $\eta$ is necessarily in a different set $C(k)$.

\item $\Pi^z_C(\Gamma)$ can now be directly calculated.
	\begin{align*} 
		\Pi^z_C(\Gamma) &= \Pi^z_C(\{\zeta \in N_C: \zeta = \{p\}, p \in A\}) \\
		& = e^{-z|A|} z |A| e^{-z|C\setminus A|} \\
		& = e^{-z|C|} z |A|,
	\end{align*}
	and thus (U3) becomes
	$$z|A| > e^{c_{A}},$$
	where $c_A := c_{\Gamma^A}$.

	In  the case $A = B(0,\rho a)\times [0, \sqrt{\frac a2(1-2\rho)}]$ for $\mathcal {LD}$, we have
	$$|A| = \frac 43 \pi (\rho a)^3 \cdot \sqrt{\frac a2(1-2\rho)} = \frac {4\pi}{3\sqrt{2}}\cdot  \rho^3 \sqrt{1-2\rho} \cdot a^{7/2}$$

\end{enumerate}
\end{remark}


\subsection{Geometric properties of the tetrahedrizations defined by $\Gamma^A$ and $M$}
\problem[inline]{Am I talking about tetrahedrization or hypergraph? Check and unify this}
To understand the advantage of the particular choice of $M$ and $\Gamma^A$ we first turn to the two-dimensional case. For $\mathbb R^2$, the two column vectors with angle $\pi/3$ define a triangulation made of equilateral triangles. Depending on the bound for $\rho$, the points never become collinear ($\sqrt 3/4$), have a bound for the circumradius that is linear in $\rho$ ($\sqrt 3/6$) or even always generate the same triangulation ($(\sqrt 3 - 1)/4$) up to the movement of points within their respective set $A$. Thus the resulting triangulation has many desirable properties. \newline

It is not however obvious that the desirable properties carry over to $\mathbb R^3$. Before we investigate the structure of the resulting tetrahedrizations, we list the properties we are interested in obtaining.
\begin{enumerate}
	\item A description of the tetrahedra present in the tetrahedrization.
	\item The number of tetrahedra incident to the point in $C$,  
		$$n_T := \mathrm{card} \{\eta \in \mathcal E(\x): \eta' \cap C \neq \emptyset\}.$$
	\item Bounds for circumdiameters of the tetrahedra.
	\item The position of points with respect to the (reinforced) general position.  
	\item Boundedness of the weight of the characteristic points.
\end{enumerate}

\problem[inline]{There's now a double use of the word regular.}
As noted previously, using an analogous definition of $M$ in $\mathbb R^2$ forms a triangulation containing equilateral triangles. Sadly, the three-dimensional case is not as simple\footnote{And it could not be, because the analogue of the two-dimensional equilateral triangle, the regular tetrahedron, does not tessellate, as Aristotle famously wrongly claimed, see e.g. \cite{Lagarias12}}.

\subsubsection{The structure of the tetrahedrization formed by $\bar\Gamma^A$}
To better understand the structure of the resulting tetrahedrizations, we choose a particular example of a configuration from $\bar\Gamma^A$. 
$$\x_0 = \{(M_ak,0) \in \Rt\times S: k \in \mathbb Z^3\} \in \bar\Gamma,$$ 
the set of zero-weight points lying in the center of their respective cells $C(k)$, where

$$
M_a := a \begin{pmatrix}
1 & \frac 12 & \frac 12 \\
0 & \frac {\sqrt{3}}2 & \frac 1{2\sqrt{3}}  \\
0 & 0 & \sqrt{\frac 23} \\
\end{pmatrix}
$$

is a particular example of the matrix $M$. \newline

\noindent From Remark \ref{rem:LaguerreToDelaunay} we know that $\mathcal {LD}_4(\x_0) = \mathcal D_4(\x_0)$, therefore we can work with its Delaunay tetrahedrization.
\problem[inline]{Again we're not using marks without comment}

To further simplify the line of reasoning, we will look at only a subset $\x_1\subset \x_0$ of the points whose preimage under $M_a$ are the boundary points of the unit cube $[0,1]^3$. The points of $\x_1$, denoted $p_1, \dots, p_8$ then are:

\problem[inline]{It's unclear what $p_i$ are}

$$\begin{matrix}
	p_1: & (0,0,0) & \longmapsto & a(0,0,0) \\
	p_2: & (1,0,0) & \longmapsto & a(1,0,0)\\
	p_3: & (0,1,0) & \longmapsto & a(1/2,\sqrt{3}/2,0)\\
	p_4: & (1,1,0) & \longmapsto & a(3/2,\sqrt{3}/2,0)\\

	p_5: & (0,0,1) & \longmapsto & a(1/2,1/(2\sqrt{3}),\sqrt{2/3})\\
	p_6: & (1,0,1) & \longmapsto & a(3/2,1/(2\sqrt{3}),\sqrt{2/3})\\
	p_7: & (0,1,1) & \longmapsto & a(1,2/\sqrt3, \sqrt{2/3})\\
	p_8: & (1,1,1) & \longmapsto & a(2,2/\sqrt3, \sqrt{2/3})\\
\end{matrix}$$
To obtain the tetrahedrization of the parallelepiped formed by $\x_1$, we could mechanically perform the INCIRCLE test on all quintuples of points in $\x_1$ (see Remark \ref{r:construct}). Such approach is lengthy and ultimately not very illuminative. We will therefore derive its structure through a few geometric observations.

\todoo[inline]{These lemmas are almost impossible to follow without figures}
\begin{lemma}
	$\mathrm{NNG}(\x_1)$ is formed by two regular tetrahedra, $\{p_1,p_2,p_3,p_5\}$ and $\{p_4,p_6,p_7,p_8\}$, and an regular octahedron $\{p_2,\dots,p_7\}$.
\end{lemma}
\begin{proof}
	Any invertible linear transformation maps a parallelepiped onto a parallelepiped. Since $\|p_2 - p_1\| = \|p_3-p_1\| = \|p_5-p_1\|=a$ by definition of $M$, we obtain that all the edges of the parallelepiped $\{p_1,\dots,p_8\}$ have length $a$. Furthermore, each face of the parallelepiped can be split into two equilateral triangles, e.g. $\|p_3-p_2\|=a$. Consequently $\{p_1,p_2,p_3,p_5\}$ and $\{p_4,p_6,p_7,p_8\}$ are regular tetrahedra, the regularity coming from the fact that all edges have length $a$. Similarly, the sextuple $\{p_2,\dots,p_7\}$ is a regular octahedron, as all its edges have length $a$.
\end{proof}

This polyhedral configuration is well known to tessellate\footnote{ The tessellation is of great importance to many fields and thus is known under many names. In mathematics, it is most commonly called the \textit{tetrahedral-octahedral honeycomb}, or the \textit{alternated cubic honeycomb}. In structural engineering, it is known as the \textit{octet truss}, as named by Buckminster Fuller, or the \textit{isotropic vector matrix}. It is stored as \textit{fcu} in the Reticular Chemistry Structure Resource\cite{RCSR}. It is also the nearest-neighbor-graph of the face-centered cubic (fcc) crystal in crystallography\cite{Gabbrielli12}.  }. The knowledge of $\mathrm{NNG}(\x_1)$ allows us to fully categorize the tetrahedra in $\mathcal D_4(\x_0)$.

\begin{proposition}\label{prop:tetraInTess} $\mathcal D_4(\x_0)$ contains two types of tetrahedra, $T_1$ and $T_2$, with edge lengths
$$T_1: (a,a,a,a,a,a) \quad T_2:(a,a,a,a,a,\sqrt 2a)$$
\end{proposition}
\begin{proof}
From Theorem \ref{prop:nng} we know that $\mathrm{NNG}(\x_1) \subset \mathcal D_2(\x_1)$. 

We know that $\text{NNG}(\x_1)$ is composed of two regular tetrahedra and one regular octahedron $O$ with all edge lengths equal to $a$. Therefore all that remains to be done is to tetrahedronize the octahedron. By the symmetry of the regular octahedron, all the tetrahedra inside $O$ must be the same up to rotation. Each tetrahedron has five out of six edge lengths equal to $a$, therefore we only need to determine the remaining edge length. We can take e.g. any four points forming a square with side lengths $a$ to see that the remaining edge length is $\sqrt 2a$.

Since $\mathcal D_4(\x_0)$ is tessellated by copies of $\mathcal D_4(\x_1)$ translated by vectors $k\in\mathbb Z^3$, we have fully characterized the tetrahedra of $\mathcal D_4(\x_0)$. 
\end{proof}

We note that the circumradii of the tetrahedra can be calculated using the Cayley-Menger determinant (see Appendix \ref{appendix}) and are $\sqrt{6}/4 \cdot a$ for $T_1$ and $1/\sqrt{2}\cdot a$ for $T_2$.

\subsubsection{Combinatorial structure of $\mathcal D_4(\x_0)$}\label{sec:combinatorial}
Now we turn to the combinatorial structure of $\mathcal D_4(\x_0)$. In the tetrahedronized regular octahedron, each vertex is incident to $\binom{5}{3}-2=8$ tetrahedra. In the tetrahedron-octahedron tessellation, each \todoo{Reference, possibly using Schlafli symbols} vertex is incident to eight regular tetrahedra and six regular octahedra. This gives us $n_T = 8 + 6\cdot 8 = 56$. While still large, this is less than quarter of \problem{Overcounting degenerate cases}$8\cdot \binom{7}{3} = 280$ for the case of regular cube tessellation induced by the choice $M=aE$. Note that $n_T$ is much smaller for the non-degenerate case, when $O$ contains only $4$ tetrahedra and its vertices are incident either to $2$ or $4$ tetrahedra. In this case, $n_T\leq 8+6\cdot 4 = 32$.

\todoo[inline]{Make it clear that the whole thing is proved in appendix. Also, make this whole thing clearer.}

\subsubsection{Circumdiameter and characteristic point weight}
The bound on circumdiameters of the circumballs and characteristic point weights is crucial for the assumption (U1) as well as (U2) and (U3) for potentials that include them. Without such a bound, we have no uniform confinement and the hyperege potential can grow to infinity. We therefore have to investigate the shape of the tetrahedra that are possible with $\x \in \bar\Gamma$. 

\begin{proposition}\label{prop:maxPeta}
	Let $\x \in \bar\Gamma^A$ with $\rho < 1/4$. Then the weight of the characteristic point is uniformly bounded. That is there exists $C>0$ such that $p_\eta'' \leq C$ for all $\eta \in \mathcal {LD}_4(\x)$. 
\end{proposition}
\begin{proof}
Denote $\eta=\{p_1,p_2,p_3,p_4\}$, denote their positions $\eta'$ and weights $\eta''$. From Theorem \ref{prop:charpointHyperplane} and the remark below it, we know that $p'_\eta = H(p_1,p_2)\cap H(p_1,p_3) \cap H(p_1,p_4)$.

Fix the positions $\eta'$.  Changing any of the points' weights amounts to translation of the radical hyperplanes defined by that point (see note after Theorem \ref{prop:charpointHyperplane}). Given the fact that weights are bounded, $S=[0,W]$, we find that for given positions $\eta'$ there exists $W_{\eta'}>0$ such that , we have $p''_\eta \leq W_{\eta'}$ regardless of the weights.

It remains to prove that $\sup_{\eta'} W_{\eta'} < \infty$, i.e. changing the points' positions can produce only bounded $p''_\eta$. This amounts to proving that the points of $\eta$ are not allowed to come arbitrarily close to (or even attain) a non-general position. This is equivalent with boundedness of the circumsphere of $\eta'$, which is proved for $\rho<1/4$ in the Appendix \ref{appendix}.
\end{proof}

\todoo[inline]{Improve references to appendix}



\subsection{Existence theorems}\label{sec:Existence}
\todoo[inline]{Specify these things as ``models'' of the form $(\mathcal D_4, \varphi_S)$.Specify the measure $\mu$ in there, too}
In this section, we will verify the assumptions for the existence of Gibbs measures with the energy function defined on the hypergraphs $\mathcal D_4$ and $\mathcal {LD}_4$. We use the general letter $\mathcal E$ when we mean either $\mathcal D$ or $\mathcal {LD}$. Two potentials will be considered

\noindent \textbf{Smooth interaction}:  For $\eta\in\mathcal E_4(\x)$ define the potential $\varphi_S$ as an unary potential such that
$$\varphi_S(\eta,\x) \leq K_0 + K_1 \delta(\eta)^{\alpha}$$
for some $K_0,K_1 \geq 0, \alpha >0$\newline
\textbf{Hard-core interaction}: For $\eta\in\mathcal E_4(\x)$ define the potential $\varphi_{HC}$ as an unary potential such that
$$\sup_{\eta: d_0 \leq \delta(\eta) \leq d_1} \varphi_{HC}(\eta,\x)  < \infty \text{ and } \varphi_{HC}(\eta,\x)=\infty \text{ if } \delta(\eta)>\alpha.$$ 
for some $0\leq d_0 < d_1 \leq \alpha$ \unsure{How exactly does this look? Why?}

We assume by Assumption 1 that $\varphi_S,\varphi_{HC}\geq 0$.




We first present a general lemma. Recall the definition of $r_\Gamma$ from (U1). 
\begin{lemma}\label{lemma:U1}
	Let $\Gamma \subset N_{lf}$ be a class of configurations. If there exists $d_{max}>0$ such that $\mathrm{diam}\Delta < d_{max}$ for the horizon $\Delta$ of any $(\eta,\x), \eta \in \mathcal E(\x), \x \in \Gamma$, then 
	$$r_\Gamma < d_{max}.$$
\end{lemma}
\begin{proof}
	Choose $\Lambda\in \mathcal B_0$ and $\x \in \Gamma$. Let $\zeta \in N_\Lambda$ and $\eta \in \mathcal E_\Lambda(\zeta \cup \x_{\Lambda^c})$ and denote $\Delta$ the finite horizon of $(\eta,\x)$. From lemma \ref{lemma:horizEset} we obtain $\Delta\cap\Lambda \neq \emptyset$. Then $\Delta \subset \Lambda + B(0,d_max)$. If we take $\tilde\x \in \Gamma$ such that $\tilde\x = \x$ on $\partial\Lambda(\x)$ then $\varphi(\eta,\zeta \cup \x_{\Lambda^c}) = \varphi(\eta,\zeta\cup \tilde \x_{\Lambda^c})$ since $\zeta\cup\x_{\Lambda^c}$ and $\zeta\cup\tilde\x_{\Lambda^c}$ differ only on $\Delta^c$.
\end{proof}


\begin{theorem}\label{thm:E1}
	There exists at least one Gibbs measure for the model $(\mathcal D_4,\varphi_S)$ and every activity 
	$$z> \frac{3}{4\pi}e^{14 K_0}   (2K_1 \alpha e^3/3)^{1/\alpha} \frac{(\delta_1(\rho)^\alpha + \delta_2(\rho)^\alpha)^{1/\alpha}}{\rho^3}.$$
\end{theorem}
\begin{proof}
\begin{enumerate}[]
	\item \textbf{(R)} The finite-horizon $\Lambda = \bar B(\eta,\x)$ with $\ell_R = 1, n_R = 0$ and $\delta_R$ arbitrary can be used. This is because it itself contains no points of $\x$ by definition of $\mathcal D$ and acts as the open ball from the definition of the range condition.
	\item \textbf{(S)} Stability is satisfied because of $\varphi$ is non-negative.
	\item \textbf{(U)} We choose $M$ and $\Gamma$ as in section \ref{sec:MGamma}.
		\begin{enumerate}[(U1)]
			\item We know from \ref{thm:Apollonius} that the exists $d_{max}>0$ such that $\text{diam}B(\eta)\leq d_{max}$. By lemma \ref{lemma:U1} $r_\Gamma\leq d_{max}.$
			\item is trivially satisfied since $n_T < 58$ and $\varphi_S$ is bounded by \ref{thm:Apollonius}.
			\item By remark \ref{r:UA} we want to find $z$ as small as possible such that $z|A|>e^{c_A}.$ We know from \ref{sec:combinatorial} that there are $8$ $T_1$ and $48$ $T_2$ tetrahedra intersecting $C$, therefore from \ref{thm:Apollonius}
				$$c_A \leq \frac a4 (8\cdot \delta_1 + 48\cdot \delta_2 )$$
				This yield the bound
				\begin{align*}z &> \frac{4\pi\rho^3}{3} e^{2(K_0 + K_1 (a\delta_1)^\alpha) + 12(K_0 + K_1(a \delta_2)^\alpha)} / a^3  \\
					&= C_0 e^{C_1 a^\alpha} / a^3  
				\end{align*}
				where $C_0 = 3e^{14K_0}/(4\pi \rho^3),  C_1 = 2k_1(\delta_1^\alpha + 6 \delta_2^\alpha)$.

				We now choose $a$ to minimize the expression above. By optimizing over $a$ we obtain $a=(3/(C_1 \alpha))^{1/\alpha}$ which yields the bound 
				$$z> C_0(C_1 \alpha e^3/ 3 )^{1/\alpha}.$$
		\end{enumerate}
\end{enumerate}
\end{proof}



\begin{theorem}\label{thm:E2}
	There exists at least one Gibbs measure for the model $(\mathcal D_4,\varphi_{HC})$ and every activity $z>0.$
\end{theorem}
\begin{proof}
\begin{enumerate}[]
	\item \textbf{(R)} Again, $\Lambda = \bar B(\eta,\x)$ with $\ell_R = 1, n_R = 0$. Because of the hard-core condition, we can also take $\delta_R = 2\alpha$.
	\item \textbf{(S)} Stability is satisfied because of $\varphi$ is non-negative.
	\item \textbf{(\^U)} We choose $M$ and $\Gamma$ as in section \ref{sec:MGamma}.
		\begin{enumerate}[(\^U1)]
			\item For all $\eta \in \mathcal D_4(\x)$ for $\x\in\Gamma^A$ such that $H_\Lambda(\x)<\infty$ we have $\delta(\eta) < \alpha$. This imposes a minimum density of points, since e.g. no ball with diameter $\alpha$ can be empty. 
			\item We have $n_T<56$ and thus the only quantity in question is $\varphi_{HC}$. By \ref{thm:Apollonius}, we have $\delta(\eta)\leq a\delta^{max}_2(\rho)$, thus we only need to choose $a$ such that $\delta^{max}_2(\rho) \leq \alpha / a$.
			\item $\Pi^z_\Lambda(\Gamma)>0$ by Remark \ref{r:UA}.
		\end{enumerate}
\end{enumerate}
\end{proof}



\begin{theorem}\label{thm:E3}
	There exists at least one Gibbs measure for the model $(\mathcal {LD}_4,\varphi_S)$ and every activity 
	$$z> \frac{3\sqrt 2}{4\pi}e^{14 K_0}   (4K_1 \alpha e^{7/2}/7)^{1/\alpha} \frac{(\delta_1(\rho)^\alpha + \delta_2(\rho)^\alpha)^{1/\alpha}}{\rho^3 \sqrt{1-2\rho}}.$$
\end{theorem}
\begin{proof}
\begin{enumerate}[]
	\item \textbf{(R)} Take the horizon set $\Delta = B(p'_\eta, \sqrt{p''_\eta + W})$. $\Delta$ can be decomposed into the sphere $p_\eta$ and $\Delta \setminus p_\eta$, a 3-dimensional annulus with width $\sqrt{p''_\eta+W} -\sqrt{p''_\eta}=W/(\sqrt{p''_\eta+W} + \sqrt{p''_\eta})$. By definition of $\mathcal {LD}$ and \todoo{which remark?}remark, $p_\eta$  cannot contain any points of $\x$. \todoo{Ugly line placements, improve} Although the annulus $\Delta \setminus p_\eta$ does not have any bound on the number of points, its width is bounded by $\sqrt W \geq  W/(\sqrt{p''_\eta+W} + \sqrt{p''_\eta})$. This means that any $x,y\in \Delta$ can be connected by the spheres $B(x,\sqrt W), p_\eta, B(y,\sqrt W)$, yielding the parameters $\ell_R = 3,n_R=0,\delta_R=2\sqrt W$.
	\item \textbf{(S)} Stability is satisfied because of $\varphi$ is non-negative.
	\item \textbf{(U)} We choose $M$ and $\Gamma$ as in section \ref{sec:MGamma}.
		\begin{enumerate}[(U1)]
			\item By Theorem \ref{prop:maxPeta} there is $C>0$ such that $p''_\eta\leq C$ for all $\eta \in \mathcal {LD}_4(\x), \x \in \bar\Gamma^A$. By lemma \ref{lemma:U1} we have $r_\Gamma\leq \sqrt{C + W}$.
			\item is trivial since $n_T<56$ and $\varphi_{S}$ is bounded by \ref{sec:combinatorial}.
			\item We proceed similarly as in \ref{thm:E1} and obtain
				$$z>C_0 e^{C_1 a^\alpha} / a^{7/2}$$
				where $C_0=3 \sqrt 2 e^{14K_0} / (4\pi \rho^2 \sqrt{1-2\rho})$, $C_1 = 2K_1(\delta_1^\alpha + 6\delta_2^\alpha)$. Optimizing over $a$ we obtain $a=(7/(2C_1\alpha))^{1/\alpha}$ arriving at the bound
				$$z> C_0 (C_1 \alpha e^{7/2} / (7/2))^{1/\alpha}.$$
		\end{enumerate}
\end{enumerate}
\end{proof}



\begin{theorem}\label{thm:E4}
	There exists at least one Gibbs measure for the model $(\mathcal {LD}_4,\varphi_{HC})$ and every activity $$z>0.$$
\end{theorem}
\begin{proof}
\begin{enumerate}[]
	\item \textbf{(R)} The horizon set is $\Delta = B(p'_\eta,\sqrt{p''_\eta +W})$. Parameters can be chosen as in Theorem \ref{thm:E3}. \unsure{Is it a problem that there's no $n_R$ circle? Cause the proof suggested something like that?}
	\item \textbf{(S)} Stability is satisfied because of $\varphi$ is non-negative.
	\item \textbf{(\^U)} We choose $M$ and $\Gamma$ as in section \ref{sec:MGamma}.
		\begin{enumerate}[(U1)]
			\item Same as in Theorem \ref{thm:E2}. Although the underlying structure is different, the potential still depends on $\delta(\eta)$ and (\^U1) requires the configuration to have non-infinite energy.
			\item Same as in Theorem \ref{thm:E2}, $n_T<56$ and we choose an appropriate $a$.
			\item $\Pi^z_\Lambda(\Gamma)>0$ by Remark \ref{r:UA}
		\end{enumerate}
\end{enumerate}
\end{proof}



\begin{remark}[Extending to other potentials]
	.
\note[inline]{Directly obtainable results: 1) Smooth interaction for other unary potentials such as $k$-facet volume (use Hadamard inequality to bound them). 2) Adding additional constraints to hardcore models as long as we can find $a$ to satisfy the constraints.}
\end{remark}


\begin{remark}[Concrete values of $z$]
	\tbd
\end{remark}
