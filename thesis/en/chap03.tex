\chapter{Existence of Gibbs-type models}\label{ch:3}
In this chapter, the theorem from \cite{DL07} will be presented and then we will proceed to check its assumptions for our models.

\section{Existence theorem}
In this section we first state the two existence theorems from \cite{DL07} and then proceed to introduce its assumptions.

\begin{theorem}
	For every hypergraph structure $\mathcal E$, hyperedge potential $\varphi$ and activity $z>0$ satisfying \textbf{(S)}, \textbf{(R)} and \textbf{(U)} there exists at least one Gibbs measure.
\end{theorem}

\begin{theorem}
	For every hypergraph structure $\mathcal E$, hyperedge potential $\varphi$ and activity $z>0$ satisfying \textbf{(S)}, \textbf{(R)} and \textbf{(\^{U})} there exists at least one Gibbs measure.
\end{theorem}

Proofs of both theorems can be found in \cite{DL07}.

\subsection{Stability}
A standard assumption without which it is impossible to define the Gibbs measure is the stability assumption.

\begin{enumerate}[\textbf{(S)}] 
	\item \textit{Stability}. The hyperedge potential $\varphi$ is called \textit{stable} if there exists a constant $c_S \geq 0$ such that 
$$H_{\Lambda,\x}(\zeta) \geq -c_S \#(\zeta \cup \partial_\Lambda \x)$$
for all $\Lambda \in \mathcal B_0, \zeta \in N_\Lambda, \x \in N^\Lambda_{\text{cr}}$.
\end{enumerate}


The first thing to note that when $\varphi$ is non-negative, then we can simply choose $c_S = 0$. The interesting cases therefore is when $\varphi$ can attain negative values.\newline

\subsubsection{Stability in $\mathbb R^2$}
\tbd
\subsubsection{Stability in $\mathbb R^3$}
\tbd
\note[inline]{Could we at least use spread for gibbs with limited distance between points?}

\subsection{Range condition} \label{sec:range}
As stated previously, the fact that the hyergraph structures posses a type of locality property is crucial for the existence of Gibbs measures. The simplest such assumption is the \textit{finite range} assumption, see e.g. [intro def7], which roughly states that there exists $R>0$ such that the energy of $\x$ in $\Delta$ only depends on points in $\Delta + b(0,R)$. This is a strong assumption and one that is not fulfilled by our models. 

This is reflected in part in the range condition introduced here and later in the uniform confinement condition \ref{eq:U1}.

\begin{enumerate}[\textbf{(R)}]
	\item \textit{Range condition}. There exist constants $\ell_R,n_R \in \mathbb N$ and $\delta_R < \infty$ such that for all $(\eta,\x) \in \mathcal E$ there exists a finite horizon $\Delta$ satisfying: For every $x,y \in \Delta$ there exist $\ell$ open balls $B_1, \dots, B_\ell$ (with $\ell \leq \ell_R$) such that
	\begin{enumerate}[-]
		\item the set $\cup^\ell_{i=1} \bar B_i$ is connected and contains $x$ and $y$, and 
		\item for each $i$, either $\text{diam} B_i \leq \delta_R$ or $|\x_{B_i}| \leq n_R$.
	\end{enumerate}
\end{enumerate}

\subsection{Upper regularity}


In order to present the upper regularity conditions, we introduce the notion of \textit{pseudo-periodic} configurations. 

Let $M\in\mathbb R^{3\times 3}$ be an invertible $3\times 3$ matrix with column vectors $(M_1,M_2,M_3)$. For each $k \in \mathbb Z^3$ define the cell
$$C(k) =  \{Mx \in \R: x-k \in \left[ -1/2, 1/2 \right)^3 \}.$$
These cells partition $\R$ into parallelotopes. We write $C=C(0)$. Let $\Gamma \in \mathcal N'_C$ be non-empty. Then we define the \textit{pseudo-periodic} configurations $\bar \Gamma$ as
$$\bar \Gamma = \{ \x \in N: \vartheta_{Mk}(\x_{C(k)}) \in \Gamma \text{ for all } k \in \mathbb Z^3 \},$$
the set of all configurations whose restriction to $C(k)$, when shifted back to $C$, belongs to $\Gamma$. The prefix pseudo- refers to the fact that the configuration itself does not need to be identical in all $C(k)$, it merely needs to belong to the same class of configurations.

\begin{enumerate}[\textbf{(U)}] 
	\item \textit{Upper regularity}. $M$ and $\Gamma$ can be chosen so that the following holds. 
		\begin{enumerate}[(U1)]
			\item \textit{Uniform confinement}: $\bar \Gamma \subset N^\Lambda_\text{cr}$ for all $\Lambda \in \mathcal B_0$ and 
			\begin{equation}\label{eq:U1}r_\Gamma := \sup_{\Lambda\in\mathcal B_0}\sup_{\x \in \bar\Gamma} r_{\Lambda, \x} < \infty\end{equation}
			\item \textit{Uniform summability}: 
			$$c^+_\Gamma := \sup_{\x \in \bar\Gamma}  \sum_{\eta \in \mathcal E(\x): \eta \cap C \neq \emptyset} \frac{\varphi^+(\eta,\x)}{\#(\hat\eta)} < \infty,$$
where $\hat\eta := \{k \in \mathbb Z^3: \eta \cap C(k) \neq \emptyset\}$ and $\varphi^+ = \max(\varphi,0)$ is the positive part of $\varphi$.
\item \textit{Strong non-rigidity}: $e^{z|C|} \Pi^z_C(\Gamma) > e^{c_\Gamma}$, where $c_\Gamma$ is defined as in (U2) with $\varphi$ in place of $\varphi^+$.
		\end{enumerate}
\end{enumerate}

Notice that (U1) is very close to the classic finite range property mentioned at the beginning of section \ref{sec:range}. The major difference is that here the property is only required of the pseudo-periodic configuration.

\todoo[inline]{Check how I treat PP and random sets. Maybe use the duality between them?}

As long as $\Pi^z_C (\Gamma) >0$, (U3) will always hold for all $z$ exceeding some threshold $z_0 \geq 0$. This is because the left hand side is an increasing function of $z$, as can be seen from the equality 
$$e^{z|C|} \Pi^z_C(\Gamma) = \sum^\infty_{k=1} \frac{z^k}{k!} \int_C \cdots \int_C 1_{\Gamma} \left(\sum^k_{i=1} \delta_{X_i}\right) dx_1, \dots, dx_k,$$
which can be derived using proposition \ref{bincalc}. Indeed, let $\Phi \sim \Gamma^z_C$ be a Poisson point process with intensity $z$, restricted to $C$, we then have
\begin{align*}
	\Pi^z_C(\Gamma) &= P(\Phi \in \Gamma) = \sum^\infty_{k=0} P(\Phi \in \Gamma | \Phi(C) = k) P(\Phi(C)=k) \\
	& = \sum^\infty_{k=0} \frac{(z|C|)^k}{k!} e^{-z|C|} P(\Phi^{(k)}\in \Gamma)\\ 
	& = \sum^\infty_{k=0} \frac{z^k}{k!} e^{-z|C|} \int_C \cdots \int_C 1_{\Gamma} \left(\sum^k_{i=1} \delta_{X_i}\right) dx_1, \dots, dx_k\\
\end{align*}
where $\Phi^{(k)} = \sum^k_{i=1}\delta_{X_i}$ denotes the Binomial point process of $k$ points in $C$ and $\Phi^{(0)} = \delta_\emptyset$.


\todoo[inline]{Remark about U3 monotonicity, possibly some other remarks about the assumptions}

\todoo[inline]{Get more intuition about U3 and comment on why \^U is useful}

For some models it is possible to replace the upper regularity assumptions by their alternative and prove the existence for all $z>0$.

\begin{enumerate}[(\textbf{\^{U}})]
	\item \textit{Alternative upper regularity}. $M$ and $\Gamma$ can be chosen so that the following holds.
	\begin{enumerate}[(\^U1)]
		\item \textit{Lower density bound}: There exist constants $c,d > 0$ such that $\#(\zeta) \geq c|\Lambda| - d$ whenever $\zeta \in N_f\cap N_\Lambda$ is such that $H_{\Lambda,\x}(\zeta)<\infty$ for some $\Lambda \in \mathcal B_0$ and some $\x \in \bar\Gamma$.
		\item = (U2) \textit{Uniform summability}.
		\item \textit{Weak non-rigidity}: $\Pi^z_C(\Gamma) > 0$.
	\end{enumerate}
\end{enumerate}






\section{Checking the assumptions}

\subsection{The choice of $\Gamma$ and $M$ for Laguerre-Delaunay models}
Fix some $A \subset C\times S$ and define
$$\Gamma^A = \{\zeta \in N_C: \zeta = \{p\}, p \in A\},$$
the set of configurations consisting of exactly one point in the set $A$. The set of pseudo-periodic configurations $\tilde\Gamma$ thus contains only one point in each $C(k), k\in\mathbb Z^3$.

Let $M$ be such that $|M_i| = a > 0$ for $i=1,2,3$ and $\angle(M_i,M_j) = \pi / 3$ for $i\neq j$.

\subsubsection{Choice of the set $A$}
In \cite{DDG12}, $A$ is chosen to be $B(0,b)$ for $b\leq \rho_0 a$ for some \unsure{The vagueness about $\rho_0$ is not satisfactory, though it's the way DDG did it. If possible, change this} sufficiently small $\rho_0 >0$. 

We will use this form for the positions of the points as well --- the question, however, is how to choose the mark set. It would be convenient to choose $A=B(0,b)\times\{w\}$ for some $w\in S$ and then only deal with a Delaunay triangulation, but this \problem{Only true if $\mu$ is non-atomic. But we could use an atomic $\mu$ for working with Delaunay.} would mean that $\Pi^z_C(\Gamma) = 0$, conflicting with both $(U3)$ and $(\hat U3)$. The choice $A=B(0,b)\times S$ could, for a small enough $a$, result in some spheres being fully contained in their neighboring spheres, possibly resulting in redundant points, thus changing the desired properties of $\Gamma$. It is thus necessary to choose the mark space dependent on $a$. For given $a$, $\rho_0$, the minimum distance between individual points is $a-2\rho_0 a = a(1-2\rho_0)$. We therefore choose 
$$A = B(0,b)\times \left[0, \sqrt{\frac a2(1-2\rho_0)}\right]$$ 
in order for spheres to never overlap \unsure{This is perhaps unnecessarily conservative, we could widen it}. 




\begin{remark}[Simplification of (U2) and (U3)]\label{r:UA}
	Using the set $\Gamma^A$, we can simplify the assumptions (U2) and (U3).
\begin{enumerate}[(U2)]	
\item	We now have \todoo{Check how I am using $|\cdot |$ and $\#$} $\#(\hat\eta) = |\eta|$, since now each point of $\eta$ is necessarily in a different set $C(k)$.

\item $\Pi^z_C(\Gamma)$ can now be directly calculated.
	\begin{align*} 
		\Pi^z_C(\Gamma) &= \Pi^z_C(\{\zeta \in N_C: \zeta = \{p\}, p \in A\}) \\
		& = e^{-z|A|} z |A| e^{-z|C\setminus A|} \\
		& = e^{-z|C|} z |A|,
	\end{align*}
	and thus (U3) becomes
	$$z|A| > e^{c_{A}},$$
	where $c_A := c_{\Gamma^A}$.

	In the case $A = B(0,\rho_0 a)\times [0, \sqrt{\frac a2(1-2\rho_0)}]$, we have
	$$|A| = \frac 43 \pi (\rho_0 a)^3 \cdot \sqrt{\frac a2(1-2\rho_0)} = \frac {4\pi}{3\sqrt{2}}\cdot  \rho_0^3 \sqrt{1-2\rho_0} \cdot a^{7/2}$$

\end{enumerate}


\subsection{Geometrical structure of the tetrihedrizations defined by $\Gamma^A$ and $M$}
\problem[inline]{Am I talking about tetrihedrization or hypergraph? Check and unify this}
The advantage of the choice of $M$ and $A$ is that the tetrihedrizations formed by the configurations in $\tilde\Gamma^A$ can be described relatively simply. In particular, a sufficiently small $\rho_0$ ensures that the structure of the tetrihedrization does not change a lot and avoids degenerate cases of points not in general position. 

For exmaple, in the $\mathbb R^2$ case, the two column vectors with angle $\pi/3$ define a triangulation made of equilateral triangles. Depending on the bound for $\rho_0$, the points never become collinear ($\sqrt 3/6$) or even always generate the same triangultaion ($(\sqrt 3 - 1)/4$) up to the movement of points within their respective set $A$. \newline

Before we investigate the structure of the resulting tetrihedrizations, we list the properties we are interested in obtaining.
\begin{enumerate}
	\item The number of tetrahedra incident to the point in $C$,  
		$$n_T := \# \{\eta \in \mathcal E(\x): \eta \cap C \neq \emptyset\}.$$
	\item The behaviour of the hyperedge potentials\todoo{Make precise later}
	\item The position of points with respect to the (reinforced) general position.  
	\item Boundedness of the weight of the characteristic points, i.e.
\end{enumerate}

\problem[inline]{There's now a double use of the word regular. Do something about this. Perhaps call them Platonic}
As noted previously, the using an analogous definition in $\mathbb R^2$ forms a triangulation containing equilateral triangles. Sadly, the three dimensional case is not as simple\footnote{And it couldn't be, because the analogue of the two-dimensional equilateral triangle, the regular tetrahedron, does not tessellate, as Aristotle famously wrongly claimed \cite{Lagarias12}}.
To better understand the structure of the resulting tetrahedrizations, we choose a particular example of a configuration from $\tilde\Gamma^a$. 
$$\x_0 = \{(M_ak,0) \in \Rt\times S: k \in \mathbb Z^3\} \in \tilde\Gamma,$$ 
the set of zero-weight points lying in the center of their respective cells $C(k)$, where

$$
M_a := \begin{pmatrix}
1 & \frac 12 & \frac 12 \\
0 & \frac {\sqrt{3}}2 & \frac 1{2\sqrt{3}}  \\
0 & 0 & \sqrt{\frac 23} \\
\end{pmatrix}.
$$

is a particular example of the matrix $M$.

From remark \ref{rem:LaguerreToDelaunay} we know that $\mathcal {LD}_4(\x_0) = \mathcal D_4(\x_0)$, therefore we can work with its Delaunay tetrihedrization.

To further simplify the line of reasoning, we will look at only a subset $\p_0$ of $\x_0$ of the points whose preimage under $M_a$ are the boundary points of the unit cube $[0,1]^3$. The points of $\p_0$, denoted $p_1, \dots, p_8$ then are:

\problem[inline]{It's unclear what $p_i$ are}

$$\begin{matrix}
	p_1: & (0,0,0) & \rightarrow & a(0,0,0) \\
	p_2: & (1,0,0) & \rightarrow & a(1,0,0)\\
	p_3: & (0,1,0) & \rightarrow & a(1/2,\sqrt{3}/2,0)\\
	p_4: & (1,1,0) & \rightarrow & a(3/2,\sqrt{3}/2,0)\\

	p_5: & (0,0,1) & \rightarrow & a(1/2,1/(2\sqrt{3}),\sqrt{2/3})\\
	p_6: & (1,0,1) & \rightarrow & a(3/2,1/(2\sqrt{3}),\sqrt{2/3})\\
	p_7: & (0,1,1) & \rightarrow & a(1,2/\sqrt3, \sqrt{2/3})\\
	p_8: & (1,1,1) & \rightarrow & a(2,2/\sqrt3, \sqrt{2/3})\\
\end{matrix}$$

To obtain the tetrihedrization of the parallelohedron formed by $\p_0$, we could mechanically perform the INCIRCLE test on all quintuples of points in $\p_0$ (see remark \ref{r:construct}). We can also use our knowledge of the Delaunay tetrahedrization and geometry to deduce the structure of the tetrihedrization.

\todoo[inline]{Format this section so that it's not just a wall of text}
\todoo[inline]{Comment on why the distances are what they are}
We know (proposition \ref{prop:nng}) that $\text{NNG}(\p_0) \subset \mathcal D_2(\p_0)$. $\text{NNG}(\p_0)$ is formed by two regular tetrahedra, $\{p_1,p_2,p_3,p_5\}$ and $\{p_4,p_6,p_7,p_8\}$, and an regular octahedron $\{p_2,\dots,p_7\}$. Their regularity comes from the fact that all edges are of length $1$. This polyhedral configuration is well known to tessellate\footnote{ The tessellation is of great importance to many fields and thus is known under many names. In mathematics, it is most commonly called the \textit{tetrahedral-octahedral honeycomb}, or the \textit{alternated cubic honeycomb}. In structural engineering, it is known as the \textit{octet truss}, as named by Buckminster Fuller, or the \textit{isotropic vector matrix}. It is stored as \textit{fcu} in the Reticular Chemistry Structure Resource\cite{RCSR}. It is also the nearest-neighbor-graph of the face-centered cubic (fcc) crystal in crystallography\cite{Gabbrielli12}.  }.

To obtain the Delaunay tetrohedronization, we need to tetrahedronize the regular octahedron $O=\{p_2,\dots, p_7\}$. A regular octahedron is a Platonic solid and as such all of its vertices are cocircular [ref]. Furthermore it contains three quadruples of points that are coplanar [ref]. This configuration produces $\binom{6}{4}-3 = 12$ tetrahedra, many of which intersect each other, a degeneracy that is nevertheless allowed in our definition of $\mathcal D_4$. In most (in fact almost surely w.r.t. $\Pi^z$)\todoo{Try to show that we really only need almost all $\omega \in \tilde\Gamma$} configurations in $\tilde\Gamma^A$ this won't be the case as the octahedron won't be regular. However, since we're interested in the supremum, we must consider this extreme case.

\subsubsection{Combinatorial structure of $\mathcal D(\x_0)$}
Now we turn to the combinatorial structure of $\mathcal D(\x)$. In the tetrahedronized regular octahedron, each vertex is incident to $\binom{5}{3}-2=8$ tetrahedra. In the tetrahedron-octahedron tessellation, each \todoo{Reference, possibly using Schlafli symbols} vertex is incident to eight regular tetrahedra and six regular octahedra. This gives us $n_T = 8 + 6\cdot 8 = 56$. While still large, this is less than quarter of \problem{Overcounting degenerate cases}$8\cdot \binom{7}{3} = 280$ for the case of regular cube tessellation induced by the choice $M=aE$. Note that $n_T$ is much smaller for the non-degenerate case, when $O$ contains only $4$ tetrahedra and its vertices are incident either to $2$ or $4$ tetrahedra. In this case, $n_T\leq 8+6\cdot 4 = 32$.


\subsubsection{Circumdiameter and characteristic point weight}
The bound on circumdiameters of the circumballs and characteristic point weights is crucial for the assumption (U1) as well as (U2) and (U3) for potentials that include them. Without such a bound, we have no uniform confinement and the hyperege potential can grow to infinity. We therefore have to investigate the shape of the tetrahedra that are possible with $\x \in \tilde\Gamma$. 

\begin{proposition} $\mathcal D_4(\x_0)$ contains two types of tetrahedra, $T_1$ and $T_2$, with edge lengths
$$T_1: (a,a,a,a,a,a) \quad T_2:(a,a,a,a,a,\sqrt 2a)$$
\end{proposition}
\begin{proof}
We know that $\text{NNG}(\p_0)$ is composed of two regular tetrahedra and one regular octahedron $O$ with all edge lengths equal to $a$. By the symmetry of the regular octahedron, all the tetrahedra inside $O$ must be the same up to rotation. Each tetrahedron has five out of six edge lengths equal to $a$, therefore we only need to determine the remaining edge length. We can take e.g. any four points forming a square with side lengths $a$ to see that the remaining edge length is $\sqrt 2a$.
Since $\mathcal D_4(\x_0)$ is tessellated by copies of $\mathcal D_4(\p_0)$ translated by vectors $k\in\mathbb Z^3$, we have fully characterized the tetrahedra of $\mathcal D_4(\x_0)$. 
\end{proof}

With this knowledge we are ready to investigate the 
\begin{proposition}
Let $\x \in \tilde\Gamma^A$. Then there exists $C>0$ such that $p_\eta'' \leq C$ for all $\eta \in \mathcal {LD}_4(\x)$. 
\end{proposition}
\begin{proof}
Denote $\eta=\{p_1,p_2,p_3,p_4\}$, denote their positions $\eta'$ and weights $\eta''$. From proposition \ref{prop:charpointHyperplane} and the remark below it, we know that $p'\eta = H(p_1,p_2)\cap H(p_1,p_3) \cap H(p_1,p_4)$.

Fix the positions $\eta'$.  Changing any of the points' weights ammounts to translation of the radical hyperplanes defined by that point (see note after proposition \ref{prop:charpointHyperplane2}). Given the fact that weights are bounded, $S=[0,W]$, we find that there exists $B_{\eta'}>0$ such that for given positions $\eta'$, we have $p''_\eta \leq B_{\eta'}$ regardless of the weights.
It remains to prove that $\sup_{\eta'} B_{\eta'} < \infty$, i.e. changing the points' positions can produce only bounded $p''_\eta$. This ammounts to proving that the points of $\eta$ are not allowed to come arbitrarily close to (or even attain) a non-general position. This is equivalent with boundedness of the circumsphere of $\eta'$, which is proved for $\rho<$\tbd in the appendix [ref].
\end{proof}



\end{remark}

