\chapter{Existence of Gibbs-type models}
In this chapter, the theorem from \cite{DL07} will be presented and then we will proceed to check its assumptions for our models.

\section{Existence theorem}
In this section we first state the two existence theorems from \cite{DL07} and then proceed to introduce its assumptions.

\begin{theorem}
	For every hypergraph structure $\mathcal E$, hyperedge potential $\varphi$ and activity $z>0$ satisfying \textbf{(S)}, \textbf{(R)} and \textbf{(U)} there exists at least one Gibbs measure.
\end{theorem}

\begin{theorem}
	For every hypergraph structure $\mathcal E$, hyperedge potential $\varphi$ and activity $z>0$ satisfying \textbf{(S)}, \textbf{(R)} and \textbf{(\^{U})} there exists at least one Gibbs measure.
\end{theorem}

Proofs of both theorems can be found in \cite{DL07}.

\subsection{Stability}
A standard assumption without which it is impossible to define the Gibbs measure is the stability assumption.

\begin{enumerate}[\textbf{(S)}] 
	\item \textit{Stability}. The hyperedge potential $\varphi$ is called \textit{stable} if there exists a constant $c_S \geq 0$ such that 
$$H_{\Lambda,\x}(\zeta) \geq -c_S \#(\zeta \cup \partial_\Lambda \x)$$
for all $\Lambda \in \mathcal B_0, \zeta \in N_\Lambda, \x \in N^\Lambda_{\text{cr}}$.
\end{enumerate}


The first thing to note that when $\varphi$ is non-negative, then we can simply choose $c_S = 0$. The interesting cases therefore is when $\varphi$ can attain negative values.\newline

\subsubsection{Stability in $\mathbb R^2$}
\tbd
\subsubsection{Stability in $\mathbb R^3$}
\tbd

\subsection{Range condition} \label{sec:range}
As stated previously, the fact that the hyergraph structures posses a type of locality property is crucial for the existence of Gibbs measures. The simplest such assumption is the \textit{finite range} assumption, see e.g. [intro def7], which roughly states that there exists $R>0$ such that the energy of $\x$ in $\Delta$ only depends on points in $\Delta + b(0,R)$. This is a strong assumption and one that is not fulfilled by our models. 

This is reflected in part in the range condition introduced here and later in the uniform confinement condition [ref].

\begin{enumerate}[\textbf{(R)}]
	\item \textit{Range condition}. There exist constants $\ell_R,n_R \in \mathbb N$ and $\delta_R < \infty$ such that for all $(\eta,\x) \in \mathcal E$ there exists a finite horizon $\Delta$ satisfying: For every $x,y \in \Delta$ there exist $\ell$ open balls $B_1, \dots, B_\ell$ (with $\ell \leq \ell_R$) such that
	\begin{enumerate}[-]
		\item the set $\cup^\ell_{i=1} \bar B_i$ is connected and contains $x$ and $y$, and 
		\item for each $i$, either $\text{diam} B_i \leq \delta_R$ or $|\x_{B_i}| \leq n_R$.
	\end{enumerate}
\end{enumerate}

\subsection{Upper regularity}


In order to present the upper regularity conditions, we introduce the notion of \textit{pseudo-periodic} configurations. 

Let $M\in\mathbb R^{3\times 3}$ be an invertible $3\times 3$ matrix with column vectors $(M_1,M_2,M_3)$. For each $k \in \mathbb Z^3$ define the cell
$$C(k) =  \{Mx \in \R: x-k \in \left[ -1/2, 1/2 \right)^3 \}.$$
These cells partition $\R$ into parallelotopes. We write $C=C(0)$. Let $\Gamma \in \mathcal N'_C$ be non-empty. Then we define the \textit{pseudo-periodic} configurations $\bar \Gamma$ as
$$\bar \Gamma = \{ \x \in N: \vartheta_{Mk}(\x_{C(k)}) \in \Gamma \text{ for all } k \in \mathbb Z^3 \},$$
the set of all configurations whose restriction to $C(k)$, when shifted back to $C$, belongs to $\Gamma$. The prefix pseudo- refers to the fact that the configuration itself does not need to be identical in all $C(k)$, it merely needs to belong to the same class of configurations.

\begin{enumerate}[\textbf{(U)}] 
	\item \textit{Upper regularity}. $M$ and $\Gamma$ can be chosen so that the following holds. 
		\begin{enumerate}[(U1)]
			\item \textit{Uniform confinement}: $\bar \Gamma \subset N^\Lambda_\text{cr}$ for all $\Lambda \in \mathcal B_0$ and 
			$$r_\Gamma := \sup_{\Lambda\in\mathcal B_0}\sup_{\x \in \bar\Gamma} r_{\Lambda, \x} < \infty$$
			\item \textit{Uniform summability}: 
			$$c^+_\Gamma := \sup_{\x \in \bar\Gamma}  \sum_{\eta \in \mathcal E(\x): \eta \cap C \neq \emptyset} \frac{\varphi^+(\eta,\x)}{\#(\hat\eta)} < \infty,$$
where $\hat\eta := \{k \in \mathbb Z^3: \eta \cap C(k) \neq \emptyset\}$ and $\varphi^+ = \max(\varphi,0)$ is the positive part of $\varphi$.
\item \textit{Strong non-rigidity}: $e^{z|C|} \Pi^z_C(\Gamma) > e^{c_\Gamma}$, where $c_\Gamma$ is defined as in (U2) with $\varphi$ in place of $\varphi^+$.
		\end{enumerate}
\end{enumerate}

Notice that (U1) is very close to the classic finite range property mentioned at the beginning of section \ref{sec:range}. The major difference is that here the property is only required of the pseudo-periodic configuration.

\todoo[inline]{Check how I treat PP and random sets. Maybe use the duality between them?}

As long as $\Pi^z_C (\Gamma) >0$, (U3) will always hold for all $z$ exceeding some threshold $z_0 \geq 0$. This is because the left hand side is an increasing function of $z$, as can be seen from the equality 
$$e^{z|C|} \Pi^z_C(\Gamma) = \sum^\infty_{k=1} \frac{z^k}{k!} \int_C \cdots \int_C 1_{\Gamma} \left(\sum^k_{i=1} \delta_{X_i}\right) dx_1, \dots, dx_k,$$
which can be derived using proposition \ref{bincalc}. Indeed, let $\Phi \sim \Gamma^z_C$ be a Poisson point process with intensity $z$, restricted to $C$, we then have
\begin{align*}
	\Pi^z_C(\Gamma) &= P(\Phi \in \Gamma) = \sum^\infty_{k=0} P(\Phi \in \Gamma | \Phi(C) = k) P(\Phi(C)=k) \\
	& = \sum^\infty_{k=0} \frac{(z|C|)^k}{k!} e^{-z|C|} P(\Phi^{(k)}\in \Gamma)\\ 
	& = \sum^\infty_{k=0} \frac{z^k}{k!} e^{-z|C|} \int_C \cdots \int_C 1_{\Gamma} (\sum^k_{i=1} \delta_{X_i}) dx_1, \dots, dx_k\\
\end{align*}
where $\Phi^{(k)} = \sum^k_{i=1}\delta_{X_i}$ denotes the Binomial point process of $k$ points in $C$ and $\Phi^{(0)} = \delta_\emptyset$.


\todoo[inline]{Remark about U3 monotonicity, possibly some other remarks about the assumptions}

\todoo[inline]{Get more intuition about U3 and comment on why \^U is useful}

For some models it is possible to replace the upper regularity assumptions by their alternative and prove the existence for all $z>0$.

\begin{enumerate}[(\textbf{\^{U}})]
	\item \textit{Alternative upper regularity}. $M$ and $\Gamma$ can be chosen so that the following holds.
	\begin{enumerate}[(\^U1)]
		\item \textit{Lower density bound}: There exist constants $c,d > 0$ such that $\#(\zeta) \geq c|\Lambda| - d$ whenever $\zeta \in N_f\cap N_\Lambda$ is such that $H_{\Lambda,\x}(\zeta)<\infty$ for some $\Lambda \in \mathcal B_0$ and some $\x \in \bar\Gamma$.
		\item = (U2) \textit{Uniform summability}.
		\item \textit{Weak non-rigidity}: $\Pi^z_C(\Gamma) > 0$.
	\end{enumerate}
\end{enumerate}






\section{Checking the assumptions}

\subsection{The choice of $\Gamma$ and $M$ for Laguerre-Delaunay models}
Fix some $A \subset C\times S$ and define
$$\Gamma^A = \{\zeta \in N_C: \zeta = \{p\}, p \in A\},$$
the set of configurations consisting of exactly one point in the set $A$. The set of pseudo-periodic configurations $\tilde\Gamma$ thus contains only one point in each $C(k), k\in\mathbb Z^3$.

Let $M$ be such that $|M_i| = a > 0$ for $i=1,2,3$ and $\angle(M_i,M_j) = \pi / 3$ for $i\neq j$.

In \cite{DDG12}, $A$ is chosen to be $B(0,b)$ for $b\leq \rho_0 a$ for some \unsure{The vagueness about $\rho_0$ is not satisfactory, though it's the way DDG did it. If possible, change this} sufficiently small $\rho_0 >0$. 

We will use this form for the positions of the points as well - the question, however, is how to choose the mark set. It would be convenient to choose $A=B(0,b)\times\{w\}$ for some $w\in S$ and then only deal with a Delaunay triangulation, but this \problem{Only true if $\mu$ is non-atomic. But we could use an atomic $\mu$ for working with Delaunay.} would mean that $\Pi^z_C(\Gamma) = 0$, conflicting with both $(U3)$ and $(\hat U3)$. The choice $A=B(0,b)\times S$ could, for a small enough $a$, result in some spheres being fully contained in their neighboring spheres, possibly resulting in redundant points, thus changing the desired properties of $\Gamma$. It is thus necessary to choose the mark space dependent on $a$. For given $a,\rho_0$, the minimum distance between individual points is $a-2\rho_0 a = a(1-2\rho_0)$. We therefore choose $A = B(0,b)\times [0, \sqrt{\frac a2(1-2\rho_0)}]$ in order for spheres to never overlap \unsure{This is perhaps unnecessarily conservative, we could widen it}. 


