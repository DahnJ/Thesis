\documentclass{article}
\usepackage[czech]{babel}
\usepackage[utf8]{inputenc}
\usepackage[T1]{fontenc}

\begin{document}

V posledních letech došlo k pokrokům v modelování polykrystalických materiálů za pomoci parametrických teselačních modelů ze stochastické geometrie. Třída teselací Gibbsova typu umožňuje uživateli specifikovat široké spektrum vlastností skrze funkci energie.
Tato práce se zaměřuje výhradně na tetrahedronizace Gibbsova typu, tedy na teselace ve tří dimenzionálním prostoru skládajících se ze čtyřstěnů. Existující výsledky pro rovinnou Delaunayho triangulaci jsou rozšířeny na případ Laguerreovy tetrahedronizace ve třech dimenzích. Tato práce dokazuje existenci modelů tohoto typu a zároveň, za pomoci vlastní C++ implementace, poskytuje výsledky MCMC simulace a odhadů parametrů skrze maximální pseudověrohodnost.



\end{document}