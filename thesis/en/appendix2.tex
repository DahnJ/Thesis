\chapter{Appendix: Implementation details}\label{appendix:implementation}
\subsubsection{C++ and CGAL}
The MCMC procedure introduced in Chapter \ref{ch:simulation}, as well as the maximum pseudolikelihood estimation procedure introduced in Chapter \ref{ch:estimation}, was implemented in \texttt{C++} using the Computational Geometry Algorithms Library (\cite{cgal}). In particular, the package for 3D-triangulations (\cite{cgal:3d-triang}) was used. CGAL provides an implementation of both Delaunay and Laguerre\footnote{Laguerre tetrahedrization is called the \textit{3D regular triangulation} in CGAL.} tetrahedrizations which allows both adding and removing points. 

The current implementation allows control of the model parameters as well as number of iterations through command line arguments. For anything else, alterations of the code are required, although many changes (such as specifications of the form of the potential) can be done quickly, as functions for various characteristics such as volume, surface area, minimum edge length, etc., are already implemented. 

Three files are outputted for each simulation: cell data and estimation results, the tetrahedrization itself, and a log of the MCMC procedure.

In the near future, a simple documentation will be provided. 

The latest version of the code is available at \url{https://github.com/DahnJ/Gibbs-Laguerre-Delaunay}. 

\subsubsection{Python analysis}
The \texttt{C++} program outputs cell data in a standard \texttt{csv} format. These were subsequently analyzed in Python using the Jupyter notebook environment. The notebook can be found in the same repository as the \texttt{C++} code in the folder \texttt{python}. 


\subsubsection{Wolfram Mathematica}
In Remark \ref{r:zMathematica} and in Theorem \ref{thm:Apollonius}, Wolfram Mathematica (\cite{Mathematica}) was used. The Mathematica notebooks are contained in the repository of this thesis, available at \url{https://github.com/DahnJ/Thesis} in the folder \texttt{mathematica}.


