\chapter{Stochastic geometry}\label{ch:2}
Ultimately we want to study the behaviour of hypergraph structures and hyperedge potentials under some probabilistic assumptions on the distribution of the configuration $\x$. This chapter introduces the theory of point processes and random tessellations, both examples of the area of stochastic geometry, the concepts that will allow us to introduce randomness into hypergraphs. The main goal of this chapter is to introduce the Gibbs-type tessellation, where the location of the points are allowed to interact with the geometric properties of the tesssellation, giving us a great freedom in the specification of our models.

\section{Point processes}
This section will develop the bare minimum of the theory necessary to define and use Gibbs point processes. For a comprehensive introductory text, we recommend \cite{MollerWaagepetersen2003}, as it is the most relevant text. 


In general, we assume $E$ to be a locally compact complete separable space. This is the setting in many texts, such as \cite{SchneiderWeil2008}\unsure{Really? Check it}.

The main aim of this text is to build Gibbs point processes with interactions based on the Laguerre tetrihedrizaion. As such, the focus is on marked points and the Delaunay case is treated as secondary. To avoid having a dual marked and unmarked theory, we will treat unmarked point as a special case of marked points in the following way. 

\begin{itemize}
	\item Marked case: We take $E=\Rt\times S$ where $S=[0,W],W>0$ is the space of marks. The measure on $E$ is $z\lambda \otimes \mu$, where $\mu$ is a non-atomic probability distribution of marks, $z>0$. 
	\item Unmarked case: We use the same space, but the distribution of marks $\mu=\delta_0$ is now concentrated on $0$.
\end{itemize}


\subsection{Basic terms}
\begin{definition}\label{def:measures}
Define a \textit{counting measure} on $E$ as a measure $\nu$ on $E$ for which
$$\nu(B)\in\mathbb N \cup \{0,\infty\}, B\in\mathcal B_0(E)\;\;\text{ and } \; \; \nu(\{x\})\leq 1, x\in E.$$
We say a measure $\nu$ is \textit{locally finite} if $\nu(B)<\infty$ for any $B\in \mathcal  B_0(E)$. Denote $N_{lf}(E)$ the space of all locally finite counting measures on $E$.
We equip the space $N_{lf}(E)$ with the $\sigma$-algebra
$$\mathcal N_{lf}(E)=\sigma(\{\nu \in N_{lf}(E): \nu(B)=n\}: B \in \mathcal B_0(E), m\in \mathbb N_0).$$
\unsure{Maybe pospone this to a later section?}Finally we define the set $N_f(E) \subset N_{lf}(E)$ of finite measures on $E$ by
$$N_f(E) = \{\nu \in N_{lf}(E): \nu(E)<\infty\}$$
with the $\sigma$-algebra $\mathcal N_{f}$ defined as the trace $\sigma$-algebra of $N_{f}(E)$ on $(N_{lf}(E),\mathcal N_{lf}(E))$.
\end{definition}

We use the shortened notation $N_{lf}(\Rt\times S) := N_{lf}$. Similarly for the terms $N_{f}, \mathcal N_{f}, \mathcal N_{lf}, \mathcal B, \mathcal B_0$.
\begin{remark}[Simple PP]
\end{remark}


\begin{remark}[Duality of locally finite counting measures and configurations]
In chapter 1, we introduced the sets $N_{lf}$ and $N_f$ as spaces of (finite) configurations --- locally finite sets. This abuse of notation is justified by the fact that there is a measurable bijection between the space of locally finite counting measures as defined here and locally finite sets. For details and a proof, see lemma 3.1.4. in \cite{SchneiderWeil2008}.   
\end{remark}

\begin{definition}
A \textit{point process} on $E$ is a measurable mapping $\Phi:(\Omega,\mathcal A, P) \to (N_{lf}(E),\mathcal N_{lf}(E))$. \newline
A \textit{marked point process} $\Phi_m$ as a point process on $\Rt\times S$ for which the projection $\Phi(B)=\Phi_m(B\times S), B \in \mathcal B$ is a point process on $\Rt$.
\end{definition}
Note that this definition requires the realizations of the projection of the marked point process to be locally finite counting measures in the sense of definition \ref{def:measures}.

\unsure[inline]{Do we need anything else?}
% History of Campbell: https://www.jstor.org/stable/3621649?seq=4#metadata_info_tab_contents

\subsubsection{Poisson point process}
Before we define the Poisson point process, we first define a process closely related it.

\begin{definition} Let $\nu$ be a measure on $E$, $B \in \mathcal B_0(E)$ such that $0<\nu(B)<\infty$. For $n\in \mathbb N$ let $X_1,\dots,X_n$ be independent and $\nu$-uniformly distributed random variables on $B$, that is
	$$P(X_i \in A) = \frac{\nu(A)}{\nu(B)},\; A\in\mathcal B(E) \subset B$$
Then we define the \textit{binomial point process} of $n$ points in $B$ as
$$\Phi(n) = \sum^{n}_{i=1}  \delta_{X_i}.$$
We use the convention $\sum^0_{i=1} \delta_{X_i} = \varnothing$, where $\varnothing(E)=0$ is the empty point process.
\end{definition}
In the marked case, $X_i=(X'_i,M_i)$ where $X'_i$ is the position and $M_i$ the mark of $Y_i$ and we can write
$$\Phi(n) = \sum^{n}_{i=1} \delta_{(X'_i,M_i)}.$$
However, similarly to chapter 1, not explicitely stating the positional and mark part leads to a cleaner notation.

\begin{proposition}\label{bincalc}
	Let $\Phi_n = \sum^{n}_{i=1}  \delta_{X_i}$ be a binomial point process on $B\in\mathcal B_0(E)$. Then for a non-negative measurable $f$ we have
\begin{equation}\label{binom}
	Ef(X_1,\dots, X_k) = \frac 1{\nu(B)^k} \int_B \cdots \int_B f(x_1,\dots, x_k) \nu(dx_1) \cdots \nu(dx_k), \quad k = 1,\dots, n
\end{equation}
\end{proposition}
\begin{proof}
From the definition of $\Phi_n$, we have for Borel $A_i \subset B, i=1,\dots,k$ that
\begin{align*}
P(X_1 \in A_1, \dots, X_k \in A_k) &= P(X_1\in A_1)\cdots P(X_k\in A_k) \\ 
& = \frac 1{\nu(B)^k} \int_B \cdots \int_B 1_{A_1}(x_1) \cdots 1_{A_k}(x_k) \nu(dx_1) \cdots \nu(dx_k) \\
\end{align*}
That is \ref{binom} for $f(x_1,\dots,x_k)=1_{A_1}(x_1)\dots 1_{A_k}(x_k)$. By a standard argument, we first extend this to a general set $C \in \mathcal B^k(E), C\subset B^k$ using the Dynkin system 
$$\{C \in \mathcal B^k(E): E 1_C (x_1,\dots,x_k) = \int \cdots \int 1_C(x_1,\dots, x_k) dx_1 \cdots dx_k \}$$
 and then from indicators to any non-negative measurable function.
\end{proof}

\problem[inline]{The $\mathcal B^k$ is weird there, considering that we kinda have $\mathcal B^3=\mathcal B$ elsewhere}



\begin{definition} Let $\nu=$ be a measure on $E$. A point process $\Phi$ satisfying
\begin{enumerate}
	\item $\Phi(B)$ has a Poisson distribution with parameter $\nu(B)$ for each $B\in \mathcal B_0(E)$,
	\item Conditonally on $\Phi_B=n, n\in\mathbb N$,  $\Phi|_B$ is the Binomial point process of $n$ points in $B$, $B \in \mathcal B_0(E)$.
\end{enumerate}
is a \textit{Poisson process} on $E$ with \textit{intesity measure} $\nu$.
For $B\in \mathcal B_0(E)$, denote $\Pi^\nu_B$ the distribution of a Poisson point process with intensity measure $\nu$ restricted to $B$.  
\end{definition}

\begin{definition}We define the \textit{marked Poisson process} is a Poisson process on $\Rt\times S$ with intensity measure $z\lambda \otimes \mu$. We call the parameter $z$ the \textit{intensity}.\newline


	For $\Lambda\in \mathcal B_0(\Rt)$, denote $\Pi^z_B$ the distribution of  marked Poisson point process with intensity $\nu$ restricted to $\Lambda$. For $z=1$, we lose the $z$ and denote the distribution simply $\Pi_\Lambda$.
\end{definition}
Notice that the set $\Lambda$ refers only to the positions of the points. This is because we will always work with the whole mark space $S$.

\unsure[inline]{We could also define $\Pi_\Lambda$ as the marginal, without marks. Think this through}

Note that thanks to \ref{bincalc} we have for a marked Poisson process $\Phi$ with intensity $z$ and $\Gamma \in \mathcal N_{lf}$ 

\begin{align}\label{eq:poiscalc}
	\Pi^z_\Lambda(\Gamma) &= P(\Phi \in \Gamma) = \sum^\infty_{k=0} P(\Phi \in \Gamma | \Phi(\Lambda) = k) P(\Phi(\Lambda)=k) \\
	& = \sum^\infty_{k=0} \frac{(z|\Lambda|)^k}{k!} e^{-z|\Lambda|} P(\Phi^{(k)}\in \Gamma)\\ 
	& = \sum^\infty_{k=0} \frac{z^k}{k!} e^{-z|\Lambda|} \int_{\Lambda\times S} \cdots \int_{\Lambda\times S} 1_{\Gamma} \left(\sum^k_{i=1} \delta_{X_i}\right) \nu(dx_1), \dots, \nu(dx_k)\\
\end{align}
where $\Phi^{(k)} = \sum^k_{i=1}\delta_{(X_i,M_i)}$ denotes the Binomial point process of $k$ points in $C$ and $\nu=\lambda \otimes \mu$. 

\begin{remark}[Points in general position]
	In section \ref{sec:tetrahedrizations} we introduced the sets $N_{gp}$ and $N_{rpp}$. \cite{Zessin2008}.
\end{remark}


\subsection{Finite point processes with density}
\todoo[inline]{Analogy with random variables, why Poisson is the best}
\problem[inline]{Restriction to finite set? Define Nf properly. Other problems with this..? Define finite point processes?}
In this chapter, we limit ourselves entirely to the case $E=\Rt \times S$. At the same time, we will stop using the term ``marked'' where we deem it redundant. 

\begin{definition}
	We say that a point process $\Psi$ on $\Rt \times S$ \textit{has the density $p$ with respect to the Poisson process} if its distribution is absolutely continuous w.r.t. $\Pi_\Lambda$ with density function $p$. That is there exists a measurable function $p:\mathcal N_f \to \mathbb R^+$ such that $\int p(\gamma) \Pi_\Lambda (\gamma)=1$ and
$$P(\Psi \in \Gamma) = \int_\Gamma p(\gamma) \Pi_\Lambda(d\gamma), \; \Gamma \in \mathcal N_{f}$$
\end{definition}

\todoo[inline]{These calculations are overly complicated now, make them clearer}
Notice that using the calculations in \ref{bincalc} and \ref{eq:poiscalc} we have
$$
P(\Psi \in \Gamma) = \sum^\infty_{k=0} \frac{1}{k!} e^{-|\Lambda|} \int_{\Lambda\times S} \cdots \int_{\Lambda\times S} 1_{\Gamma} \left(\sum^k_{i=1} \delta_{X_i}\right) p\left(\sum^k_{i=1} \delta_{X_i}\right) \nu(dx_1) \dots \nu(dx_k)
$$
where $\nu=\lambda\otimes \mu$.
	The equation above is a special case of 
$$Eh(\Psi)=Eh(\Phi)p(\Phi)$$
for $\Pi_\Lambda$-measurable function $h$, where $\Phi \sim \Pi^z_\Lambda$.

A useful function for dealing with point processes with density is the Papangeloou conditional intensity.
\begin{definition}
	For  a point process $Phi$ with density $p$ we define the \textit{Papangelou conditonal intensity} as 
	$$\lambda^*(x,\gamma) = \frac{p(\gamma + \delta_x)}{p(\gamma)}, \; x \in \Rt \times S, \gamma \in N_{f}: p(\gamma)>0.$$
\end{definition}

\begin{proposition} $\Pi_\Lambda^z \ll \Pi_\Lambda$ with density $p(\gamma)=z^{|\gamma|} \exp(|\Lambda|(1-z))$
\end{proposition}
\begin{proof}
	Denote $\Phi \sim \Pi_\Lambda$, we have for $\Gamma\in \mathcal N_{f}$, using \ref{eq:poiscalc}
	$$\Pi^z(\Gamma) = E(1_\Gamma(\Phi)  z^{|\Phi|} e^{|\Lambda|} e^{-z|\Lambda|}) $$
\end{proof}

\subsection{Gibbs Point Processes}
\todoo[inline]{This section is a mess, edit}
\problem[inline]{This really doesn't work. Instead start with connecting energy to the energy in the last chapter. Asssume it to be hereditary and define fGPP already within a hypergraph structure, DLR, then simply define GPP (careful about cr or admissible energy) through DLR. Then talk about GNZ and say what happens in the non hereditary case} 

A large class of point processes are the Gibbs point processes, the main object of our study.

\begin{definition}\label{def:fGPP}
The \textit{finite Gibbs measure} on $\Lambda$ with activity $z>0$ is the distribution $P^z_\Lambda$ such that $P^z_\Lambda \ll \Pi_\Lambda$ with density
$$p(\gamma) = \frac 1{Z^{z}_\Lambda} z^{\gamma(\Lambda)} e^{-H(\gamma)}.$$
where  \problem{Notation for $N_\Lambda$}$Z^z_\Lambda = \int z^{N_\Lambda} e^{-H} \Pi_\Lambda$ is the normalizing constant, called \textit{partition function}. \newline
The measurable function $H:N_f\to \mathbb R \cup \{+\infty\}$ such that $Z^z_\Lambda < \infty$. \newline
Process with the distribution $P^z_{\Lambda}$ is called the \textit{finite Gibbs point process} (finite GPP). 
\end{definition}

\todoo[inline]{Show DLR, since that's how infinite is later defined} 

Due to its defintion, the finite GPP favours configurations with low energy. Configurations with high energy are unlikely to happen and an infinite energy means that the configuration is not possible under the distribution, called \textit{forbidden}. A configuration that is not forbidden is \textit{allowed}. 

Before we continue onto extending the definition to $N_{lf}$, we will turn to the properties and form of the energy function.


\subsubsection{The energy function}
Here we will connect the definition of the energy function from definition \ref{def:fGPP} with that from definition \ref{def:energy}. Thanks to the energy function, we can force the realizations of the finite GPP to obey a diverse set of geometrical properties. In our case those geometrical properties come through the structure of $\mathcal D$ and $\mathcal {LD}$, see example \ref{ex:potentials}.  
The energy function is where the power of Gibbs point processes lies, but also where some of the difficulties arise. 

Traditionally, the energy function is required to satisfy some assumptions. Here we list those from \cite{Dereudre2017}.
\todoo[inline]{Make stationarity more explicit. Also connect this with the last chapter better}
\begin{itemize}
	\item \textbf{Non-degeneracy}:\unsure{I don't really understand the role of $\emptyset$ in Gibbs theory.}
		$$H(\varnothing) < +\infty.$$
	\item \textbf{Hereditarity}: For any $\gamma \in N_f$ and $x\in \gamma$
		$$H(\gamma)< + \infty \Rightarrow H(\gamma - \delta_x) < +\infty.$$
	\item \textbf{Stability}: there exists a constant $c_S\geq 0$ such that for any $\gamma \in N_f$
		$$H(\gamma) \geq c_S\cdot \gamma(\Rt\times S).$$
\end{itemize}

Stability bounds the density function $p(\gamma) \propto z^{\gamma(\Lambda)}e^{-H(\gamma)} \leq (z e^{-c_S})^{\gamma(\Lambda)} $ and thus ensures $Z^z_\Lambda < \infty$. Integrability of the density is obviously a necessary assumption and thus some form of stability cannot be avoided. 
Non-degeneracy, when paired with hereditarity, is a very natural assumption; without it, hereditarity would imply that the energy is always infinite.

Hereditarity ensures that removing a point will not result in a forbidden configuration. Equivalently it ensures that adding a point to a forbidden configuration will not result in an allowed configuration. This assumption is, however, not necessarily satisfied by our models. Take for example the hard-core exclusion potential. Removing a point can lead to emergence of a tetrahedron with a larger circumdiameter, thus resulting in a forbidden configuration.

To see the usefulness of hereditarity, we first assume $H$ is hereditary. For $\gamma \in N_f$ and $x \in \Rt\times S$, define
$$h(x,\gamma) = H(\gamma \cup \delta_x) - H(\gamma),$$
with the convention $+\infty - (+\infty) = 0$. Notice that for $\gamma \in N_f$ such that $p(\gamma) > 0$, where $p$ is now the density of a finite GPP, we have
$$\lambda^*(x,\gamma) = z \cdot e^{-h(x,\gamma)}.$$

We then obtain the following result, known as the \textbf{Georgii-Nguyen-Zessin }(\textbf{GNZ})\textbf{equations}. 

\todoo[inline]{Define supp or say we will treat them as sets}
\begin{proposition} Let $\Lambda \in \mathcal B(\Rt)$ such that $|B|>0$. For any non-negative measurable function $f$ from $(\Rt\times S)\times N_f$ to $\mathbb R$,
	\begin{equation}\label{eq:GNZ}\int \sum_{x \in \gamma} f(x,\gamma- \delta_x) P^z_\Lambda (d\gamma) = z \int \int_{\Lambda\times S} f(x,\gamma) e^{-h(x,\gamma)} dx P^z_\Lambda (d\gamma).\end{equation}
	Furthermore $P^z_\Lambda$ is uniquely defined by \ref{eq:GNZ} in the sense that if a probability measure $P$ on $N_f$ satisfies \ref{eq:GNZ}, then $P=P^z_\Lambda$.
\end{proposition}
\begin{proof}
	Direct adaption of propositions 4 and 5 from \cite{Dereudre2017}, where we take $d=4$, use the last dimension as the space of marks concentrated on $[0,W]$ and then continue the proof using \ref{bincalc}. \todoo{Also need to refer on marked Slivnya-Mecke or sth}
\end{proof}

Hereditarity thus gives us a powerful characterization of the finite Gibbs measure. This characterization remains true even for (infinite) Gibbs measures, see theorem 2 in section 2.5 in \cite{Dereudre2017}. Possibly even more important is that a number of estimation techniques (maximum pseudolikelihood used here being one of them) make use the Papangelou conditional intensity and GNZ equations.
 
Luckily, the approach in \unsure{Possibly cite the later edition? What's the approach here?} \cite{DereudreLavancier2007} allows us to directly use GNZ equations even for the non-hereditary case. Before we present the solution, we must first extend the definition of GPP to $N_{lf}$. Here we will diverge from \cite{Dereudre2017}, which requires a strong finite range property which our models do not satisfy, and take the approach of \cite{DDG12}, which uses the weaker range confinement property defined in definition \ref{def:cr}. 

\subsubsection{Infinite volume Gibbs measures}
\todoo[inline]{maybe connect this with the discussion already writte in the energy section - it doesn't make sense for infinite sets etc}

\todoo[inline]{Perhaps instead of saying all this, just assume $\gamma \in N^\Lambda_{cr}$ and say it will be clarified later}
First, define $\Theta=(\vartheta_x)_{x\in \Rt}$ be a group of translations $\vartheta_x$ defined in definition \label{def:potential}. The set $\mathcal P_\Theta$ denotes the set of all $\Theta$-invariant probability measures on $(N_{lf},\mathcal N_{lf})$ with $\int N_{[0,1]^3 \times S} dP< \infty $ \todoo{maybe connect this to intensity, i.e. define intensity etc}. Under an additional assumption presented in the next chapter, we then obtain that $\Theta$-invariant measures are already concentrated on $N^\Lambda_{cr}$.

\problem[inline]{The set $\Lambda$ should probably always have positive measure. Check this and write it somewhere}
\todoo[inline]{References to assumptions clear}
\begin{proposition}
	Let $\Lambda \in \mathcal B_0(\Rt)$. Under the assumption \ref{(R)}, there exists a set \todoo{Define $N_\Lambda$}$\hat N^\Lambda_{cr} \in N_{\Lambda^c}$ such that $\hat N^\Lambda_{cr} \subset N^\Lambda_{cr}$ and $P(\hat N^\Lambda_{cr})=1$ for all $P \in \mathcal P_\Theta$ with $P(\varnothing)=0$.
\end{proposition}
\begin{proof}
	Can be found in proposition 5.4. in \cite{DDG12}. See also remark 3.7. in connection to the marked case.
\end{proof}

Thanks to this fact we can now use the form of the energy function in proposition \ref{prop:Hcr} and define the (infinite volume) Gibbs measure and Gibbs point process.

\begin{definition}

\end{definition}


While the definition is simple and analogous to the finite case, proving the existence is not. The existence and uniqueness of Gibbs measures is an active field of research and one where we still currently do not know much, particularly in case of uniqueness. The non-uniqueness is a consequence of the fact that the existence of a Gibbs measure is typically proven only through proving tightness of a sequence of finite Gibbs measures, thus yielding only a convergent subsequence. We will not delve into the topic further here and we refer the reader to an introductory text \cite{Dereudre2017} and the paper on which we base the proof of existence for our models, \cite{DDG12}. We also recommend reading the introduction to \cite{Georgii2011} --- although the book is about Gibbs random fields rather than point processes, the introduction gives an intuitive explanation for the form of the density and in particular the connection of the non-uniqueness with phase transitions.


Having defined the Gibbs measure, we can now continue to present the approach of \cite{DereudreLavancier2007} extending the GNZ equations to Gibbs point processes with non-hereditary energy functions.

Define $N_\infty$
$$N_\infty = \{\x \in N_{lf}: \forall \Lambda \in B_0(\Rt): H_\Lambda(\x)< \infty\}$$
\unsure[inline]{Measurability?}

\begin{definition}
	Let $\gamma \in N_\infty$. We say $x\in\gamma$ is \textit{removable} if 
	$$\text{there exist } \Lambda \in \mathcal B(\Rt) \text{ such that } x\in\Lambda \text{ and } H_\Lambda(\gamma - \delta_x)<\infty$$
\end{definition}


\begin{lemma}\label{lemma:DL07}
	There exists a measurable function $\psi_{\Delta,\Lambda}:N_{lf}\to \mathbb R\cup\{+\infty\}$ such that
	$$\forall \gamma \in N_{lf},\quad H_\Lambda(\gamma) = H_\Delta(\gamma) + \psi_{\Delta,\Lambda}(\gamma_{\Delta^c})$$
\end{lemma}
\begin{proof}
	To find such function, we only need to realize that
	$$H_\Lambda(\gamma) - H_\Delta(\gamma) = \sum_{\eta \in \mathcal E_\Lambda(\gamma) \setminus \mathcal E_\Delta(\gamma)} \varphi(\eta,\gamma)$$
	depends only on $\gamma_{\Delta^c}$. As noted below the definition \ref{def:Eset}, both $\mathcal E_\Delta(\gamma)$ and $\mathcal E_\Lambda(\gamma)$ depend only on $\gamma$ outside the window $\Lambda$ and $\Delta$ respectivelly.By $\eta \notin \mathcal E_{\Delta}(\gamma)$ we have that $\forall \zeta \in N_\Delta: \varphi(\eta,\gamma)=\varphi(\eta,\zeta \cup \gamma_{\Delta^c})$ and thus we can set
	$$\psi_{\Delta,\Lambda}(\gamma_{\Delta^c}) = \sum_{\eta \in \mathcal E_\Lambda(\gamma_{\Delta^c}) \setminus \mathcal E_\Delta(\gamma_{\Delta^c})} \varphi(\eta,\gamma_{\Delta^c}).$$
\end{proof}

\todoo[inline]{Say what the set is equal to for us}


\begin{proposition}
	Let $\gamma \in N_\infty$, then $x\in\gamma$ is removable if and only if $\gamma - \delta_x \in N_\infty$.
\end{proposition}
\begin{proof}
	Follows from lemma \ref{lemma:DL07} above and proposition 1 in \cite{DereudreLavancier2007}.	
\end{proof}


\begin{definition}
	Let $x$ be a removable point in a configuration $\gamma$ in $N$. The local energy of $x$ in $\gamma - \delta_x$ is defined as
	$$h(x,\gamma - \delta_x) = H_\Lambda (\gamma) - H_\Lambda(\gamma - \delta_x)$$
	where $\Lambda \in \mathcal B_0(\Rt)$
\end{definition}
Let us remark that such set always exists by definition and that the value of $h(x,\gamma-\delta_x)$ does not depend on the choice of $\Lambda$ as a consequence of $\ref{lemma:DL07}$.


\begin{proposition}
	Let $P$ be a stationary Gibbs measure. For every bounded non-negative measurable $f:(\Rt\times S) \times N_{lf}\to\mathbb R$ we have
	$$\int 1_{N_\infty}(\gamma) \sum_{x \in \gamma} f(x,\gamma -\delta_x) P(d\gamma) = z \int \int f(x,\gamma)e^{-h(x,\gamma)} dx P(d\gamma).$$
\end{proposition}
\begin{proof}
	See proposition $2$ \unsure{But what about marks, does it work?}in \cite{DereudreLavancier2007}.
\end{proof}

Note that we lose the converse implication. That is the GNZ equations no longer characterize Gibbs point process with non-hereditary energy function. Imagine a measure $P$ under which $\gamma$ a.s. does not contain any removable points. The equation then becomes the trivial equation $0=0$.


\todoo[inline]{Measurability!}
\unsure[inline]{Any use in mentioning Markov processes and such}



\section{Random tessellations}
\unsure[inline]{Is there any use for this chapter?}
