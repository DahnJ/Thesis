\chapter*{Conclusion}
\addcontentsline{toc}{chapter}{Conclusion}

This text has dealt with stochastic Laguerre-Delaunay tetrahedrizations based on the Gibbs point process. We have presented parametric hard-core models which allow us to control various properties of the resulting tetrahedrizations, such as the surface area of the tetrahedra. Existence of these models has been proven using the approach in \cite{DDG12}. Then we focused on extending the results of \cite{DereudreLavancier2011} in simulation and estimation of model paraemeters for these models. Simulation was done through a Markov chain Monte Carlo procudure and implemented in \texttt{C++}. Estimation was done through the maximum pseudolikelihood method, extended to the non-hereditary case in \cite{DereudreLavancier2009}.


A considerable number of results is outside the scope of this text, not necessarily because of the difficulty of obtaining them, but simply because of the limited time allotted for its completion. 

The first extentsion would be to complete the results in this text --- that is extend the existing results of \cite{DereudreLavancier2009}  in irreducibility of the Markov chain and \cite{DereudreLavancier2011} in consistency of the MPLE estimates to the three-dimensional Laguerre-Delaunay case. As with existence, it is likely the case that the extension to Laguerre would be easier than the extension to three dimensions. 

Second extension is to study the stability of the potentials without the assumptions of non-negativity. The question, as noted in \todoo{Write sth on this}[ref], depends on the complexity of the tetrahedrization. It seems likely to us that there are relatively easily obtainable results on the complexity of the tetrahedrization of the Poisson process and then perhaps even for the Gibbs point process. Obtaining some results for the tetrahedrization generated by the Poisson process would be interesting in and of itself, as the study of such structures is popular in the field of computational geometry.

A number of other extensions were considered. There are many possibilities in specifying potentials of a different form, in particular studying non-unary potentials by including explicit tetrahedral interactions. This would require new proofs of existence, convergence of MCMC and consistency of the estimation techniques. In the practical sense, it should be relatively easy to extend our \texttt{C++} implementation to include tetrahedral interactions.

Different estimation techniques could be tried. Although we are limited in their choice by the absence of heredity, options do exist. One of them is the variational estimator which is applicable even in the non-hereditary case (\cite{BaddeleyDereudre2013}).

In the practical implementation, an interesting comparison would be to try to use the periodic outside configuration as \cite{DereudreLavancier2011} does.

Finally, instead of using the power distance to define the tetrahedrization, we could use a weighted Euclidean metric (\cite{Gavrilova}), resulting in the dual of the so-called Apollonius diagram, or Johnson-Mehl tessellation. 

