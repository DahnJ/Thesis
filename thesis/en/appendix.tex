\appendix

\problem[inline]{This chapter needs better notation. E.g. $S(p_1,p_2,p_3,p_4)$ for a sphere defined by those points, etc.}

\chapter{Appendix}
\section{Calculating the circumdiameter}
Consider the points $p_1,\dots, p_5 \in \mathbb R^4$ which form a $4$-simplex. Denote $d_{ij} = \|p_i - p_j\|, i,j=1,\dots,5$. Then its area $A$ is given by the \textbf{Cayley-Menger determinant}[ref sommervile]. 

$$
-9216 A^2 =
\begin{vmatrix}
0 & 1 & 1 & 1 & 1 & 1 \\
1 & 0 & d^2_{12} & d^2_{13} & d^2_{14} & d^2_{15} \\
1 & d^2_{21} & 0 & d^2_{23} & d^2_{24} & d^2_{25}  \\
1 & d^2_{31} & d^2_{32} & 0 & d^2_{34} & d^2_{35} \\ 
1 & d^2_{41} & d^2_{42} & d^2_{43} & 0 & d^2_{44} \\
1 & d^2_{51} & d^2_{52} & d^2_{53} & d^2_{54} & 0 
\end{vmatrix} 
$$

Now consider non-coplanar points $p_1,\dots, p_4 \in \Rt$ forming a $3$-simplex, i.e. a tetrahedron. To obtain the circumradius of this tetrahedron, we imagine $p_1,\dots, p_4$ to lie on a $3$-dimensional hyperplane $H$ in $\mathbb R^4$ and we consider the point $c \in H$ such that $\|c-p_i\| = r \forall i=1,\dots,4$ $d\in \mathbb R$. The point $c$ is, by definition, the center of the circumsphere of $p_1,\dots,p_4$ and $d$ is the circumradius. The circumradius $r$ can be obtain by the Cayley-Menger determinant, because $p_1,\dots,p_4,c$ now form a $4$-dimensional simplex of volume $0$. We therefore have 


$$
0 = 
\begin{vmatrix}
0 & 1 & 1 & 1 & 1 & 1 \\
1 & 0 & d^2_{12} & d^2_{13} & d^2_{14} & r^2 \\
1 & d^2_{21} & 0 & d^2_{23} & d^2_{24} & r^2 \\
1 & d^2_{31} & d^2_{32} & 0 & d^2_{34} & r^2 \\ 
1 & d^2_{41} & d^2_{42} & d^2_{43} & 0 & r^2 \\
1 & r^2 & r^2 & r^2 & r^2 & 0 \\
\end{vmatrix}, 
$$

where we have again  $d_{ij} = \|p_i - p_j\|, i,j=1,\dots,4$. 

It would be possible to solve this as an equation of $r$. We can however do better. We can subtract $r^2$ times the first row from last and subtract $r^2$ of the first column from the last to obtain the determinant. 


$$
\begin{vmatrix}
0 & 1 & 1 & 1 & 1 & 1 \\
1 & 0 & d^2_{12} & d^2_{13} & d^2_{14} & 0 \\
1 & d^2_{21} & 0 & d^2_{23} & d^2_{24} & 0 \\
1 & d^2_{31} & d^2_{32} & 0 & d^2_{34} & 0 \\ 
1 & d^2_{41} & d^2_{42} & d^2_{43} & 0 & 0 \\
1 & 0 & 0 & 0 & 0 & -2r^2 \\
\end{vmatrix}, 
$$

and expand by the last row, to obtain the equation

$$
2r^2 \begin{vmatrix}
0 & 1 & 1 & 1 & 1 \\
1 & 0 & d^2_{12} & d^2_{13} & d^2_{14} \\
1 & d^2_{21} & 0 & d^2_{23} & d^2_{24} \\
1 & d^2_{31} & d^2_{32} & 0 & d^2_{34} \\ 
1 & d^2_{41} & d^2_{42} & d^2_{43} & 0 \\
\end{vmatrix} 
-
\begin{vmatrix}
1 & 1 & 1 & 1 & 1 \\
0 & d^2_{12} & d^2_{13} & d^2_{14} & 0 \\
d^2_{21} & 0 & d^2_{23} & d^2_{24} & 0 \\
d^2_{31} & d^2_{32} & 0 & d^2_{34} & 0 \\ 
d^2_{41} & d^2_{42} & d^2_{43} & 0 & 0 \\
\end{vmatrix} = 0 
$$

, from which $r^2$ is directly expressible

\begin{equation}\label{eq:Cayley-Menger-expanded}
r^2 
=
\frac{
\begin{vmatrix}
1 & 1 & 1 & 1 & 1 \\
0 & d^2_{12} & d^2_{13} & d^2_{14} & 0 \\
d^2_{21} & 0 & d^2_{23} & d^2_{24} & 0 \\
d^2_{31} & d^2_{32} & 0 & d^2_{34} & 0 \\ 
d^2_{41} & d^2_{42} & d^2_{43} & 0 & 0 \\
\end{vmatrix}}
{2 \begin{vmatrix}
0 & 1 & 1 & 1 & 1 \\
1 & 0 & d^2_{12} & d^2_{13} & d^2_{14} \\
1 & d^2_{21} & 0 & d^2_{23} & d^2_{24} \\
1 & d^2_{31} & d^2_{32} & 0 & d^2_{34} \\ 
1 & d^2_{41} & d^2_{42} & d^2_{43} & 0 \\
\end{vmatrix} 
}.
\end{equation}

It is worth noting that the determinant cannot equal zero, since it is again a Cayley-Menger determinant and we assumed $p_1,\dots,p_4$ to be non-coplanar. 



\section{Bounding the circumdiameter hyperedge potential}

We have the following optimization problems.

For the regular tetrahedron, the problem is 
\begin{equation}\label{prob:tetra1}
\begin{aligned}
& \underset{p_1,p_2,p_3,p_4\in \Rt}{\text{maximize}}
& & \delta(\{p_1,p_2,p_3,p_4\}) \\
& \text{subject to}
& & \| p_i - t_i\| \leq \rho_0 a, t_i\in \Rt i=1,2,3,4, \\
& & &\|t_i - t_j\| = a, i=1,2,3,4. 
\end{aligned}
\end{equation}

To state the second problem, first denote 
$$D = \begin{pmatrix}
0 & \sqrt a & a & a \\
\sqrt a & 0 & a & a \\
a & a & 0 & a\\
a & a & a & 0
\end{pmatrix}.
$$

Denote the entries of matrix $D$ as $d_{ij}, i,j=1,2,3,4$. Then the statement is:
\begin{equation}\label{prob:tetra2}
\begin{aligned}
& \underset{p_1,p_2,p_3,p_4\in \Rt}{\text{maximize}}
& & \delta(\{p_1,p_2,p_3,p_4\}) \\
& \text{subject to}
& & p_i \in \bar B(t_i, \rho_0 a), t_i\in \Rt i=1,2,3,4, \\
& & & \|t_i-t_j\| = d_{ij}, i,j=1,2,3,4.\\
\end{aligned}
\end{equation}


This is a non-linear optimization problem. We can arrive at its solution through some careful geometric arguments.


First, define the \textit{circumdiameter function} of point $p \in \Rt$ with respect to non-collinear points $p_1,p_2,p_3 \in \Rt$:
$$c(p) = \delta(\{p,p_1,p_2,p_3\}).$$
Denote $(x_i,y_i,z_i)$ the coordinates of $p_i, i=1,\dots,3$. The following lemma describes the properties of $c(p)$.


\begin{lemma} $c(p)$ is continuous, has a global minimum $c_{min} := \delta(\{p_1,p_2,p_3\})$ and 
$$L_a := \{p \in \Rt: c(p)=a\} = S_{a1} \cup S_{a2}, a \geq c_{min}$$ 
where $S_{a1}$ and $S_{a2}$ are two spheres with diameter $a$ such that $p_1,p_2,p_3 \in S_{a1}\cap S_{a2}$. Furthermore, the centers $c_1, c_2$ of $S_{a1},S_{a2}$ respectivelly, lie in the halfspaces
$$H_+ = \{x \in \Rt: Ax \geq 0 \}, H_- = \{x \in \Rt: Ax \leq 0\},$$
where $A$ defines the hyperplane $H=\{x\in\Rt: Ax = 0\}$ on which $p_1,p_2,p_3$ lie.
\end{lemma}
\begin{proof}
Continuity: From \ref{Cayley-Menger-expanded} we see that $c(p)$ can be seen as a composition of a norm, determinants and division. Determinant is continuous as a function of elements of the matrix since it's a polynomial function. Thus $c(p)$ is continuous.


The we can rewrite $L_a$ as
$$\{p \in \Rt: \exists \text{ sphere } S \text{ s.t. } p_1,p_2,p_3,p \in S \text{ and diam}S = a\}.$$
We must therefore find the number of spheres going through the points $p_1,p_2,p_3$ with the diameter $a$. Denote $S$ a sphere such that $\{p_1,p_2,p_3\}\subset S$ with diameter $a$. Define the hyperplanes
$$H_{12} = \{x\in\Rt: \|x-p_1\| = \|x-p_2\|\}, \;\; H_{23} = \{x\in\Rt: \|x-p_2\|=\|x-p_3\|\}.$$
Then their intersection $H_{12}\cap H_{23}$ is a line $L$, as $p_1,p_2,p_3$ are non-collinear.  The center of $S$ is at distance $a/2$ from all three points and thus lies on $L$. For any point, there are at most two points on a line at a given distance from the point. This proves that there are at most two spheres satisfying the definition of $S$.

Using line $L$, we can also deduce the rest of the proposition. The point on $L$ at a minimum distance to $p_1,p_2,p_3$ is the point $p_{min}:=L\cap H$. We know that $p_{min}$ is equidistant from $p_1,p_2,p_3$ and that it lies on the hyperplane $H$, therefore we have $c(p_{min}) = \delta(\{p_1,p_2,p_3\})$.  

\todoo[inline]{Improve the last bit, possibly simplify}
To see that $c_1$ and $c_2$ must be (non-strictly) separated by the hyperplane $H$, assume WLOG $\{c_1,c_2\}\subset H_+, c_1\neq c_2$. Let $p \in S_{a1}$ and let  $p_R\in \Rt$ be the reflection of $p$ through the hyperplane $H$. The tetrahedron $p_1,p_2,p_3,p_R$ then is a reflection of the tetrahedron $p_1,p_2,\dots, p$ and therefore its circumsphere has diameter $a$ and centre in $H_-$, which is a contradiction. 

\end{proof}

Note that $S_{a1}$ and $S_{a2}$ are not necessarily distinct. In fact, we can see from the proof that the case $S_{a1}=S_{a2}$ is precisely when $a=c_{min}$.



\begin{proposition}
Any solution $(p_1,p_2,p_3,p_4)$ of the problem \ref{prob:tetra1} will lie on a sphere $S$ that is (internally or externally) tangent to the spheres $\partial B(t_i,\rho_0 a), i =1,2,3,4$.
\end{proposition}
\begin{proof}
Denote $c(p_1)=\delta(\{p_1,p_2,p_3,p_4\})=c$ and $S$ the sphere such that $\{p_1,\dots,p_4\}$. 
First, WLOG assume that $p_1 \in B(t_1,\rho_0 a)$ Because $p_1$ maximizes the function $c(p)$, we have $c(p_1)\geq c(p), p\in U$, where $U$ is some small neighborhood of $p_1$. Choose two points, $p_O,p_I\in U\setminus S$ such that 
\begin{enumerate} 
\item $c(p_O)=c(p_I)=b$,
\item $p_I$ is on the inside of $S$ and $p_0$ on the outside of $S$ 
\item $S(p_I,p_2,p_3,p_4)$ and $S(p_O,p_2,p_3,p_4)$ do not equal and their centers lie on the same halfspace ($H_+$ or $H_-$) as $S$. 
\end{enumerate}
Such choice is possible due to continuity of $c(p)$. Yet we arrive at a contradiction, as the level-set $L_b$ now contains two distinct spheres with centres in the same halfspace. 

Assume now that $p_1 \in \partial B(t_1,\rho_0 a)=: S_1$. We now choose $p_I$ and $p_O$ with the additional requirement that they must both lie on $\partial B(t_1,\rho_0 a)$. This fails precisely when $S_1$ and $S$ are tangent, since then \todoo{Make sure "inside" a sphere has a clear meaning} $S_1$ lies either completely inside or outside $S$ and it is no longer possible to choose points both outside and inside. 
\end{proof}

We have found that the solutions to \ref{prob:tetra1} and \ref{prob:tetra2} must lie on a sphere that tangent to the spheres within which points can move. This is a major improvement. One, because now the space of possible solution narrows down to just $2^4=16$ possible quadruples of points (and even less beacause of symmetries), and two, because the two-dimensional equivalent of this problem is a well known \textbf{Apollonius problem}.

First note that if two externally tangent spheres $S_1=((x_1,y_1,z_1),r_1), S_2=((x_2,y_2,z_2),r_2)$  satisfy
$$\|(x_1,y_1,z_1) - (x_2,y_2,z_2)\| = r_1+r_2,$$
similarly, two externally tangent spheres satisfy
$$\|(x_1,y_1,z_1) - (x_2,y_2,z_2)\| = |r_1 - r_2|.$$
By squaring, we obtain the equality
$$(x_1-x_2)^2 + (y_1-y_2)^2 + (z_1-z_2)^2 = (r_1 \pm r_2)^2$$
Where we use $+$ for externally and $-$ for internally tangent spheres.

This means, that the Apollonius problem for spheres $S_1,S_2,S_3,S_4$ is solved by any $S=((x,y,z),r)$ such that
\begin{align}\label{eq:Apollonius}
  (x_1-x)^2 + (y_1-y)^2 + (z_1-z)^2 &= (r_1 \pm r)^2 \\
  (x_2-x)^2 + (y_2-y)^2 + (z_2-z)^2 &= (r_2 \pm r)^2 \nonumber \\
  (x_3-x)^2 + (y_3-y)^2 + (z_3-z)^2 &= (r_3 \pm r)^2 \nonumber \\
  (x_4-x)^2 + (y_4-y)^2 + (z_4-z)^2 &= (r_4 \pm r)^2 \nonumber  
\end{align}

where we can take any combination of $+$ or $-$, yielding altogether $16$ possible solutions. This excludes degenerate cases, which are not relevant in our case. As noted previously, the number of solutions for both $T_1$ and $T_2$ will reduce significantly. For $T_1$, the spheres are completely interchangeable and thus only solutions with different number of $+$ will differ. This yields $5$ possible solutions. Geometrically the number of $+$ can be seen as the number of spheres the solution is externally tangent to. For $T_2$ the situation is more complex -  

Sadly, for most choices of $+$ and $-$, these equations still seem to be too complex for Mathematica to solve. Luckily, we can simplify them further. 

First, for clarity, we define the variables $s_i\in\{+1,-1\},i=1,\dots,4$ instead of relying on the notation $\pm$. We begin by expanding the parentheses to obtain the equations
$$x^2+y^2+z^2 + x_i^2 + y_i^2 + z_i^2 - 2xx_i - 2yy_i - 2zz_i = r^2 + r_i^2 + 2(s_1 r_1 - s_2 r_2)r,\quad i=1,2,3,4$$

By subtracting the $2,3,4$-th equation from the first, we get rid of the quadratic terms and obtain a system of linear equations with four variables and three equations:
$$-2(x_1-x_i)x - 2(y_1-y_i)y -2 (z_1-z_i)z - 2(s_1r_1 - s_2r_2)r + x_1^2-x_i^2 + y_1^2-y_i^2 + z_1^2 - z_i^2 -r_1^2 + r_i^2 = 0, \quad i=2,3,4$$

This system can be solved to obtain expression of $x,y,z$ in terms of $r$. We then substitute those expression into \ref{eq:Apollonius} to obtain $r$. 

Note that exact solutions of $x,y,z$, which we are not interested in, could then by obtained through the linear system.

