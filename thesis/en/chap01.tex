\chapter{Geometric preliminaries}
\unsure[inline]{Are graphs geometric? I mean, geometric graphs are geometric. But graphs in general? Are potentials part of this?}
Before diving into the mathematics of Gibbs-Laguerre-Delaunay tetrihedrization models, we must first lay out the fundamentals of their geometric and combinatorial structure. The key geometric component is the empty sphere property [...] which determines the edge structure, which is in turn analyzed in terms of hypergraphs.
\unsure[inline]{$\mathcal F$ or $\mathcal N$}
Let $\mathcal F_{lf}$ be the set of locally finite sets on $\Rt$, and $\mathcal F_{f} \subset \mathcal F_{lf}$ the set of all finite sets on $\Rt$. An elements of $F_{lf}$ will be usually denoted $\x$ and called a \textit{configuration} and its subset $\eta$. If $|\eta|=4$, as will be the case for the majority of this text, then $\eta$ will be called \textit{tetrahedron}.

\section{Tetrahedrizations}
The aim of this section is to introduce the geometric concepts necessary for the definition of the hypergraph structures in the following section. Definitions might be postponed. 
This text is concerned with two types of tetrihedrizations. 


We introduce the notion of (reinforced) general position. This requirement will be later relaxed.

\begin{definition}
Let $\x \in \mathcal F_{lf}$. We say $\x$ is in \textbf{general position} if 
$$ \eta \subset \x, 2 \leq |\eta| \leq 3 \Rightarrow \eta \text{ is affinely independent.} $$   
Denote $\mathcal F_{gp}\subset \mathcal F_{lf}$ the set of all locally finite configurations in general position.
\end{definition}
\todoo[inline]{Commment on measurability of the set of locally finite sets in general position. This comes from cite[Zessin2008] and the $\mathcal F$ $\mathcal M$ equivalence?}
\todoo[inline]{Also comment on the fact that we need a vector space with measurable inner product etc?}
\note[inline]{It's sufficient to check only subsets with $d+1$ points}
 
\begin{definition}
Let $\x \in \mathcal F_{gp}$. We say $\x$ is in \textbf{reinforced general position} if 
$$ \eta \subset \x, 3 \leq |\eta| \leq 4 \Rightarrow \eta \text{ is non-circular.} $$   
Denote $\mathcal F_{rgp}$ the set of all locally finite configurations in reinforced general position.
\end{definition}
\todoo[inline]{Define cocircular in general}
\todoo[inline]{Again, only need to check $d+2$}


\subsection{Delaunay tetrihedrization}

\begin{definition}
	Let $\eta \in \mathcal F_{gp}$, $|\eta|=4$ be a tetrahedron. The open ball $B(\eta)$ such that $\eta \subset \partial B(\eta)$ is called a \textit{circumball}. The boundary $\partial B(\eta)$ is called a \textit{circumsphere}.
\end{definition}	

Note that the circumball is uniquely defined by $\eta$. 

\begin{definition}
	Let $\x \in \mathcal F_{lf}$ and $\eta \subset \x$. We say that $(\eta,\x)$ satisfies the \textit{empty sphere property} if $B(\eta) \cap \x = \emptyset$. 
\end{definition}

\todoo[inline]{Existence and uniqueness}



\subsection{Laguerre tetrihedrization}


\unsure[inline]{Restrict on non-redundant points? Measurability?}


\section{Hypergraph structures}
Both Delaunay and Laguerre tetrihedrizations can be seen as graphs where two points $p,q\in\x$ are joined if they are part of the same tetrahedron\todoo{satisfying ESP or sth}. For the purposes of this text, a more natural structure will be the hypergraph.
 
\begin{definition}
	A \textit{hypergraph structure} is a measurable subset $\mathcal E$ of $(F_f\times N, \mathcal F_f \otimes \mathcal F)$ such that $\eta \subset \x$ for all $(\eta,\x)\in\mathcal E$. We call $\eta$ a \textit{hyperedge} of $\x$ and write $\eta \in \mathcal E(\x)$, where $\mathcal E(\x) = \{\eta: (\eta,\x) \in \mathcal E\}$. For a given $\x \in \mathcal F_{lf}$, the pair $(\x, \mathcal E(\x))$ is called a \textit{hypergraph}.
\end{definition}
A hypergraph is thus a generalization of a graph in the sense that edges are now allowed to "join" any number of points. A hypergraph structure can be thought of as a rule that turns a configuration $\x$ into the hypergraph $(\x,\mathcal E(\x))$. 



