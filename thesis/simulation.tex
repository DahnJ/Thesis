\documentclass[12pt,a4paper]{report}
\setlength\textwidth{145mm}
\setlength\textheight{247mm}
\setlength\oddsidemargin{15mm}
\setlength\evensidemargin{15mm}
\setlength\topmargin{0mm}
\setlength\headsep{0mm}
\setlength\headheight{0mm}
% \openright makes the following text appear on a right-hand page
\let\openright=\clearpage

%% Settings for two-sided (duplex) printing
% \documentclass[12pt,a4paper,twoside,openright]{report}
% \setlength\textwidth{145mm}
% \setlength\textheight{247mm}
% \setlength\oddsidemargin{14.2mm}
% \setlength\evensidemargin{0mm}
% \setlength\topmargin{0mm}
% \setlength\headsep{0mm}
% \setlength\headheight{0mm}
% \let\openright=\cleardoublepage

%% Character encoding: usually latin2, cp1250 or utf8:
\usepackage[utf8]{inputenc}


%% Further useful packages (included in most LaTeX distributions)
\usepackage{amsmath}        % extensions for typesetting of math
\usepackage{amsfonts}       % math fonts
\usepackage{amsthm}         % theorems, definitions, etc.
\usepackage{bm}             % boldface symbols (\bm)
\usepackage{graphicx}       % embedding of pictures
\usepackage{fancyvrb}       % improved verbatim environment
\usepackage{dcolumn}        % improved alignment of table columns
\usepackage{booktabs}       % improved horizontal lines in tables
\usepackage{paralist}       % improved enumerate and itemize
\usepackage[pdftex,dvipsnames]{xcolor}  % typesetting in color

\usepackage{cite}
\usepackage[colorinlistoftodos,prependcaption,textsize=tiny]{todonotes}
\usepackage{xargs}


\newcommandx{\problem}[2][1=]{\todo[linecolor=red,backgroundcolor=red!25,bordercolor=red,#1]{#2}}
\newcommandx{\todoo}[2][1=]{\todo[linecolor=blue,backgroundcolor=blue!25,bordercolor=blue,#1]{#2}}
\newcommandx{\note}[2][1=]{\todo[linecolor=OliveGreen,backgroundcolor=OliveGreen!25,bordercolor=OliveGreen,#1]{#2}}
\newcommandx{\unsure}[2][1=]{\todo[linecolor=Plum,backgroundcolor=Plum!25,bordercolor=Plum,#1]{#2}}



\begin{document}
\chapter{Simulation and Estimation of Gibbs-Delaunay}


\section{Preliminaries}
For a bounded observation window $\Lambda_0 \in\mathcal B(\mathbb R^3)$ we define  $\Lambda := \Lambda_0 \times W \in \mathcal B(\mathbb R^3 \times W)$, where  $W:=(0,w_0]\subset \mathbb R$ is the set of possible weights. For a locally finite set $\gamma\subset \mathbb R^3\times W $, denote its Delaunay (Regular) triangulation $Del(\gamma)$ and the restriction to $\Lambda$ as  $\gamma_\Lambda = \gamma \cap \Lambda$. Denote $\pi^z$ the distribution of a homogenous Poisson process with intensity $z>0$ and $\pi^z_\Lambda$ its restriction to the observation window $\Lambda$. 

\note[inline]{Define a function to get points/weights from $\gamma$? Might be easier}
\todoo[inline]{Define $\mathcal M_\infty$ etc?}
\problem[inline]{This should probably be w.r.t. the weighted Poisson}
The set $Del(\gamma)$ contains tetrahedra which are here defined as quadruples of points from $\gamma$, i.e. $T=\{a,b,c,d\}\subset \gamma$. A tetrahedron $T$ is said to be inside the window $\Lambda$ \problem{Program uses barycenter: rewrite?}if its circumsphere intersects $\Lambda$. Denote $Del_\Lambda(\gamma)$ the set of all tetrahedra in $Del(\gamma)$ which are in $\Lambda$. For two tetrahedra $T,T'\in Del(\gamma)$ we say $T\sim_{Del} T'$ if $T$ and $T'$ share a face, that is if $\text{card}(T\cap T') \geq 3$. 

\todoo[inline]{Define Gibbs properly - DLR + finite volume, but this will do for now}
To obtain a Gibbs point process from the Poisson process, its density with respect to the Poisson process needs to be defined. For every $\Lambda \in \mathcal B_b(\mathbb R^3\times W)$, we consider the conditional density $f_\Lambda$ with respect to the Poisson process $\pi^z_\Lambda$:

$$ f_\Lambda(\gamma_\Lambda, \gamma_{\Lambda^c}) = \frac 1{Z_\Lambda(\gamma_{\Lambda^c})} e^{-E_\Lambda (\gamma_\Lambda, \gamma_{\Lambda^c})},$$ 

\unsure[inline]{Make sure I understand in what sense the configuration is w.r.t. $\gamma_{\Lambda^c}$}
where $E_\Lambda(\gamma_\Lambda, \gamma_{\Lambda^c})$ is the energy function of $\gamma_\Lambda$ given the outside configuration $\gamma_{\Lambda^c}$. The energy function will be specified in the next section. $Z_\Lambda(\gamma_{\Lambda^c}) := \int e^{-E_\Lambda(\gamma_\Lambda, \gamma_{\Lambda^c})} \pi^z_\Lambda(d\gamma_\Lambda)$ is the normalizing constant.
For a comprehensive introduction to Gibbs point processes, see~\cite{dereudre2017}.

\subsection{Model}
Here we define the model through its energy function. In general, the model specification is of the following form:
$$E_\Lambda(\gamma_\Lambda, \gamma_\Lambda^c)= \sum_{T \in Del_\Lambda(\gamma)} V_1(T) + \sum_{\substack{\{T,T'\} \subset Del(\gamma) \\ T\sim_{Del} T' \\ T \text{ or  } T' \text{ in  } Del_\Lambda(\gamma)}} V_2(T,T').$$

In our case, we introduce hardcore parameters $\epsilon$ and $\alpha$ to limit the area of the smallest face of $T$, and its circumradius respectivelly. The intensity of the Poisson process is $z$ and the functions $V_1, V_2$ are of the form:


$$V_1(T) = 
\left\{
    \begin{array}{ll}
        \infty & \mbox{if } a(T)\leq \epsilon, \\
        \infty & \mbox{if } R(T)\geq \alpha, \\
        \theta Sur(T) & \mbox{otherwise, }
    \end{array}
\right. \quad \mbox{and } \quad V_2 = 0.$$

Where $a(T)$ is the area of the smallest face of the tetrahedron $T$, $R(T)$ is the circumradius of $T$, and $Sur(T)$ is the surface area of the tetrahedron.

Note that the point weights do not play an explicit role in the energy calculation. However they do change the structure of $Del(\gamma)$ and through that the energy too.




\section{Simulation}
The simulations were done in CGAL~\cite{cgal},~\cite{cgal:3d-triang}.
\subsection{Initial configuration}
\unsure[inline]{If we used the empty outside configuration, why not just start from an entirely empty position? Would that be wrong?}
\unsure[inline]{We are simulating finite volume approximation, how is it connected to the infinite process, exactly?}

In~\cite{DL10}, three options for the initial configuration are suggested: the empty configuration, a specific fixed outside configuration, and periodic configuration.~\cite{DL10} rejects the empty configuration because it "produces non bounded Delaunay-Voronoi cells". While this is true for a Voronoi diagram, it does not hold for the Delaunay case and so such configuration would be possible. 
The method chosen was to fix a regular grid of points in and out of $\Lambda$ such that the resulting tessellation fulfills the hardcore conditions. This does mean that the initial configuration is dependent on the values of $\alpha$ and $\epsilon$. The reason for this choice is simple: the CGAL implementation of 3d periodic triangulations~\cite{cgal:3d-period} \unsure{Yeah, but why not empty?} runs much slower than in the non-periodic case.


\subsection{MCMC}

The algorithm is based on a classic \unsure{Why is it of this form} Birth-Death-Move algorithm from~\cite{moller-waagepetersen:2003}. However, one crucial alteration has to be made. One of the main differences between the Regular and Delaunay triangulation is that in a regular triangulation, points may be redundant and thus not be included in the triangulation. A new point may make other points redundant, or be itself redundant. To remove this effect, \unsure{is this a good approach? The space for adding new points changes each step, is that a problem?} we decided to only add points from a subset $A \subset [0,1]^3\times W$ of those possible points that will not result in any redundant points. \newline
First, start from a permissible initial configuration $\gamma_0$.
\begin{enumerate}
    \item Let $n = \text{card}(\gamma_0 \cap \Lambda)$.
    \item Draw independently $a$ and $b$ uniformly on $[0,1]$.
    \item If $a<1/3$, then generate $x$ uniformly on $A$ and set
        $$\gamma_1 = 
        \left\{
            \begin{array}{ll}
                \gamma_0 \cup \{x\} & \mbox{if }  b < \frac{z f(\gamma_0 \cup \{x\})}{(n+1)f(\gamma_0)}, \\
                \gamma_0 & \mbox{otherwise. }
            \end{array}
        \right. $$
    \item If $a>2/3$, then generate $x$ uniformly on $\gamma_0$ and set
        $$\gamma_1 = 
        \left\{
            \begin{array}{ll}
                \gamma_0 \setminus \{x\} & \mbox{if }  b < \frac{n f(\gamma_0 \setminus \{x\})}{zf(\gamma_0)}, \\
                \gamma_0 & \mbox{otherwise. }
            \end{array}
        \right. $$
    \item If $1/3 < a < 2/3$, then generate $x$ uniformly on $\gamma_0$, generate $y\sim \mathcal N (x, \sigma^2 I)$ such that $y \in A$ and set
        $$\gamma_1 = 
        \left\{
            \begin{array}{ll}
                \gamma_0 \setminus \{x\} \cup \{y\} & \mbox{if }  b < \frac{f(\gamma_0 \setminus \{x\} \cup \{y\})}{f(\gamma_0)}, \\
                \gamma_0 & \mbox{otherwise. }
            \end{array}
        \right. $$
    \item Set $\gamma_0 \leftarrow \gamma_1$ and go to 1.
\end{enumerate}
\problem[inline]{This is not true at the moment, currently the algorithm just skips points in conflict with other points.}

\unsure[inline]{Why does this work? Can we choose different proposal density to improve the convergence?}

If the moved point would fall outside $[0,1]^3$, it is \unsure{Is this a good approach? I makes the density of moved points next to boundary concentrate more in some places} \todoo{Describe this properly}'bounced back' from the boundary of $[0,1]^3$ as if the boundary of the unit box was a solid wall. 
This differs from~\cite{DL10}, where the point would be replaced inside $[0,1]^2$ by the periodic property. 
The idea is that a small perturbation to the point's position should not result in a radically different positionof the point. 


\subsection{Explicit expression for the proposal densities}
\todoo[inline]{This section is poorly described. Notation and some new terms needs to be defined/improved for this to be understandable and precise. Improve after writing a chapter on Delaunay}

Since the energy function is a simple sum over the individual tetrahedra, the density ratios can be further simplified in all three cases. Since all the density ratios in contain both density of $\gamma$ and density of $\gamma$ with point added or removed in a ratio, most terms in the sum will be canceled out. In each step some tetrahedra are removed and some added - the key is in finding which tetrahedra those are. 

When a point $x$ is added into $Del(\gamma)$, the tetrahedra in conflict with $x$ are precisely those that will be deleted. Likewise, the tetrahedra in $Del(\gamma\cup\{x\})$ that are adjacent to $x$ (contain $x$ as one of its vertices) are precisely those that were newly created.

For $x\in \gamma$, denote $\text{DT}^\ell(x,\gamma)$ the set of tetrahedra adjacent to $x$ in $Del(\gamma)$. For $x \notin \gamma$, denote $\text{DT}^\otimes (x,\gamma)$ the set of tetrahedra conflicting with point $x$. 

Birth step:
\begin{align*}
\frac{f(\gamma_0 \cup \{x\})}{f(\gamma_0)} &= \exp\left({\sum_{T\in Del_\Lambda(\gamma_0 \cup \{x\})} V_1(T) - \sum_{T \in Del_\Lambda(\gamma_0)}V_1(T)}\right) \\
&= \exp\left(  \sum_{T \in \text{DT}^\otimes (x,\gamma_0)} V_1(T)  - \sum_{T\in \text{DT}^\ell (x,\gamma_0\cup\{x\})} V_1(T) \right)  
\end{align*}

Death step:
\begin{align*}
\frac{f(\gamma_0 \setminus \{x\})}{f(\gamma_0)}&= \exp\left({\sum_{T\in Del_\Lambda(\gamma \setminus \{x\})} V_1(T)- \sum_{T \in Del_\Lambda(\gamma)}V_1(T)}\right)\\
&= \exp\left( \sum_{T\in \text{DT}^\ell (x,\gamma_0)} V_1(T) - \sum_{T \in \text{DT}^\otimes (x,\gamma_0 \setminus \{x\})} V_1(T)   \right)
\end{align*}

Move step:
\begin{align*}
\frac{f(\gamma_0 \setminus \{x\} \cup \{y\})}{f(\gamma_0)}&= 
\frac{f(\gamma_0 \setminus \{x\} \cup \{y\})}{f(\gamma_0\setminus\{x\})} \frac{f(\gamma_0\setminus\{x\})}{f(\gamma_0)} \\ 
&= \exp \Bigg(  \sum_{T \in \text{DT}^\otimes (x,\gamma_0\setminus\{x\})} V_1(T)  - \sum_{T\in \text{DT}^\ell (x,\gamma_0\setminus\{x\}\cup\{y\})} V_1(T)  \\
&+ \sum_{T\in \text{DT}^\ell (x,\gamma_0)} V_1(T) - \sum_{T \in \text{DT}^\otimes (x,\gamma_0\setminus\{x\})} V_1(T)   
\Bigg)
\end{align*}



These expressions simplify the energy calculation immensely. Whereas calculating the energy for the whole tessellation requires all the tetrahedra, and thus depends on $\text{card}(\gamma\cap\Lambda)$, the final expressions only contain the tetrahedra local to $x$, and thus the energy can be calculated in constant time.



\section{Estimation}

Assume now that we obtain the point configuration $\gamma$ on the observation window $\Lambda_n = [-n,n]^3\times W$ and wish to estimate the model parameters.
\unsure[inline]{Boundary problems. Do they simply exist because we're assuming to *only* know the configuration on $\Lambda_n$?}
The estimation procedure closely follows that from~\cite{DL10}. That is a two-step approach, first estimating the hardcore parameters $\beta = (\epsilon,\alpha)$ and then using the estimates to obtain the estimate of $\theta$ through maximum pseudolikelihood (MPLE). 
\unsure[inline]{What exactly is the role of the growing window in~\cite{DL10}?}
\subsection{Estimation of the hardcore parameters}
Thanks to the fact that the hardcore parameter $\epsilon$ satisfies
$$ \text{if } \epsilon < \epsilon' \text{ then  } \forall \Lambda, \; E^{\epsilon, \alpha,\theta}_\Lambda(\gamma_\Lambda,\gamma_{\Lambda^c}) < \infty \Rightarrow  E^{\epsilon',\alpha,\theta}_\Lambda(\gamma_\Lambda,\gamma_{\Lambda^c})<\infty,$$ 

and the hardcore parameter $\alpha$ satisfies

$$ \text{if } \alpha > \alpha' \text{ then  } \forall \Lambda, \; E^{\epsilon,\alpha,\theta}_\Lambda(\gamma_\Lambda,\gamma_{\Lambda^c}) < \infty \Rightarrow  E^{\epsilon,\alpha',\theta}_\Lambda(\gamma_\Lambda,\gamma_{\Lambda^c})<\infty,$$ 

their \unsure{Why are the consistent}consistent estimators are:
$$\hat\epsilon = \inf\{\epsilon > 0, E_\Lambda(\gamma_\Lambda, \gamma_\Lambda^c) < \infty \},$$
$$\hat\alpha = \sup\{\alpha > 0, E_\Lambda(\gamma_\Lambda, \gamma_\Lambda^c) < \infty \}.$$

\unsure{Are these consistent? Why?}In practice, the parameters are estimated as
$$\hat\epsilon = \min\{a(T), T\in Del_\Lambda(\gamma)\},$$
$$\hat\alpha = \max\{r(T), T\in Del_\Lambda(\gamma)\}.$$

The estimate $\hat\beta = (\hat\epsilon,\hat\alpha)$ is then used in the pseudo-likelihood function in the second estimation step.


\subsection{Estimation of the smooth interaction parameters}
\todoo[inline]{Equation references}


The classical version of MPLE requires hereditarity of the interactions. Hereditarity means that for every permissible $\gamma$, the point pattern $\gamma\setminus\{x\}$ remains permissible for every $x\in\gamma$, that is any point can be removed from the point pattern. The hardcore interaction in our model does not satisfy this condition. However,~\cite{DL07} \unsure{How does it relate to my case, exactly?}extends MPLE to the non-hereditary case. 

Since some points cannot be removed from the tessellation, we need to introduce the notion of a removable points. A point $x\in\gamma$ is \note{The definition in~\cite{DL07} is actually different and this is given as a proposition.}\unsure{Is there any reason to define it the way~\cite{DL07} does?}removable in $\gamma$ iff $\gamma\setminus\{x\}$ is permissible. We denote $\mathcal R^\beta(\gamma)$ the set of removable points of $\gamma$. Similarly the notion of an addable point will be useful. A point $x\in\gamma$ is addable in $\gamma$ iff $\gamma \cup \{x\}$ is permissible.


In the non-hereditary case, the pseudo-likelihood function then becomes

$$PLL_{\Lambda_n}(\gamma,z,\beta, \theta) = \int_{\Lambda'_n} z \exp (-h^{\beta,\theta}(x,\gamma)) dx + \sum_{x\in\mathcal R^\beta(\gamma)\cap \Lambda_n} \big(h^{\beta,\theta}(x,\gamma\setminus\{x\}) - \ln(z)\big),$$
where $\Lambda'_n$ is the set of all addable points in $\Lambda_n$ and $h^{\beta,\theta}(x, \gamma \setminus \{x\})$ is local energy of $x$ in $\gamma$ defined for every $x\in\mathcal R^\beta(\gamma)$ by:
$$h^{\beta,\theta}(x, \gamma \setminus \{x\}) = E^{\beta,\theta}_\Lambda(\gamma_\Lambda, \gamma_{\Lambda^c}) - E^{\beta,\theta}_\Lambda(\gamma_\Lambda\setminus\{x\}, \gamma_{\Lambda^c}).$$

The estimates $\hat\theta$ and $\hat z$ are obtained through minimizing the $PLL_{\Lambda_n}$ function:
$$(\hat z, \hat\theta) = \text{argmin}_{z,\theta} PLL_{\Lambda_n} (\gamma, z, \hat\beta,\theta).$$

By differentiating the expressions w.r.t. $z$, respectivelly $\theta$, and setting them equal to zero, we obtain the estimate $\hat z$:

$$ \hat z = \frac{\mbox{card}(\mathcal R^\beta(\gamma)\cap \Lambda_n)}{\int_{\Lambda_n} \exp{\left( -h^{\hat\beta,\theta}(x,\gamma)\right)} dx},$$

and the estimate $\hat\theta$ as the solution of

$$ z \int_{\Lambda'_n} (h^{\hat\beta,1}(x,\gamma)\exp{\left(-h^{\hat\beta,\theta}(x,\gamma)\right)}) dx = \sum_{x \in \mathcal R^{\hat\beta}(\gamma)\cap \Lambda_n} h^{\hat\beta,1}(x,\gamma\setminus\{x\}),$$

where we have used the fact that the local energy depends on $\theta$ linearly, yielding

$$\frac{\partial h^{\hat\beta,\theta}}{\partial \theta} (x,\gamma) = h^{\hat\beta,1}(x,\gamma).$$

\subsubsection{Practical implementation}
We obtain the estimate of $\theta$ by substituting $z$ by its estimate $\hat z$.
This leads to the equation
$$ 
\frac{\int_{\Lambda'_n} (h^{\hat\beta,1}(x,\gamma)\exp{\left(-h^{\hat\beta,\theta}(x,\gamma)\right)}) dx} {  \int_{\Lambda_n} \exp{\left( -h^{\hat\beta,\theta}(x,\gamma)\right)} dx} 
= \frac {\sum_{x \in \mathcal R^{\hat\beta}(\gamma)\cap \Lambda_n} h^{\hat\beta,1}(x,\gamma\setminus\{x\})} { \mbox{card}(\mathcal R^\beta(\gamma)\cap \Lambda_n) }. 
$$

In order to simplify the estimation of $\theta$, we can simplify this equation further. First, we denote the righ-hand-side of the equation as $c$ as it  is constant with respect to $\theta$. Second, we note that $x \notin \Lambda_n'  \Rightarrow \exp{\left(-h^{\hat\beta,\theta}(x,\gamma)\right)}= 0$ which enables us to integrate over $\Lambda'_n$ instead of the whole $\Lambda_n$. Lastly we denote the local energy $h^{\hat\beta,1}(x,\gamma) =: h(x)$, yielding the expression
$$ \int_{\Lambda'_n} h(x) \exp{\left(-\theta h(x)\right)} dx = c \int_{\Lambda'_n} \exp{\left(-\theta h(x)\right)}, $$
leading into the final expression
$$ \int_{\Lambda'_n} \exp{\left(-\theta h(x)\right)} (h(x) - c) dx .$$ 

The integral is estimated using Monte-Carlo integration, i.e. is approximately equal to
$$ \frac 1N \sum_{i=0}^N 1_{\Lambda'_n}(x_i) \exp{\left( - \theta h_i \right )} (h_i- c) dx $$
where $h_i = h^{\hat\beta,1}(x_i, \gamma)$ and $x_1,\dots,x_N$ is a random sample from the \unsure{Do we need the indicator function if we're only sampling from $\Lambda'_n$ ?}uniform distribution on $\Lambda'_n$

After $\hat\theta$ is estimated, we then obtain the estimate $\hat z$ with $\hat\theta$ instead of $\theta$ and the integral replaced by a MC-integration approximation.

%%%%%%%%%%%%%%%%%%%%%%%%%%%%%%%%%%%%%%%%

\bibliography{bibliography}{}
\bibliographystyle{plain}


\listoftodos

\section*{List of Abbreviations}
\begin{enumerate}
    \item $\mathcal B_b(X)$ bounded Borel sets in the space $X$
    \item $\text{card}(A)$ cardinality of the set $A$
\end{enumerate}

\end{document}
